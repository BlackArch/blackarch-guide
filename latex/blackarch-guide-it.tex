%%%%%%%%%%%%%%%%%%%%%%%%%%%%%%%%%%%%%%%%%%%%%%%%%%%%%%%%%%%%%%%%%%%%%%%%%%%%%%%%
%                                                                              %
% BlackArch Linux Guide                                                        %
%                                                                              %
%%%%%%%%%%%%%%%%%%%%%%%%%%%%%%%%%%%%%%%%%%%%%%%%%%%%%%%%%%%%%%%%%%%%%%%%%%%%%%%%

\documentclass[a4paper, oneside, 11pt]{book}

%%% INCLUDES %%%
\renewcommand{\familydefault}{\sfdefault}

\usepackage{array}
\usepackage{color}
\usepackage{enumerate}
\usepackage{fancyhdr}
\usepackage{fancyvrb}
\usepackage{geometry}
\usepackage{graphicx}
\usepackage{html}
\usepackage{hyperref}
\usepackage{ifpdf}
\usepackage{listings}
\usepackage{pstricks}
\usepackage{supertabular}
\usepackage{tocloft}
\usepackage[utf8]{inputenc}

%%% LAYOUT %%%
\setlength{\parindent}{0em}
\setlength{\parskip}{1.5ex plus0.5ex minus0.5ex}
\geometry{left=2.5cm, textwidth=16cm, top=3cm, textheight=25cm, bottom=3cm}
\widowpenalty=2000
\clubpenalty=1000
\frenchspacing
\sloppy
\pagecolor[HTML]{FFFFFF}
\color[HTML]{333333}
\setcounter{tocdepth}{10}
\setcounter{secnumdepth}{10}

\hypersetup{
  pdftitle={BlackArch Linux, Guida a BlackArch Linux},
  pdfsubject={BlackArch Linux, Guida a BlackArch Linux},
  pdfauthor={BlackArch Linux, BlackArch Linux},
  pdfkeywords={BlackArch Linux, Penetration Testing, Security, Hacking, Linux},
  pdfcenterwindow=true,
  colorlinks=true,
  breaklinks=true,
  linkcolor=red,
  menucolor=red,
  urlcolor=red
}

% syntax highlighting
% all options should be set here document wide
\lstset{
backgroundcolor=\color[HTML]{EEEEEE},
frame=single,
basicstyle=\footnotesize\ttfamily,
float,
deletekeywords={return},
otherkeywords={mkdir, curl, sudo, sha1sum, grep, cut, sort, wget, makepkg,
pacman, blackman},
keywordstyle=\color{orange},
commentstyle=\color{blue},
stringstyle=\color{red},
language=bash,
showspaces=false,
showtabs=false,
tabsize=2
}

%%% HEADER / FOOTER %%%
\setlength{\headheight}{33pt}
\setlength{\headsep}{33pt}
\lhead{{\includegraphics[width=1cm,height=1cm]{images/logo.png}}}
\rhead{Guida a BlackArch Linux}

%%% CUSTOM MACROS %%%
% for html links
\ifpdf\else
\def\href#1#2{\htmladdnormallink{#2}{#1}}
\fi

%------------------%
%  TITLE PAGE      %
%------------------%
\begin{document}
\pagestyle{empty}
\begin{center}
\begin{figure}[htbp]
\centering
\vspace{0.5cm}
\includegraphics[width=8cm]{images/logo.png}
\label{fig:logo}
\end{figure}
\vspace{0.5cm}
\Huge{\textbf{Guida a BlackArch Linux}}\\
\vspace{1cm}
\Large{\color{red}https://www.blackarch.org/}\\
\vspace{0.5cm}
\end{center}
\newpage
\tableofcontents
\newpage
\pagestyle{fancy}

%------------------%
%  Chapter 1       %
%------------------%

\chapter{Introduzione}

\section{Panoramica}
La guida a BlackArch Linux è divisa in alcune parti:
\begin{itemize}
\item Introduzione - Fornisce un'ampia panoramica, presentazione ed altre informazioni utili sul progetto
\item Guida utente - Tutto ciò di cui un utente tipico necessita per utilizzare al meglio BlackArch
\item Guida sviluppatore - Come iniziare a sviluppare e contribuire a BlackArch
\item Guida strumenti - Dettagli approfonditi con esempi di utilizzo degli strumenti
\end{itemize}

\section{Che Cosa è BlackArch Linux?}
BlackArch è una distribuzione Linux completa per pentesters e ricercatori di sicurezza.
Deriva da \href{https://www.archlinux.org/}{ArchLinux} e gli utenti possono installare i componenti di BlackArch
individualmente o in gruppo direttamente su di essa.

L'insieme degli strumenti è distribuito come un
\href{https://wiki.archlinux.org/index.php/Unofficial\_User\_Repositories}
{repository utente non ufficiale} per ArchLinux, quindi puoi installare BlackArch sopra ad un'installazione di ArchLinux già esistente. I pacchetti possono essere installati individualmente o per categoria.

Il repository in costante espansione include al momento oltre \href{https://www.blackarch.org/tools.html}{1300} strumenti.
Tutti gli strumenti vengono testati a fondo prima di essere aggiunti, in maniera da mantenere la qualità del repository.
% should quickly describe the testing methods/code review procedures etc

\section{Storia di BlackArch Linux}
Prossimamente...

\section{Piattaforme Supportate}
Prossimamente...

\section{Partecipa}
Puoi metterti in contatto con il team di BlackArch attraverso i seguenti metodi:

Sito web: \url{https://www.blackarch.org/}

Mail: \href{mailto:team@blackarch.org}{team@blackarch.org}

IRC: \url{irc://irc.freenode.net/blackarch}

Twitter: \url{https://twitter.com/blackarchlinux}

Github: \url{https://github.com/Blackarch/}

%------------------%
%  Chapter 2       %
%------------------%


\chapter{Guida dell'utente}

\section{Installazione}
Le sezioni seguenti ti mostreranno come impostare il repository BlackArch ed installare i pacchetti. BlackArch supporta l'installazione sia dal repository utilizzando i pacchetti binari, sia compilando ed installando i sorgenti.

BlackArch è compatibile con le normali installazioni Arch. Si comporta come un repository utente non ufficiale. Se invece vuoi un'ISO, guarda la sezione
\href{https://www.blackarch.org/downloads.html#iso}{Live ISO}.

\subsection{Installazione su ArchLinux}
Esegui \href{https://blackarch.org/strap.sh}{strap.sh} come root e segui le istruzioni. Vedi l'esempio seguente.
\begin{lstlisting}
   curl -O https://blackarch.org/strap.sh
   sha1sum strap.sh # should match: 86eb4efb68918dbfdd1e22862a48fda20a8145ff
   sudo ./strap.sh
\end{lstlisting}

Ora scarica una copia aggiornata della lista principale dei pacchetti e sincronizza i pacchetti:
\begin{lstlisting}
  sudo pacman -Syyu
\end{lstlisting}


\subsection{Installare i Pacchetti}
Ora puoi installare gli strumenti dal repository blackarch.
\begin{enumerate}
\item Per elencare tutti gli strumenti disponibili, esegui
\begin{lstlisting}
  pacman -Sgg | grep blackarch | cut -d' ' -f2 | sort -u
\end{lstlisting}

\item Per installare tutti gli strumenti, esegui
\begin{lstlisting}
  pacman -S blackarch
\end{lstlisting}

\item Per installare una categoria di strumenti, esegui
\begin{lstlisting}
  pacman -S blackarch-<category>
\end{lstlisting}

\item Per elencare le categorie di blackarch, esegui
\begin{lstlisting}
  pacman -Sg | grep blackarch
\end{lstlisting}

\end{enumerate}

\subsection{Installare i Pacchetti da Sorgente}
Come parte di un metodo alternativo di installazione, puoi generare i pacchetti BlackArch da sorgente. Puoi trovare i PKGBUILDs su
\href{https://github.com/BlackArch/blackarch/tree/master/packages}{github}. Per
generare l'intero repo, puoi utilizzare lo strumento
\href{https://github.com/BlackArch/blackman}{Blackman}.
\begin{itemize}
\item Come prima cosa, devi installare Blackman. Se il repository BlackArch è impostato sulla tua macchina, puoi installare Blackman:
\begin{lstlisting}
  pacman -S blackman
\end{lstlisting}

\item Puoi compilare ed installare Blackman dal sorgente:
\begin{lstlisting}
  mkdir blackman
  cd blackman
  wget https://raw2.github.com/BlackArch/blackarch/master/packages/blackman/PKGBUILD
  # Make sure the PKGBUILD has not been maliciously tampered with.
  makepkg -s
\end{lstlisting}

\item Oppure puoi installare Blackman da AUR:
\begin{lstlisting}
  <whatever AUR helper you use> -S blackman
\end{lstlisting}

\end{itemize}

\subsection{Utilizzo Base di Blackman} Blackman è molto semplice da utilizzare, sebbene i flags siano differenti da quello che ti aspetteresti da qualcosa simile a pacman. L'utilizzo base viene descritto nel seguito.
\begin{itemize}
\item Scarica, compila e installa i pacchetti:
\begin{lstlisting}
  sudo blackman -i package
\end{lstlisting}

\item Scarica, compila e installa un'intera categoria:
\begin{lstlisting}
  sudo blackman -g group
\end{lstlisting}

\item Scarica, compila e installa tutti gli strumenti di BlackArch:
\begin{lstlisting}
  sudo blackman -a
\end{lstlisting}

\item Per elencare le categorie di BlackArch:
\begin{lstlisting}
  blackman -l
\end{lstlisting}

\item Per elencare gli strumenti di una categoria:
\begin{lstlisting}
  blackman -p category
\end{lstlisting}

\end{itemize}

\subsection{Installazione da live-, netinstall- ISO or ArchLinux}
Puoi installare BlackArch Linux da una delle nostre live- o netinstall-ISOs.Vedi
\url{https://www.blackarch.org/download.html#iso}. Dopo l'avvio dell'ISO sono richiesti i seguenti passaggi.
\begin{itemize}
\item Installa il pacchetto blackarch-installer:
\begin{lstlisting}
  sudo pacman -S blackarch-installer
\end{lstlisting}

\item Esegui
\begin{lstlisting}
  sudo blackarch-install
\end{lstlisting}

\end{itemize}

%------------------%
%  Chapter 3       %
%------------------%

\chapter{Guida Sviluppatore}

\section{Build System e Repositories di Arch}

I files PKGBUILD sono dei build scripts. Ognuno dice a makepkg(1) come creare un pacchetto. I files PKGBUILD sono scritti in Bash.

Per altre informazioni, leggi (o sfoglia) le seguenti pagine:
\begin{itemize}
\item \href{https://wiki.archlinux.org/index.php/Creating_Packages}{Arch Wiki: Creating Packages}
\item \href{https://wiki.archlinux.org/index.php/Makepkg}{Arch Wiki: makepkg}
\item \href{https://wiki.archlinux.org/index.php/PKGBUILD}{Arch Wiki: PKGBUILD}
\item \href{https://wiki.archlinux.org/index.php/Arch_Packaging_Standards}{Arch Wiki: Arch Packaging Standards}
\end{itemize}

\section{Blackarch PKGBUILD standards}
Per semplicità, i nostri PKGBUILDs sono simili a quelli di AUR,
con qualche piccola differenza evidenziata di seguito. Ogni pacchetto deve appartenere almeno a blackarch, ci saranno anche molte sovrapposizioni tra vari pacchetti appartenenti a più gruppi.

\subsection{Gruppi}
Per permettere agli utenti di installare velocemente e facilmente uno specifico insieme di pacchetti,
questi sono stati separati in gruppi. I gruppi permettono agli utenti di dare semplicemente
"pacman -S <group name>" per installare tanti pacchetti.

\subsubsection{blackarch}
Il gruppo blackarch è il gruppo base a cui tutti i pacchetti devono appartenere. Questo consente
agli utenti di installare tutti i pacchetti con facilità.

Cosa dovrebbe stare qui: tutto.

\subsubsection{blackarch-anti-forensic}
Pacchetti che sono usati per contrastare le attività forensi,
includono crittografia, steganografia, e qualsiasi cosa che modifichi i files o i loro attributi.
Questo vale per tutti gli strumenti che lavorano con qualsiasi cosa in generale che modifichi un sistema per nascondere informazioni.

Ad esempio: luks, TrueCrypt, Timestomp, dd, ropeadope, secure-delete

\subsubsection{blackarch-automation}
Pacchetti che sono usati per l'automazione del flusso di lavoro.

Ad esempio: blueranger, tiger, wiffy

\subsubsection{blackarch-backdoor}
Pacchetti che sfruttano o aprono backdoors su sistemi già vulnerabili.

Ad esempio: backdoor-factory, rrs, weevely

\subsubsection{blackarch-binary}
Pacchetti che in qualche modo operano su file binari.

Ad esempio: binwally, packerid

\subsubsection{blackarch-bluetooth}
Pacchetti che sfruttano qualsiasi cosa riguardi lo standard Bluetooth (802.15.1).

Ad esempio: ubertooth, tbear, redfang

\subsubsection{blackarch-code-audit}
Pacchetti che effettuano la revisione di codice sorgente per analisi di vulnerabilità.

Ad esempio: flawfinder, pscan

\subsubsection{blackarch-cracker}
Pacchetti che sono usati per il cracking di funzioni crittografiche, ie hashes.

Ad esempio: hashcat, john, crunch

\subsubsection{blackarch-crypto}
Pacchetti che lavorano con la crittografia, esclusi quelli per il cracking.

Ad esempio: ciphertest, xortool, sbd

\subsubsection{blackarch-database}
Pacchetti che comportano l'exploitation a qualsiasi livello di databases.

Ad esempio: metacoretex, blindsql

\subsubsection{blackarch-debugger}
Pacchetti che permettono all'utente di vedere in realtime cosa sta facendo un particolare programma.

Ad esempio: radare2, shellnoob

\subsubsection{blackarch-decompiler}
Pacchetti che cercano di ricavare il codice sorgente da un programma compilato.

Ad esempio: flasm, jd-gui

\subsubsection{blackarch-defensive}
Pacchetti che sono utilizzati per proteggere un utente da attacchi di malware da parte di altri utenti.

Ad esempio: arpon, chkrootkit, sniffjoke

\subsubsection{blackarch-disassembler}
Questo è simile a blackarch-decompiler, e probabilmente ci saranno molti programmi che ricadranno in entrambe le categorie, tuttavia questi pacchetti producono l'output in assembly
anziché in codice sorgente di alto livello.

Ad esempio: inguma, radare2

\subsubsection{blackarch-dos}
Pacchetti che utilizzano attacchi DoS (Denial of Service).

Ad esempio: 42zip, nkiller2

\subsubsection{blackarch-drone}
Pacchetti che sono utilizzati per gestire droni.

Ad esempio: meshdeck, skyjack

\subsubsection{blackarch-exploitation}
Pacchetti che approfittano di exploits in altri programmi o servizi.

Ad esempio: armitage, metasploit, zarp

\subsubsection{blackarch-fingerprint}
Pacchetti per fare fingerprinting.

Ad esempio: dns-map, p0f, httprint

\subsubsection{blackarch-firmware}
Pacchetti che sfruttano vulnerabilità nel firmware.

Ad esempio: None yet, amend asap.

\subsubsection{blackarch-forensic}
Pacchetti che sono utilizzati per trovare dati su dischi fisici o memorie embedded.

Ad esempio: aesfix, nfex, wyd

\subsubsection{blackarch-fuzzer}
Pacchetti che utilizzano il principio di testing del fuzzing, ie
inviano degli inputs casuali per vedere cosa succede.

Ad esempio: msf, mdk3, wfuzz

\subsubsection{blackarch-hardware}
Pacchetti che sfruttano o gestiscono qualsiasi cosa che abbia a che fare con l'hardware fisico.

Ad esempio: arduino, smali

\subsubsection{blackarch-honeypot}
Pacchetti che agiscono da "honeypots", ie programmi che sembrano servizi vulnerabili, utilizzati per trarre in trappola gli hackers.

Ad esempio: artillery, bluepot, wifi-honey

\subsubsection{blackarch-keylogger}
Pacchetti che registrano e conservano le battiture su un altro sistema.

Ad esempio: None yet, amend asap.

\subsubsection{blackarch-malware}
Pacchetti che contano come qualsiasi tipo di software malevolo o malware detection.

Ad esempio: malwaredetect, peepdf, yara

\subsubsection{blackarch-misc}
Pacchetti che non rientrano specificamente in nessuna categoria.

Ad esempio: oh-my-zsh-git, winexe, stompy

\subsubsection{blackarch-mobile}
Pacchetti che manipolano piattaforme mobile.

Ad esempio: android-sdk-platform-tools, android-udev-rules

\subsubsection{blackarch-networking}
Pacchetti che coinvolgono l'IP networking.

Ad esempio: più o meno tutto

\subsubsection{blackarch-nfc}
Pacchetti che usano nfc (near-field communications).

Ad esempio: nfcutils

\subsubsection{blackarch-packer}
Pacchetti che operano su o coinvolgono packers.

\textit{i packers sono programmi che incorporano del malware dentro altri eseguibili}

Ad esempio: packerid

\subsubsection{blackarch-proxy}
Pacchetti che agiscono come proxy, ie redirigendo il traffico attraverso un altro nodo su internet

Ad esempio: burpsuite, ratproxy, sslnuke

\subsubsection{blackarch-recon}
Pacchetti che cercano attivamente vulnerabilità da exploitare in the wild. Più di un gruppo ombrello per pacchetti simili.

Ad esempio: canri, dnsrecon, netmask

\subsubsection{blackarch-reversing}
Questo è un gruppo ombrello per qualsiasi decompilatore, disassemblatore o ogni programma simile.

Ad esempio: capstone, radare2, zerowine

\subsubsection{blackarch-scanner}
Pacchetti che analizzano alla ricerca di vulnerabilità i sistemi selezionati.

Ad esempio: scanssh, tiger, zmap

\subsubsection{blackarch-sniffer}
Pacchetti che coinvolgono l'analisi del traffico di rete.

Ad esempio: hexinject, pytactle, xspy

\subsubsection{blackarch-social}
Pacchetti che attaccano principalmente siti di social networking.

Ad esempio: jigsaw, websploit

\subsubsection{blackarch-spoof}
Pacchetti che provano a falsificare l'attaccante, in modo da non apparire come tale alla vittima.

Ad esempio: arpoison, lans, netcommander

\subsubsection{blackarch-threat-model}
Pacchetti che vengono usati per scrivere report sul threat model delineatosi in un particolare scenario.

Ad esempio: magictree

\subsubsection{blackarch-tunnel}
Pacchetti che sono utilizzati per fare tunneling del traffico di rete.

Ad esempio: ctunnel, iodine, ptunnel

\subsubsection{blackarch-unpacker}
Pacchetti che sono utilizzati per estrarre da un eseguibile del malware precedentemente incorporato.

Ad esempio: js-beautify

\subsubsection{blackarch-voip}
Pacchetti che operano su programmi e protocolli voip.

Ad esempio: iaxflood, rtp-flood, teardown

\subsubsection{blackarch-webapp}
Pacchetti che operano su applicazioni internet-facing.

Ad esempio: metoscan, whatweb, zaproxy

\subsubsection{blackarch-windows}
Questo gruppo è per qualsiasi pacchetto nativo Windows che viene eseguito via wine.

Ad esempio: 3proxy-win32, pwdump, winexe

\subsubsection{blackarch-wireless}
Pacchetti che operano su reti wireless a qualsiasi livello.

Ad esempio: airpwn, mdk3, wiffy

\section{Struttura del Repository}
Puoi trovare il principale git repo di BlackArch qui:
\href{https://github.com/BlackArch/blackarch}{https://github.com/BlackArch/blackarch}.
Ci sono anche alcuni repos secondari qui:
\href{https://github.com/BlackArch}{https://github.com/BlackArch}.

All'interno del principale git repo, ci sono tre importanti direttori:

\begin{itemize}
\item docs - Documentazione.
\item packages - PKGBUILD files.
\item scripts - Piccoli utili scripts.
\end{itemize}

\subsection{Scripts}
Qui c'è un riferimento per gli scripts nel direttorio \verb|scripts/| :

\begin{itemize}
\item baaur - Presto questo caricherà i pacchetti su AUR.
\item babuild - Genera un pacchetto.
\item bachroot - Gestisce un chroot per testing.
\item baclean - Pulisce i vecchi files .pkg.tar.xz dal repo dei pacchetti.
\item baconflict - Presto questo rimpiazzerà \verb|scripts/conflicts|.
\item bad-files - Trova files corrotti nei pacchetti generati.
\item balock - Ottiene o rilascia il lock del repo dei pacchetti.
\item banotify - Notifica IRC riguardo ai push dei pacchetti.
\item barelease - Rilascia i pacchetti al repo.
\item baright - Stampa le info sul copyright di BlackArch.
\item basign - Firma i pacchetti.
\item basign-key - Firma una chiave.
\item blackman - Questo si comporta tipo pacman ma compila da git (da non confondere con Blackman di nrz).
\item check-groups - Controlla i gruppi.
\item checkpkgs - Controlla gli errori nei pacchetti.
\item conflicts - Controlla i conflitti tra file.
\item dbmod - Modifica il database di un pacchetto.
\item depth-list - Crea una lista ordinata per profondità delle dipendenze.
\item deptree - Crea un albero delle dipendenze, elencando solo i pacchetti forniti da blackarch.
\item get-blackarch-deps - Ottiene una lista di dipendenze blackarch per un pacchetto.
\item get-official - Ottiene i pacchetti ufficiali per il rilascio.
\item list-loose-packages - Elenca i pacchetti che non sono nei gruppi e non sono
    dipendenze per altri pacchetti.
\item list-needed - Elenca le dipendenze mancanti.
\item list-removed - Elenca i pacchetti che sono nel repo dei pacchetti ma non in git.
\item list-tools - Elenca gli strumenti.
\item outdated - Cerca i pacchetti out-dated nel repo dei pacchetti
    rispetto al git repo.
\item pkgmod - Modifica un build package.
\item pkgrel - Incrementa pkgrel in un pacchetto.
\item prep - Ripulisce lo stile di un file PKGBUILD e trova gli errori.
\item sitesync - Effettua la sincronizzazione tra una copia locale del repo dei pacchetti e una copia remota.
\item size-hunt - Cerca i pacchetti grandi.
\item source-backup - Effettua il backup dei files sorgente di un pacchetto.
\end{itemize}

\section{Contribuire ad un Repository}
Questa sezione ti mostra come contribuire al progetto BlackArch Linux. Accettiamo pull requests di qualsiasi dimensione, da piccole correzioni di battitura a nuovi pacchetti.\\Per
aiuto, suggerimenti, o domande sentiti libero di contattarci.
\\\\
Tutti sono invitati a contribuire. Tutti i contributi sono apprezzati.

\subsection{Tutorials Richiesti}
Per favore leggete i seguenti tutorials prima di contribuire:
\begin{itemize}
\item
\href{https://wiki.archlinux.org/index.php/Arch\_Packaging\_Standards)}{Arch
Packaging Standards}
\item \href{https://wiki.archlinux.org/index.php/Creating\_Packages}{Creating
Packages}
\item \href{https://wiki.archlinux.org/index.php/PKGBUILD}{PKGBUILD}
\item \href{https://wiki.archlinux.org/index.php/Makepkg}{Makepkg}
\end{itemize}

\subsection{Passaggi per Contribuire}
Per inviare i tuoi cambiamnti al progetto BlackArchLinux segui i seguenti passaggi:
\begin{enumerate}
\item Forka il repository da
\url{https://github.com/BlackArch/blackarch}
\item Modifica i files necessari, (e.g. PKGBUILD, .patch files, etc).
\item Effettua il commit delle tue modifiche.
\item Effettua il push delle tue modifiche.
\item Richiedici di includere le tue modifiche, preferibilmente attraverso una pull request.
\end{enumerate}

\subsection{Esempio}
L'esempio successivo dimostra come proporre un nuovo pacchetto al progetto BlackArch. Utilizziamo \href{https://wiki.archlinux.org/index.php/yaourt}{yaourt}
(you can use pacaur as well) per ottenere un file PKGBUILD pre-esistente per
\textbf{nfsshell} da \href{https://aur.archlinux.org/}{AUR} e modifichiamolo secondo i nostri bisogni.

\subsubsection{Ottenere il PKGBUILD}
Ottieni il file \textit{PKGBUILD} utilizzando yaourt or pacaur:
\begin{lstlisting}
  user@blackarchlinux $ yaourt -G nfsshell
  ==> Download nfsshell sources
  x LICENSE
  x PKGBUILD
  x gcc.patch
  user@blackarchlinux $ cd nfsshell/
\end{lstlisting}

\subsubsection{Ripulire il PKGBUILD}
Ripulisci il file \textit{PKGBUILD} e risparmia un po' di tempo:
\begin{lstlisting}
  user@blackarchlinux nfsshell $ ./blarckarch/scripts/prep PKGBUILD
  cleaning 'PKGBUILD'...
  expanding tabs...
  removing vim modeline...
  removing id comment...
  removing contributor and maintainer comments...
  squeezing extra blank lines...
  removing '|| return'...
  removing leading blank line...
  removing $pkgname...
  removing trailing whitespace...
\end{lstlisting}

\subsubsection{Adjust PKGBUILD}
Modifica il file \textit{PKGBUILD}:
\begin{lstlisting}
  user@blackarchlinux nfsshell $ vi PKGBUILD
\end{lstlisting}

\subsubsection{Genera il Pacchetto}
Genera il pacchetto:
\begin{lstlisting}user@blackarchlinux nfsshell $ makepkg -sf
==> Making package: nfsshell 19980519-1 (Mon Dec  2 17:23:51 CET 2013)
==> Checking runtime dependencies...
==> Checking buildtime dependencies...
==> Retrieving sources...
-> Downloading nfsshell.tar.gz...
% Total    % Received % Xferd  Average Speed   Time    Time     Time
CurrentDload  Upload   Total   Spent    Left  Speed100 29213  100 29213    0
0  48150      0 --:--:-- --:--:-- --:--:-- 48206
-> Found gcc.patch
-> Found LICENSE
...
<lots of build process and compiler output here>
...
==> Leaving fakeroot environment.
==> Finished making: nfsshell 19980519-1 (Mon Dec  2 17:23:53 CET 2013)
\end{lstlisting}

\subsubsection{Installa e testa il pacchetto}
Installa e testa il pacchetto:
\begin{lstlisting}
  user@blackarchlinux nfsshell $ pacman -U nfsshell-19980519-1-x86_64.pkg.tar.xz
  user@blackarchlinux nfsshell $ nfsshell # test it
\end{lstlisting}

\subsubsection{Aggiungi, esegui il commit e il push del pacchetto}
Aggiungi, esegui il commit e il push del pacchetto
\begin{lstlisting}user@blackarchlinux nfsshell $ cd /blackarchlinux/packages
user@blackarchlinux ~/blackarchlinux/packages $ mv ~/nfsshell .
user@blackarchlinux ~/blackarchlinux/packages $ git commit -am nfsshell && git push
\end{lstlisting}

\subsubsection{Crea una pull request}
Crea una pull request su \href{https://github.com/}{github.com}
\begin{lstlisting}
  firefox https://github.com/<contributor>/blackarchlinux
\end{lstlisting}

\subsubsection{Aggiungere un Remoto per Upstream}
Una cosa utile da fare se stai lavorando upstream e su un fork è di fare il pull del tuo fork e aggiungerlo al repo ba principale come remoto.

\begin{lstlisting}
  user@blackarchlinux ~/blackarchlinux $ git remote -v
  origin <the url of your fork> (fetch)
  origin <the url of your fork> (push)
  user@blackarchlinux ~/blackarchlinux $ git remote add upstream https://github.com/blackarch/blackarch
  user@blackarchlinux ~/blackarchlinux $ git remote -v
  origin <the url of your fork> (fetch)
  origin <the url of your fork> (push)
  upstream https://github.com/blackarch/blackarch (fetch)
  upstream https://github.com/blackarch/blackarch (push)
\end{lstlisting}

Per default, git dovrebbe fare il push direttamente all'origine, ma assicurati che la tua configurazione di git sia corretta.
Questo non sarà un problema a meno che tu abbia i diritti di commit dato che non sarai in grado di effettuare il push upstream senza di essi.

Se hai la capacità di fare commit, potresti avere più successo usando
\textit{git@github.com:blackarch/blackarch.git} ma dipende da te.

\subsection{Richieste}
\begin{enumerate}
\item Non aggiungere commenti \textbf{Maintainer} o \textbf{Contributor} ai files
\textit{PKGBUILD}. Aggiungi i nomi di maintainer e contributor alla sezione
AUTHORS della guida BlackArch.
\item Per consistenza, per favore segui lo stile generale degli altri file \textit{PKGBUILD} nel repo e usa l'indentazione a due spazi.
\end{enumerate}

\subsection{Consigli Generali}
\href{http://wiki.archlinux.org/index.php/Namcap}{namcap} può controllare i pacchetti alla ricerca di errori.

%------------------%
%  Chapter 4       %
%------------------%

\chapter{Guida Strumenti}
Prossimamente...

\section{Coming Soon}
Prossimamente...

%%% APPENDIX %%%
\appendix
\include{latex/appendix}

\end{document}

%%% EOF %%%
