%%%%%%%%%%%%%%%%%%%%%%%%%%%%%%%%%%%%%%%%%%%%%%%%%%%%%%%%%%%%%%%%%%%%%%%%%%%%%%%%
%                                                                              %
% BlackArch Linux Guide                                                        %
%                                                                              %
%%%%%%%%%%%%%%%%%%%%%%%%%%%%%%%%%%%%%%%%%%%%%%%%%%%%%%%%%%%%%%%%%%%%%%%%%%%%%%%%


%%% BEGIN %%%
\documentclass[a4paper, oneside, 11pt]{book}


%%% INCLUDES %%%
\renewcommand{\familydefault}{\sfdefault}
\usepackage{verbatim}
\usepackage[utf8]{inputenc}
\usepackage{geometry}
\usepackage{array}
\usepackage{graphicx}
\usepackage{supertabular}
\usepackage{verbatim}
\usepackage{pstricks}
\usepackage{fancyhdr}
\usepackage{fancyvrb}
\usepackage{tocloft}
\usepackage{color}
\usepackage{enumerate}
\usepackage[
pdftitle={BlackArch Linux, The BlackArch Linux Guide},
pdfsubject={BlackArch Linux, The BlackArch Linux Guide},
pdfauthor={BlackArch Linux, BlackArch Linux},
pdfkeywords={BlackArch Linux, Penetration Testing, Security, Hacking, Linux},
pdfcenterwindow=true
]{hyperref}


%%% LAYOUT %%%
\setlength{\parindent}{0em}
\setlength{\parskip}{1.5ex plus0.5ex minus0.5ex}
\geometry{left=2.5cm, textwidth=16cm, top=3cm, textheight=25cm, bottom=3cm}
\widowpenalty=2000
\clubpenalty=1000
\frenchspacing
\sloppy
\pagecolor{black}
\color{white}
\hypersetup{pdftex=true, colorlinks=true, breaklinks=true, linkcolor=red,
menucolor=red, pagecolor=red, urlcolor=red}
\setcounter{tocdepth}{10}
\setcounter{secnumdepth}{10}


%%% HEADER / FOOTER %%%
\setlength{\headheight}{1.0cm}
\setlength{\headsep}{1.0cm}
\lhead{{\includegraphics[width=1cm,height=1cm]{logo.png}}}
\rhead{The BlackArch Linux Guide}
%\lfoot{\small{foo}}
%\rfoot{\small{bar}}


%%% OWN MACROS %%%
% put own macros here, like renewcommand newcommand etc.


%%% DOCUMENT %%%
\begin{document}
\pagestyle{empty}
\begin{center}
\begin{figure}[htbp]
\centering
\vspace{1cm}
\Huge{\textbf{BlackArch \color{red}Linux}}\\
\vspace{2cm}
\includegraphics[width=8cm]{logo.png}
\label{fig:logo}
\end{figure}
\vspace{1cm}
\Huge{\textbf{The BlackArch Linux Guide}}\\
\vspace{1cm}
\Large{\color{red}http://www.blackarch.org/}\\
\vspace{0.5cm}
\Large{\today}
\end{center}
\newpage
\tableofcontents
\newpage
\pagestyle{fancy}

%%%%%%%%%%%%%%%%%%%%%%%%%%%%%%%%%%%%%%%%%%%%%%%%%%%%%%%%%%%%%%%%%%%%%%%%%%%%%%%%

\chapter{Introduction}

\section{What is BlackArch Linux?}
\href{http://www.blackarch.org}{BlackArch Linux} is a lightweight expansion to
Arch Linux for penetration testers.
\\\\
The toolset is distributed as an Arch Linux
\href{https://wiki.archlinux.org/index.php/Unofficial\_User\_Repositories}
{unofficial user repository} so you can install BlackArchLinux on top of an
existing Arch Linux installation. Packages may be installed individually or by
category.
\\\\
We currently have over 650 tools in our toolset and the repository is constantly
expanding. All tools are thoroughly tested before being added to the codebase to
maintain the quality of the repository.

%\section{The story of BlackArch Linux}
%foo bar

%\section{Supported platforms}
%foo bar

\section{Get involved}
You can get in touch with the BlackArch team. Just check out the following:
\\\\
Web: \url{http://www.blackarch.org/}
\\\\
Mail: \href{mailto:blackarchlinux@gmail.com}{blackarchlinux@gmail.com}
\\\\
IRC: \url{irc://irc.freenode.net/blackarch}

%%%%%%%%%%%%%%%%%%%%%%%%%%%%%%%%%%%%%%%%%%%%%%%%%%%%%%%%%%%%%%%%%%%%%%%%%%%%%%%%

\chapter{User Guide}

\section{Installation}
The following sections will show you how to setup the BlackArch repository and
install packages. BlackArch supports both, installing from repository using
binary packages as well as compiling and installing from sources.
\\\\
BlackArch is compatible with normal Arch installations. It acts as an unofficial
user repository. If you want an ISO instead, see the
\href{http://www.blackarch.org/download.html#iso}{Live ISO} section.
\\\\

\subsection{Setting up repository}
There are 6 steps in order to setup and use the BlackArch repository
successfully. You must follow the instuctions in order. Do not add
\textbf{blackarch} to your \textit{pacman.conf} file without following steps 0
to 2 first.
\begin{enumerate}
\item If you have installed BlackArch before and there is an existing
\textbf{[blackarch]} entry in \textit{/etc/pacman.conf}, remove or comment out
the entry and run \textit{pacman -Syy}.
\item Run the following as root. This is for package signing.
{\small
\color{gray}
\begin{verbatim}
wget -q http://blackarch.org/keyring/blackarch-keyring.pkg.tar.xz{,.sig}
gpg --keyserver hkp://pgp.mit.edu --recv 4345771566D76038C7FEB43863EC0ADBEA87E4E3
gpg --keyserver-o no-auto-key-retrieve --with-f blackarch-keyring.pkg.tar.xz.sig
pacman-key --init
rm blackarch-keyring.pkg.tar.xz.sig
pacman --noc -U blackarch-keyring.pkg.tar.xz
\end{verbatim}
}
\item If possible, please verify the signing key's fingerprint against as many
sources as possible.
\item Append the following lines to your /etc/pacman.conf file:
{\small
\color{gray}
\begin{verbatim}
[blackarch]
Server = <mirror_site>/$repo/os/$arch
\end{verbatim}
}
Replace \textit{\textless mirror\_site\textgreater} with a mirror site of your
choosing. Please use one of our official mirrors.
\item Now run:
{\small
\color{gray}
\begin{verbatim}
# pacman -Syyu
\end{verbatim}
}
\end{enumerate}

\subsection{Installing packages}
You may now install tools from the blackarch repository.
\begin{enumerate}
\item To list all of the available tools, run
{\small
\color{gray}
\begin{verbatim}
# pacman -Sgg | grep blackarch | cut -d' ' -f2 | sort -u
\end{verbatim}
}
\item To install all of the tools, run
{\small
\color{gray}
\begin{verbatim}
# pacman -S blackarch
\end{verbatim}
}
\item To install a category of tools, run
{\small
\color{gray}
\begin{verbatim}
# pacman -S blackarch-<category>
\end{verbatim}
}
\item To see the blackarch categories, run
{\small
\color{gray}
\begin{verbatim}
# pacman -Sg | grep blackarch
\end{verbatim}
}
\end{enumerate}

\subsection{Installing packages from source}
As part of an alternative method of installation, you can build the BlackArch
packages from source. You can find the PKGBUILDs on
\href{https://github.com/BlackArch/blackarch/tree/master/packages}{github}. To
build the entire repo, you can use the
\href{https://github.com/BlackArch/blackman}{Blackman} tool.
\begin{itemize}
\item First, you have to install Blackman. If the BlackArch package repository
is setup on your machine, you can install Blackman:
{\small
\begin{verbatim}
pacman -S blackman
\end{verbatim}
}
\item You can build and install Blackman from source:
{\small
\color{gray}
\begin{verbatim}
mkdir blackman
cd blackman
wget https://raw2.github.com/BlackArch/blackarch/master/packages/blackman/PKGBUILD
# Make sure the PKGBUILD has not been maliciously tampered with.
makepkg -s
\end{verbatim}
}
\item Or you can install Blackman from the AUR:
{\small
\color{gray}
\begin{verbatim}
<whatever AUR helper you use> -S blackman
\end{verbatim}
}
\end{itemize}

\subsection{Basic Blackman usage} Blackman is very simple to use, though the flags are different from what you
would typically expect from something like pacman. Basic usage has been outlined below.  
\begin{itemize}
\item Download, compile and install packages:
{\small
\color{gray}
\begin{verbatim}
$ sudo blackman -i package
\end{verbatim}
}
\item Download, compile and install whole category:
{\small
\color{gray}
\begin{verbatim}
$ sudo blackman -g group
\end{verbatim}
}
\item Download, compile and install all of the BlackArch tools:
{\small
\color{gray}
\begin{verbatim}
$ sudo blackman -a
\end{verbatim}
}
\item To list the blackarch categories:
{\small
\color{gray}
\begin{verbatim}
$ blackman -l
\end{verbatim}
}
\item To list category tools:
{\small
\color{gray}
\begin{verbatim}
$ blackman -p category
\end{verbatim}
}
\end{itemize}

\subsection{Installing from live-, netinstall- ISO or ArchLinux}
You can install BlackArch Linux from one of our live- or netinstall-ISOs.\\See
\url{http://www.blackarch.org/download.html#iso}. The following steps are
required after the ISO boot up.
\begin{itemize}
\item Install blackarch-install-scripts package:
{\small
\color{gray}
\begin{verbatim}
$ sudo pacman -S blackarch-install-scripts
\end{verbatim}
}
\item Run
{\small
\color{gray}
\begin{verbatim}
$ sudo blackarch-install
\end{verbatim}
}
\end{itemize}

%%%%%%%%%%%%%%%%%%%%%%%%%%%%%%%%%%%%%%%%%%%%%%%%%%%%%%%%%%%%%%%%%%%%%%%%%%%%%%%%

\chapter{Developer Guide}

\section{Arch's Build System and Repositories}

PKGBUILD files are build scripts. Each one tells makepkg(1) how to create a package. PKGBUILD files
are written in Bash.

For more information, read (or skim through) the following:
\begin{itemize}
	\item \href{https://wiki.archlinux.org/index.php/Creating_Packages}{Arch Wiki: Creating Packages}
	\item \href{https://wiki.archlinux.org/index.php/Makepkg}{Arch Wiki: makepkg}
	\item \href{https://wiki.archlinux.org/index.php/PKGBUILD}{Arch Wiki: PKGBUILD}
	\item \href{https://wiki.archlinux.org/index.php/Arch_Packaging_Standards}{Arch Wiki: Arch Packaging Standards}
\end{itemize}

\section{Blackarch PKGBUILD standards}

For the sake of simplicity, our PKGBUILDs are similar to that of the AUR ones, with a few small differences that I will outline below. Every package must belond to blackarch at the minimum, there will also be a lot of crossover with multiple packages belonging to multiple groups.

\subsection{Groups}

To allow users to install a specific range of packages quickly and easily, we have separated our packages into groups. These groups allow users to simple go "pacman -S <group name>" in order to pull a lot of packages.

\subsubsection{blackarch}

The blackarch group is what every package must be included in along with its relevant group, to allow users to install every package very easily.

What should be in here: Everything.

\subsubsection{blackarch-anti-forensic}

This group is for packages that are used for countering forensic activities, such as encryption, steganography, anything that modifies files/file attributes, or just anything that makes changes to the system for the purposes of hiding information.

Examples: luks, TrueCrypt, Timestomp, dd, ropeadope, secure-delete

\subsubsection{blackarch-automation}

This group is for packages that are used for automating anything. Should be fairly self-explanatory.

Examples: blueranger, tiger, wiffy

\subsubsection{blackarch-backdoor}

This group is for packages that exploit or open backdoors on already vulnerable systems. Again should be reasonably self-explanatory.

Examples: backdoor-factory, rrs, weevely

\subsubsection{blackarch-binary}

This group is for packages that operate on binary files in one form or another.

Examples: binwally, packerid

\subsubsection{blackarch-bluetooth}

This group is for any packages that exploits anything to do with the Bluetooth standard (802.15.1).

Examples: ubertooth, tbear, redfang

\subsubsection{blackarch-code-audit}

This group is for any packages that audit existing source code for vulnerability analysis.

Examples: flawfinder, pscan

\subsubsection{blackarch-cracker}

This group is for any packages that is used for cracking any kind of cryptographic function, ie hashes.

Examples: hashcat, john, crunch

\subsubsection{blackarch-crypto}

This group is for any packages that is to do with cryptography, be it cracking or analysis or anything.

Examples: ciphertest, xortool, sbd

\subsubsection{blackarch-database}

This group is for any packages that involve some kind of database exploitation on any level.

Examples: metacoretex, blindsql

\subsubsection{blackarch-debugger}

This group is for packages that allow the user to view what a particular program is "doing" as it runs its instructions.

Examples: radare2, shellnoob

\subsubsection{blackarch-decompiler}

This group is for packages that attempt to reverse a compiled program into bare source code.

Examples: flasm, jd-gui

\subsubsection{blackarch-defensive}

This group is for any packages that are used to protect a user from malware \& attacks from other users.

Examples: arpon, chkrootkit, sniffjoke

\subsubsection{blackarch-disassembler}

This group is similar to blackarch-decompiler, and there will probably be a lot of programs that fall into both, however these packages produce assembly output rather than the raw source code.

Examples: inguma, radare2

\subsubsection{blackarch-dos}

This group is for packages that use DoS (Denial of Service) attacks.

Examples: 42zip, nkiller2

\subsubsection{blackarch-drone}

This group is for packages that are used for managing physical engineered drones.

Examples: meshdeck, skyjack

\subsubsection{blackarch-exploitation}

This group is for any packages that takes advantages of exploits in other programs or services.

Examples: armitage, metasploit, zarp

\subsubsection{blackarch-fingerprint}

This group is for any packages that exploit fingerprint biometric equipment.

Examples: dns-map, p0f, httprint

\subsubsection{blackarch-firmware}

This group is for any packages that exploit vulnerabilities in firmware

Examples: None yet, amend asap.

\subsubsection{blackarch-forensic}

This group is for any packages that are used to find data using on physical disks or embedded memory.

Examples: aesfix, nfex, wyd 

\subsubsection{blackarch-fuzzer}

This group is for any packages that use the fuzz testing principle, ie "throwing" random inputs at the subject to see what happens.

Examples: msf, mdk3, wfuzz

\subsubsection{blackarch-hardware}

This group is for any packages that exploit or manage anything to do with physical hardware.

Examples: arduino, smali

\subsubsection{blackarch-honeypot}

This group is for packages that act as "honeypots", ie programs that appear to be vulnerable services used to attract hackers into a trap.

Examples: artillery, bluepot, wifi-honey

\subsubsection{blackarch-keylogger}

This group is for packages that record and retain keystrokes on another system.

Examples: None yet, amend asap.

\subsubsection{blackarch-malware}

This group is for packages that count as any type of malicious software or malware detection.

Examples: malwaredetect, peepdf, yara

\subsubsection{blackarch-misc}

This group is for packages that don't particularly fit into any categories.

Examples: oh-my-zsh-git, winexe, stompy

\subsubsection{blackarch-mobile}

This group is for packages that manipulate mobile platforms.

Examples: android-sdk-platform-tools, android-udev-rules

\subsubsection{blackarch-networking}

This group is for any package that involes IP networking at any point for any reason.

Examples: Anything pretty much

\subsubsection{blackarch-nfc}

This group is for any package that uses nfc (near-field communications) at any point for any reason.

Examples: nfcutils

\subsubsection{blackarch-packer}

This group is for any packages that operate on or invlve packers, packers being programs that embed malware within other executables.

Examples: packerid

\subsubsection{blackarch-proxy}

This group is for any package that acts as a proxy, ie redirecting traffic through another node on the internet.

Examples: burpsuite, ratproxy, sslnuke

\subsubsection{blackarch-recon}

This group is for any package that actively seeks vulnerable exploits in the wild. More of an umbrella group for similar packages.

Examples: canri, dnsrecon, netmask

\subsubsection{blackarch-reversing}

This group is for any package that acts as an umbrella group for any decompiler, disassembler or any similar program.

Examples: capstone, radare2, zerowine

\subsubsection{blackarch-scanner}

This group is for packages that scan selected systems for vulnerabilities. 

Examples: scanssh, tiger, zmap

\subsubsection{blackarch-sniffer}

This group is for any package that involve analyzing network traffic.

Examples: hexinject, pytactle, xspy

\subsubsection{blackarch-social}

This group is for any package that attacks social networking sites primarily.

Examples: jigsaw, websploit

\subsubsection{blackarch-spoof}

This group is for any package that attempts to spoof the attacker such, in that the attacker doesn't show up as an attacker to the victim.

Examples: arpoison, lans, netcommander

\subsubsection{blackarch-threat-model}

This group is for any package that would be used for reporting/recording the threat model outlined in a particular scenario.

Examples: magictree

\subsubsection{blackarch-tunnel}

This group is for packages that are used to tunnel network traffic on a given network.

Examples: ctunnel, iodine, ptunnel

\subsubsection{blackarch-unpacker}

This group is for any package that is used to extract pre-packed malware from an executable.

Examples: js-beautify

\subsubsection{blackarch-voip}

This group is for packages that operate on voip programs and protocols.

Examples: iaxflood, rtp-flood, teardown

\subsubsection{blackarch-webapp}

This group is for packages that operate on internet-facing applications.

Examples: metoscan, whatweb, zaproxy

\subsubsection{blackarch-windows}

This groups is for any native Windows package that runs via wine.

Examples: 3proxy-win32, pwdump, winexe

\subsubsection{blackarch-wireless}

iThis group is for any package that operates on wireless networks on any level.

Examples: airpwn, mdk3, wiffy

\section{Repository structure}

You can find the main BlackArch git repo here:
\href{https://github.com/BlackArch/blackarch}{https://github.com/BlackArch/blackarch}. There are
also several secondary repos here:
\href{https://github.com/BlackArch}{https://github.com/BlackArch}.

Within the main git repo, there are three important directories:

\begin{itemize}
	\item docs - Documentation.
	\item packages - PKGBUILD files.
	\item scripts - Useful little scripts.
\end{itemize}

\subsection{Scripts}

Here is a reference for scripts in the \verb|scripts/| directory:

\begin{itemize}
	\item baaur - Soon, this will upload packages to the AUR.
	\item babuild - Build a package.
	\item bachroot - Manage a chroot for testing.
	\item baclean - Clean old .pkg.tar.xz files from the package repo.
	\item baconflict - Soon this will replace \verb|scripts/conflicts|.
	\item bad-files - Find bad files in built packages.
	\item balock - Obtain or release the package repo lock.
	\item banotify - Notify IRC about package pushes.
	\item barelease - Release packages to the package repo.
	\item baright - Print the BlackArch copyright info.
	\item basign - Sign packages.
	\item basign-key - Sign a key.
	\item blackman - This behaves sort of like pacman but builds from git (not to be confused with nrz's Blackman).
	\item check-groups - Check groups.
	\item checkpkgs - Check packages for errors.
	\item conflicts - Check for file conflicts.
	\item dbmod - Modify a package database.
	\item depth-list - Create a list sorted by dependency depth.
	\item deptree - Create a dependency tree, listing only blackarch-provided packages.
	\item get-blackarch-deps - Get a list of blackarch dependencies for a package.
	\item get-official - Get official packages for release.
	\item list-loose-packages - List packages that are not in groups and are not dependencies for other packages.
	\item list-needed - List missing dependencies.
	\item list-removed - List packages that are in the package repo but not in git.
	\item list-tools - List tools.
	\item outdated - Look for packages that are out-dated in the package repo relative to the git repo.
	\item pkgmod - Modify a build package.
	\item pkgrel - Increment pkgrel in a package.
	\item prep - Clean up a PKGBUILD file's style and find errors.
	\item sitesync - Sync between a local copy of the package repo and a remote copy.
	\item size-hunt - Hunt for large packages.
	\item source-backup - Backup package source files.
\end{itemize}

\section{Contributing to repository}
This section shows you how to contribute to the BlackArch Linux project. We
accept pull requests of all sizes, from tiny typo fixes to new packages.\\For
help, suggestions, or questions feel free to contact us.
\\\\
Everyone is welcome to contribute. All contributions are appreciated.

\subsection{Required tutorials}
Please read the following tutorials before contributing:
\begin{itemize}
\item
\href{https://wiki.archlinux.org/index.php/Arch\_Packaging\_Standards)}{Arch
Packaging Standards}
\item \href{https://wiki.archlinux.org/index.php/Creating\_Packages}{Creating
Packages}
\item \href{https://wiki.archlinux.org/index.php/PKGBUILD}{PKGBUILD}
\item \href{https://wiki.archlinux.org/index.php/Makepkg}{Makepkg}
\end{itemize}

\subsection{Steps for contributing}
In order to submit your changes to the BlackArchLinux project, follow these
steps:
\begin{enumerate}
\item Fork the repository from
\url{https://github.com/BlackArchLinux/blackarchlinux}
\item Hack the necessary files, (e.g. PKGBUILD, .patch files, etc).
\item Commit your changes.
\item Push your changes.
\item Ask us to merge in your changes, preferably through a pull request.
\end{enumerate}

\subsection{Example}
The following example demonstrates submitting a new package to the BlackArch
project. We use \href{https://wiki.archlinux.org/index.php/yaourt}{yaourt}
(you can use pacaur as well) to fetch a pre-existing PKGBUILD file for
\textbf{nfsshell} from the \href{https://aur.archlinux.org/}{AUR} and adjust it
according to our needs.

\subsubsection{Fetch PKGBUILD}
Fetch the \textit{PKGBUILD} file using yaourt or pacaur:
{\small
\color{gray}
\begin{verbatim}
user@blackarchlinux $ yaourt -G nfsshell
==> Download nfsshell sources
x LICENSE
x PKGBUILD
x gcc.patch
user@blackarchlinux $ cd nfsshell/
\end{verbatim}
}

\subsubsection{Clean up PKGBUILD}
Clean up the \textit{PKGBUILD} file and save some time:
{\small
\color{gray}
\begin{verbatim}
user@blackarchlinux nfsshell $ ./blarckarch/scripts/prep PKGBUILD
cleaning 'PKGBUILD'...
expanding tabs...
removing vim modeline...
removing id comment...
removing contributor and maintainer comments...
squeezing extra blank lines...
removing '|| return'...
removing leading blank line...
removing $pkgname...
removing trailing whitespace...
\end{verbatim}
}

\subsubsection{Adjust PKGBUILD}
Adjust the \textit{PKGBUILD} file:
{\small
\color{gray}
\begin{verbatim}
user@blackarchlinux nfsshell $ vi PKGBUILD
\end{verbatim}
}

\subsubsection{Build the package}
Build the package:
{\small
\color{gray}
\begin{verbatim}
user@blackarchlinux nfsshell $ makepkg -sf
==> Making package: nfsshell 19980519-1 (Mon Dec  2 17:23:51 CET 2013)
==> Checking runtime dependencies...
==> Checking buildtime dependencies...
==> Retrieving sources...
-> Downloading nfsshell.tar.gz...
% Total    % Received % Xferd  Average Speed   Time    Time     Time
CurrentDload  Upload   Total   Spent    Left  Speed100 29213  100 29213    0
0  48150      0 --:--:-- --:--:-- --:--:-- 48206
-> Found gcc.patch
-> Found LICENSE
...
<lots of build process and compiler output here>
...
==> Leaving fakeroot environment.
==> Finished making: nfsshell 19980519-1 (Mon Dec  2 17:23:53 CET 2013)
\end{verbatim}
}

\subsubsection{Install and test the package}
Install and test the package:
{\small
\color{gray}
\begin{verbatim}
user@blackarchlinux nfsshell $ pacman -U nfsshell-19980519-1-x86_64.pkg.tar.xz
user@blackarchlinux nfsshell $ nfsshell # test it
\end{verbatim}
}

\subsubsection{Add, commit and push package}
Add, commit and push the package
{\small
\color{gray}
\begin{verbatim}
user@blackarchlinux nfsshell $ cd /blackarchlinux/packages
user@blackarchlinux ~/blackarchlinux/packages $ mv ~/nfsshell .
user@blackarchlinux ~/blackarchlinux/packages $ git commit -am nfsshell && git push
\end{verbatim}
}

\subsubsection{Create a pull request}
Create a pull request on \href{https://github.com/}{github.com}
{\small
\color{gray}
\begin{verbatim}
firefox https://github.com/<contributor>/blackarchlinux
\end{verbatim}
}

\subsubsection{Adding a remote for upstream}
A smart thing to do if you're working upstream and on a fork is to pull your own fork and add the main ba repo as a remote
{\small
\color{gray}
\begin{verbatim}
user@blackarchlinux ~/blackarchlinux $ git remote -v
origin <the url of your fork> (fetch)
origin <the url of your fork> (push)
user@blackarchlinux ~/blackarchlinux $ git remote add upstream https://github.com/blackarch/blackarch
user@blackarchlinux ~/blackarchlinux $ git remote -v
origin <the url of your fork> (fetch)
origin <the url of your fork> (push)
upstream https://github.com/blackarch/blackarch (fetch)
upstream https://github.com/blackarch/blackarch (push)
\end{verbatim}
By defualt, git should push straight to origin, but make sure your git config is configured correctly. This won't be an issue unless you have commit rights as you won't be able to push upstream without them.

If you do have the ability to commit, you might have more success using \textit{git@github.com:blackarch/blackarch.git} but it's up to you.
}
\subsection{Requests}
\begin{enumerate}
\item Don't add \textbf{Maintainer} or \textbf{Contributor} comments to
\textit{PKGBUILD} files. Add maintainer and contributor names to the
AUTHORS section of BlackArch guide.
\item For the sake of consistency, please follow the general style of the other
\textit{PKGBUILD} files in the repo and use two-space indentation.
\end{enumerate}

\subsection{General tips}
\href{http://wiki.archlinux.org/index.php/Namcap}{namcap} can check packages for
errors.

%%% APPENDIX %%%
\appendix
\include{appendix}

\end{document}

%%% EOF %%%
