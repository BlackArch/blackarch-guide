%%%%%%%%%%%%%%%%%%%%%%%%%%%%%%%%%%%%%%%%%%%%%%%%%%%%%%%%%%%%%%%%%%%%%%%%%%%%%%%%
%                                                                              %
% BlackArch Linux Guide                                                        %
%                                                                              %
%%%%%%%%%%%%%%%%%%%%%%%%%%%%%%%%%%%%%%%%%%%%%%%%%%%%%%%%%%%%%%%%%%%%%%%%%%%%%%%%

\documentclass[a4paper, oneside, 11pt]{book}

%%% INCLUDES %%%
\renewcommand{\familydefault}{\sfdefault}

\usepackage{array}
\usepackage{color}
\usepackage{enumerate}
\usepackage{fancyhdr}
\usepackage{fancyvrb}
\usepackage{geometry}
\usepackage{graphicx}
\usepackage{html}
\usepackage{hyperref}
\usepackage{ifpdf}
\usepackage{listings}
\usepackage{pstricks}
\usepackage{supertabular}
\usepackage{tocloft}
\usepackage[utf8]{inputenc}

%%% LAYOUT %%%
\setlength{\parindent}{0em}
\setlength{\parskip}{1.5ex plus0.5ex minus0.5ex}
\geometry{left=2.5cm, textwidth=16cm, top=3cm, textheight=25cm, bottom=3cm}
\widowpenalty=2000
\clubpenalty=1000
\frenchspacing
\sloppy
\pagecolor[HTML]{FFFFFF}
\color[HTML]{333333}
\setcounter{tocdepth}{10}
\setcounter{secnumdepth}{10}

\hypersetup{
  pdftitle={BlackArch Linux, Guia BlackArch Linux},
  pdfsubject={BlackArch Linux, Guia BlackArch Linux},
  pdfauthor={BlackArch Linux, BlackArch Linux},
  pdfkeywords={BlackArch Linux, Penetration Testing, Security, Hacking, Linux},
  pdfcenterwindow=true,
  colorlinks=true,
  breaklinks=true,
  linkcolor=red,
  menucolor=red,
  urlcolor=red
}

% syntax highlighting
% all options should be set here document wide
\lstset{
backgroundcolor=\color[HTML]{EEEEEE},
frame=single,
basicstyle=\footnotesize\ttfamily,
float,
deletekeywords={return},
otherkeywords={mkdir, curl, sudo, sha1sum, grep, cut, sort, wget, makepkg,
pacman, blackman},
keywordstyle=\color{orange},
commentstyle=\color{blue},
stringstyle=\color{red},
language=bash,
showspaces=false,
showtabs=false,
tabsize=2
}

%%% HEADER / FOOTER %%%
\setlength{\headheight}{33pt}
\setlength{\headsep}{33pt}
\lhead{{\includegraphics[width=1cm,height=1cm]{images/logo.png}}}
\rhead{Guia BlackArch Linux}

%%% CUSTOM MACROS %%%
% for html links
\ifpdf\else
\def\href#1#2{\htmladdnormallink{#2}{#1}}
\fi

%------------------%
%  TITLE PAGE      %
%------------------%
\begin{document}
\pagestyle{empty}
\begin{center}
\begin{figure}[htbp]
\centering
\vspace{0.5cm}
\includegraphics[width=8cm]{images/logo.png}
\label{fig:logo}
\end{figure}
\vspace{0.5cm}
\Huge{\textbf{Guia BlackArch Linux}}\\
\vspace{1cm}
\Large{\color{red}https://www.blackarch.org/}\\
\vspace{0.5cm}
\end{center}
\newpage
\tableofcontents
\newpage
\pagestyle{fancy}

%------------------%
%  Chapter 1       %
%------------------%

\chapter{Introdução}

\section{Resumo}
O guia do BlackArch linux é dividido em algumas partes:
\begin{itemize}
\item Introdução - Apresenta uma visão geral e informações adicionais sobre o projeto
\item Guia do Usuário - Tudo que o usuário precisa saber para usar o BlackArch de forma efetiva
\item Guia do Desenvolvedor - Como começar a desenvolver e contribuir para o BlackArch
\item Guia das Ferramentas - Detalhes e exemplos de uso das ferramentas
\end{itemize}

\section{O que é o BlackArch Linux?}
BlackArch é uma distribuição Linux para testes de penetração e pesquisadores de segurança.É uma derivação do \href{https://www.archlinux.org/}{ArchLinux} e os usuários podem instalar componentes do BlackArch individualmente ou por grupos em cima da distribuição.
As ferramentas são distribuídas através do Arch Linux
\href{https://wiki.archlinux.org/index.php/Unofficial\_User\_Repositories}
{unofficial user repository} então você pode instalar o BlackArch em cima de uma instalação Arch Linux já existente.Os pacotes pode ser instalados individualmente ou por categorias.

O repositório continua aumentando e já possui  \href{https://www.blackarch.org/tools.html}{1300} ferramentas.
Todas as ferramentas são testadas depois da adição de algum codigo mantendo a qualidade do repositório.
% should quickly describe the testing methods/code review procedures etc

\section{A historia do BlackArch Linux}
Coming soon...

\section{Plataformas suportadas}
Coming soon...

\section{Junte-se ao BlackArch}
Você pode se comunicar com a equipe BlackArch através das formas abaixo:

Website: \url{https://www.blackarch.org/}

Mail: \href{mailto:team@blackarch.org}{team@blackarch.org}

IRC: \url{irc://irc.freenode.net/blackarch}

Twitter: \url{https://twitter.com/blackarchlinux}

Github: \url{https://github.com/Blackarch/}

%------------------%
%  Chapter 2       %
%------------------%


\chapter{Guia do Usuário}

\section{Instalação} 
A seção seguinte mostra como configurar o repositório BlackArch e instalar os pacotes.
O BlackArch suporta a instalação direta pelo repositório usando pacotes binários ou compilando, como também através de outras fontes.
O BlackArch é compatível como toda instalação do Arch.A instalação é feita através de um repositório não oficial, se você quiser usar uma ISO olhe na seção \href{https://www.blackarch.org/downloads.html#iso}{Live ISO}.

\subsection{Instalando em cima do ArchLinux}
Execute \href{https://blackarch.org/strap.sh}{strap.sh} como root e siga as instruções abaixo.
\begin{lstlisting}
   curl -O https://blackarch.org/strap.sh
   sha1sum strap.sh # should match: 86eb4efb68918dbfdd1e22862a48fda20a8145ff
   sudo ./strap.sh
\end{lstlisting}
Agora baixe uma copia atualizada da lista de pacotes e os sincronize.
\begin{lstlisting}
  sudo pacman -Syyu
\end{lstlisting}


\subsection{Instalando os pacotes}
Agora você pode instalar os pacotes do repositório do blackarch.
\begin{enumerate} 
\item Para listar todas as ferramentas disponíveis execute
\begin{lstlisting}
  pacman -Sgg | grep blackarch | cut -d' ' -f2 | sort -u
\end{lstlisting}

\item Para instalar todas a ferramentas execute
\begin{lstlisting}
  pacman -S blackarch
\end{lstlisting}

\item Para instalar toda uma categoria execute
\begin{lstlisting}
  pacman -S blackarch-<category>
\end{lstlisting}

\item Para ver as categorias execute
\begin{lstlisting}
  pacman -Sg | grep blackarch
\end{lstlisting}

\end{enumerate}

\subsection{Instalando os pacotes pelo codigo fonte}
De maneira alternativa você pode criar os pacotes do BlackArch diretamente do codigo fonte.Você pode achar os PKGBUILDs no
\href{https://github.com/BlackArch/blackarch/tree/master/packages}{github}.Para criar todo o repositório você pode usar a ferramenta
\href{https://github.com/BlackArch/blackman}{Blackman}.
\begin{itemize}
\item Primeramente, se você tem que instalar o Blackman, se o repositório do BlackArch está configurado na sua máquina, você pode instalar o Blackman:
\begin{lstlisting}
  pacman -S blackman
\end{lstlisting}

\item Você pode construir e instalar o Blackman pelo código fonte:
\begin{lstlisting}
  mkdir blackman
  cd blackman
  wget https://raw2.github.com/BlackArch/blackarch/master/packages/blackman/PKGBUILD
  # Make sure the PKGBUILD has not been maliciously tampered with.
  makepkg -s
\end{lstlisting}

\item Ou você pode instalar o Blackman pelo AUR::
\begin{lstlisting}
  <whatever AUR helper you use> -S blackman
\end{lstlisting} 

\end{itemize}

\subsection{Uso básico do Blackman} Blackman possui um uso simples, suas flags são um pouco diferentes do pacman.O básico será mostrado abaixo.
\begin{itemize}
\item Baixa, compila e instala o pacote:
\begin{lstlisting}
  sudo blackman -i package
\end{lstlisting}

\item Baixa, compila e instala toda uma categoria:
\begin{lstlisting}
  sudo blackman -g group
\end{lstlisting}

\item Baixa, compila e instala todas as ferramentas do BlackArch:
\begin{lstlisting}
  sudo blackman -a
\end{lstlisting}

\item Lista todas as categorias de ferramentas:
\begin{lstlisting}
  blackman -l
\end{lstlisting}

\item Lista uma categoria de ferramentas:
\begin{lstlisting}
  blackman -p categorytem 
\end{lstlisting}

\end{itemize}

\subsection{Instalação pelo live-, netinstall- ISO ou ArchLinux}
Você pode instalar o BlackArch através de um live- ou netinstall-ISOs.\\Veja em
\url{https://www.blackarch.org/download.html#iso}.Os passos a seguir são executados depois da inicialização da ISO.
\begin{itemize}
\item Instalando o pacote blackarch-installer:
\begin{lstlisting}
  sudo pacman -S blackarch-installer
\end{lstlisting}

\item Execute
\begin{lstlisting}
  sudo blackarch-install
\end{lstlisting}

\end{itemize}

%------------------%
%  Chapter 3       %
%------------------%

\chapter{Guia do Desenvolvedor}

\section{Construir sistemas e repositórios Arch}

Os PKGBUILD são scripts de construição.Cada um diz ao makepkg(1) como construir o pacote.Os arquivos PKGBUILD são escritos em Bash.

Para mais informações, ler abaixo:
\begin{itemize}
\item \href{https://wiki.archlinux.org/index.php/Creating_Packages}{Arch Wiki: Creating Packages}
\item \href{https://wiki.archlinux.org/index.php/Makepkg}{Arch Wiki: makepkg}
\item \href{https://wiki.archlinux.org/index.php/PKGBUILD}{Arch Wiki: PKGBUILD}
\item \href{https://wiki.archlinux.org/index.php/Arch_Packaging_Standards}{Arch Wiki: Arch Packaging Standards}
\end{itemize}

\section{Padrão PKGBUILD do Blackarch}
Por sua simplicidade, os PKGBUILDs são similares aos do AUR, como algumas pequenas diferenças como mostrado a baixo.Cada pacote deve pertencer ao blackarch o minimo possível, e deve haver o máximo de crossover com vários pacotes percentes a vários grupos.
\subsection{Grupos}
Permite que os usuários instalem um específico tipo de pacote de forma rápida e fácil, os pacotes são separados por grupos.Os grupos permite que o usuário digite "pacman -S <group name>" e consiga vários pacotes.

\subsubsection{blackarch}
O grupo blackarch é a o grupo base onde todos os pacotes se encontram.Isto permite que o usuário instale todos os pacotes com facilidade.

O que deve estar aqui: Tudo.

\subsubsection{blackarch-anti-forensic}
Pacotes relacionados a o uso de contra forense, incluindo encriptação, estenografia, e tudo para modificar os atributos de arquivo/arquivos.
Inclui todas as ferramentas para trabalhar com tudo em geral para modificar um sistema com o propósito de esconder uma informação.

Exemplo: luks, TrueCrypt, Timestomp, dd, ropeadope, secure-delete

\subsubsection{blackarch-automation}
Pacotes para uso de ferramentas ou automatização de tarefas.

Exemplo: blueranger, tiger, wiffy

\subsubsection{blackarch-backdoor}
Pacotes para explorar ou abrir blackdoors em sistemas vulneráveis.

Exemplo: backdoor-factory, rrs, weevely

\subsubsection{blackarch-binary}
Pacotes para modificar arquivos binários de algum modo.

Exemplo: binwally, packerid

\subsubsection{blackarch-bluetooth}
Pacotes para explorar tudo relacionado ao Bluetooth padrão (802.15.1).

Exemplo: ubertooth, tbear, redfang

\subsubsection{blackarch-code-audit}
Pacotes para analisar um codigo em busca de vulnerabilidades.

Exemplo: flawfinder, pscan

\subsubsection{blackarch-cracker}
Pacotes usados para quebrar criptografia.

Exemplo: hashcat, john, crunch

\subsubsection{blackarch-crypto}
Pacotes para trabalhar com criptografia sem ser sua quebra.

Exemplo: ciphertest, xortool, sbd

\subsubsection{blackarch-database}
Pacotes que envolvam exploração de banco de dados em algum nível.

Exemplo: metacoretex, blindsql

\subsubsection{blackarch-debugger}
Pacotes para permitir que o usuário veja o que um programa em particular esta "fazendo" em tempo real.

Exemplo: radare2, shellnoob

\subsubsection{blackarch-decompiler}
Pacotes voltados para converter um programa compilado em codigo fonte.

Exemplo: flasm, jd-gui

\subsubsection{blackarch-defensive}
Pacotes para proteger o usuário de malware \& ataques de outros usuários.

Exemplos: arpon, chkrootkit, sniffjoke

\subsubsection{blackarch-disassembler}
Similar ao blackarch-decompiler, possui semelhança em alguns pacotes mas esses são mais voltados a produzir uma saída em assembly, não em codigo fonte.

Exemplo: inguma, radare2

\subsubsection{blackarch-dos}
Pacotes para o uso de ataques DoS (Denial of Service).

Exemplo: 42zip, nkiller2

\subsubsection{blackarch-drone}
Pacotes usados para o manejo físico de engenharia de drones.

Exemplo: meshdeck, skyjack

\subsubsection{blackarch-exploitation}
Pacotes para explorar outros programas ou serviços.

Exemplo: armitage, metasploit, zarp

\subsubsection{blackarch-fingerprint}
Pacotes para explorar equipamentos de biometria digital.

Exemplo: dns-map, p0f, httprint

\subsubsection{blackarch-firmware}
Pacotes para explorar vulnerabilidades no firmware

Exemplo: None yet, amend asap.

\subsubsection{blackarch-forensic}
Pacotes usados para procurar informação em discos físicos ou memoria.

Exemplo: aesfix, nfex, wyd

\subsubsection{blackarch-fuzzer}
Pacotes que usam o principio do teste fuzz, "jogando" números aleatórios com o objetivo de ver o que acontece.

Exemplo: msf, mdk3, wfuzz

\subsubsection{blackarch-hardware}
Pacotes que explorar ou manejam hardware.

Exemplo: arduino, smali

\subsubsection{blackarch-honeypot}
Pacotes que agem como "honeypots" (pote de mel), programas para simular uma vulnerabilidade e cria uma armadilha para os hackers.

Exemplo: artillery, bluepot, wifi-honey

\subsubsection{blackarch-keylogger}
Pacotes para gravar o digito em outros sistemas.

Exemplo: None yet, amend asap.

\subsubsection{blackarch-malware}
Pacotes para qualquer tipo de software malicioso ou detecção de
malware detectio.

Exemplo: malwaredetect, peepdf, yara

\subsubsection{blackarch-misc}
Packages that don't particularly fit into any categories.

Exemplo: oh-my-zsh-git, winexe, stompy

\subsubsection{blackarch-mobile}
Pacotes para manipulação de sistemas moveis.

Exemplo: android-sdk-platform-tools, android-udev-rules

\subsubsection{blackarch-networking}
Pacotes que involvem uma rede IP.

Exemplo: Praticamente tudo

\subsubsection{blackarch-nfc}
Pacotes que usam nfc (near-field communications).

Exemplo: nfcutils

\subsubsection{blackarch-packer}
Pacotes que operam ou envolvem packers.

/textif{packers são programas que juntam malware com outros executáveis.}

Exemplo: packerid

\subsubsection{blackarch-proxy}
Pacotes que agem com proxy, como redirecionar o tráfego para outro nó na internet.

Exemplo: burpsuite, ratproxy, sslnuke

\subsubsection{blackarch-recon}
Pacotes que procuram exploits.Mais no grupo umbrella para pacotes similares.

Exemplo: canri, dnsrecon, netmask

\subsubsection{blackarch-reversing}
Este é um grupo umbrella para todo decompilador, disassembler ou programas similares.

Exemplo: capstone, radare2, zerowine

\subsubsection{blackarch-scanner}
Pacotes que scaneiam um sistema em busca de vulnerabilidades.

Exemplo: scanssh, tiger, zmap

\subsubsection{blackarch-sniffer}
Pacotes que envolvem analise de tráfego de rede.

Exemplo: hexinject, pytactle, xspy

\subsubsection{blackarch-social}
Pacotes
Packages that primarily attack social networking sites.

Examples: jigsaw, websploit

\subsubsection{blackarch-spoof}
Packages that attempt to spoof the attacker such, in that
the attacker doesn't show up as an attacker to the victim.

Examples: arpoison, lans, netcommander

\subsubsection{blackarch-threat-model}
Packages that would be used for reporting/recording the
threat model outlined in a particular scenario.

Examples: magictree

\subsubsection{blackarch-tunnel}
Packages that are used to tunnel network traffic on a given
network.

Examples: ctunnel, iodine, ptunnel

\subsubsection{blackarch-unpacker}
Packages that are used to extract pre-packed malware from an
executable.

Examples: js-beautify

\subsubsection{blackarch-voip}
Packages that operate on voip programs and protocols.

Examples: iaxflood, rtp-flood, teardown

\subsubsection{blackarch-webapp}
Packages that operate on internet-facing applications.

Examples: metoscan, whatweb, zaproxy

\subsubsection{blackarch-windows}
This group is for any native Windows package that runs via wine.

Examples: 3proxy-win32, pwdump, winexe

\subsubsection{blackarch-wireless}
Packages that operates on wireless networks on any level.

Examples: airpwn, mdk3, wiffy

\section{Repository structure}
You can find the main BlackArch git repo here:
\href{https://github.com/BlackArch/blackarch}{https://github.com/BlackArch/blackarch}.
There are also several secondary repos here:
\href{https://github.com/BlackArch}{https://github.com/BlackArch}.

Within the main git repo, there are three important directories:

\begin{itemize}
\item docs - Documentation.
\item packages - PKGBUILD files.
\item scripts - Useful little scripts.
\end{itemize}

\subsection{Scripts}
Here is a reference for scripts in the \verb|scripts/| directory:

\begin{itemize}
\item baaur - Soon, this will upload packages to the AUR.
\item babuild - Build a package.
\item bachroot - Manage a chroot for testing.
\item baclean - Clean old .pkg.tar.xz files from the package repo.
\item baconflict - Soon this will replace \verb|scripts/conflicts|.
\item bad-files - Find bad files in built packages.
\item balock - Obtain or release the package repo lock.
\item banotify - Notify IRC about package pushes.
\item barelease - Release packages to the package repo.
\item baright - Print the BlackArch copyright info.
\item basign - Sign packages.
\item basign-key - Sign a key.
\item blackman - This behaves sort of like pacman but builds from git (not to be
    confused with nrz's Blackman).
\item check-groups - Check groups.
\item checkpkgs - Check packages for errors.
\item conflicts - Check for file conflicts.
\item dbmod - Modify a package database.
\item depth-list - Create a list sorted by dependency depth.
\item deptree - Create a dependency tree, listing only blackarch-provided packages.
\item get-blackarch-deps - Get a list of blackarch dependencies for a package.
\item get-official - Get official packages for release.
\item list-loose-packages - List packages that are not in groups and are not
    dependencies for other packages.
\item list-needed - List missing dependencies.
\item list-removed - List packages that are in the package repo but not in git.
\item list-tools - List tools.
\item outdated - Look for packages that are out-dated in the package repo
    relative to the git repo.
\item pkgmod - Modify a build package.
\item pkgrel - Increment pkgrel in a package.
\item prep - Clean up a PKGBUILD file's style and find errors.
\item sitesync - Sync between a local copy of the package repo and a remote copy.
\item size-hunt - Hunt for large packages.
\item source-backup - Backup package source files.
\end{itemize}

\section{Contributing to repository}
This section shows you how to contribute to the BlackArch Linux project. We
accept pull requests of all sizes, from tiny typo fixes to new packages.\\For
help, suggestions, or questions feel free to contact us.
\\\\
Everyone is welcome to contribute. All contributions are appreciated.

\subsection{Required tutorials}
Please read the following tutorials before contributing:
\begin{itemize}
\item
\href{https://wiki.archlinux.org/index.php/Arch\_Packaging\_Standards)}{Arch
Packaging Standards}
\item \href{https://wiki.archlinux.org/index.php/Creating\_Packages}{Creating
Packages}
\item \href{https://wiki.archlinux.org/index.php/PKGBUILD}{PKGBUILD}
\item \href{https://wiki.archlinux.org/index.php/Makepkg}{Makepkg}
\end{itemize}

\subsection{Steps for contributing}
In order to submit your changes to the BlackArchLinux project, follow these
steps:
\begin{enumerate}
\item Fork the repository from
\url{https://github.com/BlackArch/blackarch}
\item Hack the necessary files, (e.g. PKGBUILD, .patch files, etc).
\item Commit your changes.
\item Push your changes.
\item Ask us to merge in your changes, preferably through a pull request.
\end{enumerate}

\subsection{Example}
The following example demonstrates submitting a new package to the BlackArch
project. We use \href{https://wiki.archlinux.org/index.php/yaourt}{yaourt}
(you can use pacaur as well) to fetch a pre-existing PKGBUILD file for
\textbf{nfsshell} from the \href{https://aur.archlinux.org/}{AUR} and adjust it
according to our needs.

\subsubsection{Fetch PKGBUILD}
Fetch the \textit{PKGBUILD} file using yaourt or pacaur:
\begin{lstlisting}
  user@blackarchlinux $ yaourt -G nfsshell
  ==> Download nfsshell sources
  x LICENSE
  x PKGBUILD
  x gcc.patch
  user@blackarchlinux $ cd nfsshell/
\end{lstlisting}

\subsubsection{Clean up PKGBUILD}
Clean up the \textit{PKGBUILD} file and save some time:
\begin{lstlisting}
  user@blackarchlinux nfsshell $ ./blarckarch/scripts/prep PKGBUILD
  cleaning 'PKGBUILD'...
  expanding tabs...
  removing vim modeline...
  removing id comment...
  removing contributor and maintainer comments...
  squeezing extra blank lines...
  removing '|| return'...
  removing leading blank line...
  removing $pkgname...
  removing trailing whitespace...
\end{lstlisting}

\subsubsection{Adjust PKGBUILD}
Adjust the \textit{PKGBUILD} file:
\begin{lstlisting}
  user@blackarchlinux nfsshell $ vi PKGBUILD
\end{lstlisting}

\subsubsection{Build the package}
Build the package:
\begin{lstlisting}user@blackarchlinux nfsshell $ makepkg -sf
==> Making package: nfsshell 19980519-1 (Mon Dec  2 17:23:51 CET 2013)
==> Checking runtime dependencies...
==> Checking buildtime dependencies...
==> Retrieving sources...
-> Downloading nfsshell.tar.gz...
% Total    % Received % Xferd  Average Speed   Time    Time     Time
CurrentDload  Upload   Total   Spent    Left  Speed100 29213  100 29213    0
0  48150      0 --:--:-- --:--:-- --:--:-- 48206
-> Found gcc.patch
-> Found LICENSE
...
<lots of build process and compiler output here>
...
==> Leaving fakeroot environment.
==> Finished making: nfsshell 19980519-1 (Mon Dec  2 17:23:53 CET 2013)
\end{lstlisting}

\subsubsection{Install and test the package}
Install and test the package:
\begin{lstlisting}
  user@blackarchlinux nfsshell $ pacman -U nfsshell-19980519-1-x86_64.pkg.tar.xz
  user@blackarchlinux nfsshell $ nfsshell # test it
\end{lstlisting}

\subsubsection{Add, commit and push package}
Add, commit and push the package
\begin{lstlisting}user@blackarchlinux nfsshell $ cd /blackarchlinux/packages
user@blackarchlinux ~/blackarchlinux/packages $ mv ~/nfsshell .
user@blackarchlinux ~/blackarchlinux/packages $ git commit -am nfsshell && git push
\end{lstlisting}

\subsubsection{Create a pull request}
Create a pull request on \href{https://github.com/}{github.com}
\begin{lstlisting}
  firefox https://github.com/<contributor>/blackarchlinux
\end{lstlisting}

\subsubsection{Adding a remote for upstream}
A smart thing to do if you're working upstream and on a fork is to pull your own fork and add the main ba repo as a remote
\begin{lstlisting}
  user@blackarchlinux ~/blackarchlinux $ git remote -v
  origin <the url of your fork> (fetch)
  origin <the url of your fork> (push)
  user@blackarchlinux ~/blackarchlinux $ git remote add upstream https://github.com/blackarch/blackarch
  user@blackarchlinux ~/blackarchlinux $ git remote -v
  origin <the url of your fork> (fetch)
  origin <the url of your fork> (push)
  upstream https://github.com/blackarch/blackarch (fetch)
  upstream https://github.com/blackarch/blackarch (push)
\end{lstlisting}

By default, git should push straight to origin, but make sure your git config is
configured correctly. This won't be an issue unless you have commit rights as
you won't be able to push upstream without them.

If you do have the ability to commit, you might have more success using
\textit{git@github.com:blackarch/blackarch.git} but it's up to you.

\subsection{Requests}
\begin{enumerate}
\item Don't add \textbf{Maintainer} or \textbf{Contributor} comments to
\textit{PKGBUILD} files. Add maintainer and contributor names to the
AUTHORS section of BlackArch guide.
\item For the sake of consistency, please follow the general style of the other
\textit{PKGBUILD} files in the repo and use two-space indentation.
\end{enumerate}

\subsection{General tips}
\href{http://wiki.archlinux.org/index.php/Namcap}{namcap} can check packages for
errors.

%------------------%
%  Chapter 4       %
%------------------%

\chapter{Tools Guide}
Coming soon...

\section{Coming Soon}
Coming soon...

%%% APPENDIX %%%
\appendix
\include{appendix}

\end{document}

%%% EOF %%%