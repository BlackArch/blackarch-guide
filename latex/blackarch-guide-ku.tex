%%%%%%%%%%%%%%%%%%%%%%%%%%%%%%%%%%%%%%%%%%%%%%%%%% %%%%%%%%%%%%%%%%%%%%%%%%%%%%%%
%%
% BlackArch Guide Linux%
%%
%%%%%%%%%%%%%%%%%%%%%%%%%%%%%%%%%%%%%%%%%%%%%%%%%% %%%%%%%%%%%%%%%%%%%%%%%%%%%%%%

\ Documentclass [a4paper, oneside, 11pt] {pirtûka}

%%% INCLUDES %%%
\ Renewcommand {\ familydefault} {\ sfdefault}

\ Usepackage {array}
\ Usepackage {color}
\ Usepackage {xaka}
\ Usepackage {fancyhdr}
\ Usepackage {fancyvrb}
\ Usepackage {geometrî}
\ Usepackage {graphicx}
\ Usepackage {html}
\ Usepackage {hyperref}
\ Usepackage {ifpdf}
\ Usepackage {veşêrim}
\ Usepackage {pstricks}
\ Usepackage {supertabular}
\ Usepackage {tocloft}
\ Usepackage [utf8] {inputenc}

%%% LAYOUT %%%
\ Setlength {\ parindent} {0em}
\ Setlength {\ parskip} {1.5ex plus0.5ex minus0.5ex}
\ Geometrî {çepê = 2.5cm, textwidth = 16cm, top = 3cm, textheight = 25cm, bottom = 3cm}
\ Widowpenalty = 2000
\ Clubpenalty = 1000
\ frenchspacing
\bêbal
\ Pagecolor [HTML] {#FFFFFF}
\ Color [HTML] {333333}
\ Setcounter {tocdepth} {10}
\ Setcounter {secnumdepth} {10}

\ Hypersetup {
  pdftitle = {BlackArch Linux, The BlackArch Guide Linux},
  pdfsubject = {BlackArch Linux, The BlackArch Guide Linux},
  pdfauthor = {BlackArch Linux, BlackArch Linux},
  pdfkeywords = {BlackArch Linux, Testing êzgeha, Ewlekariya, Hacking, Linux},
  pdfcenterwindow = true,
  colorlinks = true,
  breaklinks = true,
  linkcolor = red,
  menucolor = red,
  urlcolor = red
}

% Syntax ronahîyê
% Hemû options vir divê set belgekirina wide
\ Lstset {
backgroundcolor = \ color [HTML] {EEEEEE},
frame = yek,
basicstyle = \ footnotesize \ ttfamily,
avbazîn,
deletekeywords = {vegera},
otherkeywords = {mkdir, curl, sudo, sha1sum, grep, cut, sort, Wget, makepkg,
pacman, blackman},
keywordstyle = \ color {orange},
commentstyle = \ color {şîn},
stringstyle = \ color {sor},
zimanê = bash,
showspaces = false,
showtabs = false,
tabsize = 2
}

%%% HEADER / footer %%%
\ Setlength {\ headheight} {33pt}
\ Setlength {\ headsep} {33pt}
\ Lhead {{\ includegraphics [width = 1cm, height = 1cm] {images / logo.png tên}}}
\ Rhead {The Guide BlackArch Linux}

%%% macros CUSTOM %%%
% Ji bo girêdan html
\ Ifpdf \ din
\ Def \ href # 1 # 2 {\ htmladdnormallink {# 2} {1}}
\ fi

% ------------------%
% TITLE PAGE%
% ------------------%
\ Destpê {belge}
\ Pagestyle {vala}
\ Destpê {navenda}
\ Destpê {hejmar} [htbp]
\ bearing
\ Vspace {0.5cm}
\ Includegraphics [width = 8cm] {images / logo.png}
\ Label {hêjîrê: logo}
\ Dawiya {hejmar}
\ Vspace {0.5cm}
\ Huge {\ textbf {The BlackArch Guide Linux}} \\
\ Vspace {1cm}
\ Large {\ color {sor}} https://www.blackarch.org/ \\
\ Vspace {0.5cm}
\ Dawiya {navenda}
\ newpage
\ tableofcontents
\ newpage
\ Pagestyle {ketim}

% ------------------%
% Chapter 1%
% ------------------%

\ Beşa {DESTPÊK}

\ Beşa {Overview}
The rêberê BlackArch Linux nav çend beşan pêk tê:
\ Destpê {itemize}
\ Babete Pêşgotin - bidete wan nêrîneka berfireh, bi cî û din agahî projeya alîkar
\ Guide babete User - Her tişt a user tîpîk divê bizanibe ku ji bo bi bandor bi kar tînin BlackArch
\ Babete Guide Developer - How to dest get pêşxistina bo û beşdarbin BlackArch
\ Babete Tool Guide - Li-kûr hûragahiyan li tool li mînak bi boçûnên (WIP)
\ Dawiya {itemize}

\ Beşa {çi BlackArch Linux e?}
BlackArch dabeşkirina temam Linux ji bo kontrolkar û êzgeha û lêkolînerên ewlekariyê ye.
Ev ji \ href {https://www.archlinux.org/} {ArchLinux} Navdêr û bikarhêner dikarin pêkhateyên BlackArch saz bike
bi serê xwe yan li komên rasterast bi ser de jî.

The toolset wek Linux Arch belavkirin
\ Href {https://wiki.archlinux.org/index.php/Unofficial\_User\_Repositories}
{Ensîklopediya user ne fermî} da tu BlackArch li ser sazkirinê
installation Arch Linux heyî. Packages dibe ku bi serê xwe an jî bi sazkirin
liq.

Ensîklopediya her tim berfireh niha li ser \ href {https://www.blackarch.org/tools.html} {1300} Amûrên de.
Hemû Amûrên bi tûmî, berî ku ew ji codebase added ji bo parastina bi kalîteya ensîklopediya ceribandin.
% Divê zû salix rêbazên testkirina / prosedurên review code hwd.

\ Beşa {History of BlackArch Linux}
Ji nêzda...

\ Beşa {piþtgiriyê platformên}
Ji nêzda...

\ Beşa {Tev lê bibin}
Tu dikarî di têkiliyê de bi tîma BlackArch bikaranîna caddeyên li jêr bikin:

Website: \ url {https://www.blackarch.org/}

Mail: \ href {mailto: team@blackarch.org} {team@blackarch.org}

IRC: \ url {irc: //irc.freenode.net/blackarch}

Twitter: \ url {https://twitter.com/blackarchlinux}

Github: \ url {https://github.com/Blackarch/}

% ------------------%
% Chapter 2%
% ------------------%


\ Beşa {User Guide}

\ Beşa {Installation}
The beşên li jêr wê hûn çawa setup ensîklopediya BlackArch nîşan û
sazkirina paketên. BlackArch piştgiriya herdu, bernameya ji ensîklopediya bikaranîna
pakêtên binary herweha komkirin û sazkirin ji çavkaniyên.

BlackArch lihevhatî bi makîneyên Arch normal e. Ev wekî gayrî fermî tevbigere
ensîklopediya user. Heke tu dixwazî ​​an ISO li şûna, dîtina
\ Href {https://www.blackarch.org/downloads.html#iso} {Live ISO} beşa.

\ Bêgor {Damezrîna li ser ArchLinux}
Run \ href {https://blackarch.org/strap.sh} {strap.sh} wekî root û li pey
talîmatên. ji mînaka li jêr bibînin.
\ Destpê {lstlisting}
   curl -O https://blackarch.org/strap.sh
   sha1sum strap.sh # divê nagirin: 86eb4efb68918dbfdd1e22862a48fda20a8145ff
   sudo ./strap.sh
\ Dawiya {lstlisting}

Now a copy nû ya di lîsteyê de pakêta master download û cîhavên pakêtên:
\ Destpê {lstlisting}
  sudo pacman -Syyu
\ Dawiya {lstlisting}


\ Bêgor {pakêtên Sazkirin}
Tu niha dikarî xwe Amûrên ji ensîklopediya blackarch saz bike.
\ Destpê {xaka}
\ Babete To lîsteya hemû amûrên, run
\ Destpê {lstlisting}
  pacman -Sgg | grep blackarch | birrîn -d '' -f2 | sort genetîv
\ Dawiya {lstlisting}

\ Babete bo sazkirin û hemû amûrên, run
\ Destpê {lstlisting}
  blackarch pacman -S
\ Dawiya {lstlisting}

\ Babete bo sazkirin kategoriyê de ji amûrên, run
\ Destpê {lstlisting}
  pacman -S blackarch- <kategoriyê>
\ Dawiya {lstlisting}

\ Babete Bo bînînî kategoriyan blackarch, run
\ Destpê {lstlisting}
  pacman -Sg | blackarch grep
\ Dawiya {lstlisting}

\ Dawiya {xaka}

\ Bêgor {Sazkirina pakêtên ku ji source}
Wekî beşek ji rêbaza ku alternatîf yên damezrînê, tu dikarî BlackArch ava
pakêtên ku ji source. Hûn dikarin ji PKGBUILDs li bibînin
\ Href {https://github.com/BlackArch/blackarch/tree/master/packages} {github}. Ber
ava tevahiya repo, tu dikarî bikar bîne
\ Href {https://github.com/BlackArch/blackman} {Blackman} tool.
\ Destpê {itemize}
\ Babete First, we hene ji bo sazkirina Blackman. Ger ensîklopediya pakêta BlackArch
setup li gor xwe ye, tu dikarî sazkirina Blackman:
\ Destpê {lstlisting}
  pacman -S blackman
\ Dawiya {lstlisting}

\ Babete Tu dikarî avakirina û sazkirina Blackman ji source:
\ Destpê {lstlisting}
  blackman mkdir
  blackman cd
  Wget https://raw2.github.com/BlackArch/blackarch/master/packages/blackman/PKGBUILD
  # Piştrast bike ku PKGBUILD hatiye dîtin maliciously bi ramaneke ne.
  makepkg -s
\ Dawiya {lstlisting}

\ Babete An jî tu Blackman ji Aur saz bike:
\ Destpê {lstlisting}
  <Çi alîkarê Aur hûn bi kar> -S blackman
\ Dawiya {lstlisting}

\ Dawiya {itemize}

\ Bêgor {Bikaranîna Basic Blackman} Blackman pir hêsan de bi kar e, tevî ku alên cuda ji tu çi yî
dê tîpîk ji tiştekî wek pacman hêvî. Bikaranîna bingehîn hatiye li jêr de hatîye dîtin.
\ Destpê {itemize}
\ Babete Download, kom kirin û sazkirina paketên:
\ Destpê {lstlisting}
  sudo blackman pakêta -i
\ Dawiya {lstlisting}

\ Babete Download, raporekê û saz hemû category:
\ Destpê {lstlisting}
  sudo blackman -G koma
\ Dawiya {lstlisting}

\ Babete Download, raporekê û saz hemû amûrên BlackArch:
\ Destpê {lstlisting}
  sudo blackman -a
\ Dawiya {lstlisting}

\ Babete To lîsteya kategoriyan blackarch:
\ Destpê {lstlisting}
  blackman -l
\ Dawiya {lstlisting}

\ Babete To lîsteya Amûrên category:
\ Destpê {lstlisting}
  category blackman -p
\ Dawiya {lstlisting}

\ Dawiya {itemize}

\ Bêgor {sazkirin ji live-, ISO netinstall- an ArchLinux}
Tu dikarî BlackArch Linux ji yek ji cînayetên me an netinstall-ISOs saz bike. \\ Binêre
\ Url {https://www.blackarch.org/download.html#iso}. Ev gavên ku li jêr in
pêwîst piştî ISO boot xwe.
\ Destpê {itemize}
\ Babete sazkirinê pakêt blackarch-installer:
\ Destpê {lstlisting}
  sudo pacman -S blackarch-installer
\ Dawiya {lstlisting}

\ Babete Run
\ Destpê {lstlisting}
  sudo blackarch-saz bike
\ Dawiya {lstlisting}

\ Dawiya {itemize}

% ------------------%
% Chapter 3%
% ------------------%

\ Beşa {Guide Developer}

\ Beşa {System Build Arch û Repositories}

files PKGBUILD ne ava Skrîpta. Her yek makepkg (1) çawa biafirîne dibêje
pakêt. files PKGBUILD bi li Bash nivîsîn.

Ji bo agahiyên zêdetir, xwendin (an jî vê gotarê di riya) li jêr e:
\ Destpê {itemize}
\ Babete \ href {https://wiki.archlinux.org/index.php/Creating_Packages} {DEVICE Wiki: Packages Afrandina}
\ Babete \ href {https://wiki.archlinux.org/index.php/Makepkg} {DEVICE Wiki: makepkg}
\ Babete \ href {https://wiki.archlinux.org/index.php/PKGBUILD} {DEVICE Wiki: PKGBUILD}
\ Babete \ href {https://wiki.archlinux.org/index.php/Arch_Packaging_Standards} {DEVICE Wiki: Arch Packaging Standardî}
\ Dawiya {itemize}

\ Beşa {standardên Blackarch PKGBUILD}
Ji bo xatirê yên sadebûn, PKGBUILDs me mîna ku ji kesên ku Aur in,
bi çend cudahiyên biçûk ristê. Her pakêta divê
endamên blackarch li kêmtirîn, li wir jî dê gelek strukturan de bi be
pakêtên multiple ku mensûbê koman.

\ Bêgor {Groups}
To rê bikarhênerên bo sazkirina pakêteke taybet pakêtên ku zû û bi hêsanî,
pakêtên ku ji nav komên ji hev cuda kirin. Groups formę, bikarhęner dikarin bi asanî
go "pacman -s <name koma>" Ji bo vekêşana gelek pakêtên.

\ Subsubsection {blackarch}
Koma blackarch koma bingehîn ku hemû pakêtên ku divê jî girêdayî ye. Ev rê dide
users ji bo sazkirina her pakêta bi rihetî.

Her tişt divê çi li vir bin.

\ Subsubsection {blackarch-anti-tipa}
Pakêtên bi ku ji bo pêşî li çalakiyên tipa edlî tê bikaranîn,
di nav de şîfrekirinê, steganography, û tiştekî ku modifies files / file hildide.
Ev hemû tê de amûrên ji bo kar bi tiştekî giştî ku jî guhertinên di sîstema
ji bo armancên xwe vedişêrin, agahiyên.

Nimûne: Luks, TrueCrypt, Timestomp, dd, ropeadope, ewle-delete

\ Subsubsection {blackarch-automation}
Pakêtên bi ku ji bo tool an automation workflow bikaranîn.

Nimûne: blueranger, tiger, wiffy

\ Subsubsection {blackarch-Backdoor}
Pakêtên ku biçêrin an backdoors vekirî li ser heşaş Sîstemên berê.

Nimûne: Backdoor-factory, rrs, weevely

\ Subsubsection {blackarch-binary}
Pakêtên ku li ser pelên binary li hin form kar.

Nimûne: binwally, packerid

\ Subsubsection {blackarch-bluetooth}
Pakêtên ku biçêrin tiştekî li ser standard Bluetooth (802.15.1).

Nimûne: ubertooth, tbear, redfang

\ Subsubsection {blackarch-code-(Riksrevisionen)}
Pakêtên ku source code audit heyî ji bo analîza doh.

Nimûne: flawfinder, pscan

\ Subsubsection {blackarch-cracker}
Pakêtên ji bo cracking fonksiyonên Cryptographic, ango hashes.

Nimûne: hashcat, john, bidaya

\ Subsubsection {blackarch-crypto}
Pakêtên ku bi cryptography kar, ji bilî cracking.

Nimûne: ciphertest, xortool, SBD

\ Subsubsection {blackarch-da heye}
Pakêtên ku hilperike standina heye li ser hemû astê.

Nimûne: metacoretex, blindsql

\ Subsubsection {blackarch-Debuggera bi}
Pakêtên ku rê ji user ji bo dîtina çi bernameyeke taybetî "çi" li hevdem.

Nimûne: radare2, shellnoob

\ Subsubsection {blackarch-Decompiler}
Pakêtên ku hewl ji bo berevajîkirina a bernameya berhev nav kodê çavkanîyê.

Nimûne: flasm, jd-gui

\ Subsubsection {blackarch-xweparêz}
Pakêtên bi ku ji bo parastina a user ji avabûnên bikaranîn \ & êrîşên ji bikarhênerên din ra.

Nimûne: arpon, chkrootkit, sniffjoke

\ Subsubsection {blackarch-disassembler}
Ev dişibe e ji bo blackarch-Decompiler, û li wir dibe gelek dê bibe
ji bernameyên ku nav herdu jî wê bikevin, lê belê van pakêtan berhemên encam civînê
bêtir ji source code raw.

Nimûne: inguma, radare2

\ Subsubsection {blackarch-dos}
Pakêtên ku bi kar tînin dos (Înkarkirina Service) êrîşan.

Nimûne: 42zip, nkiller2

\ Subsubsection {blackarch-bêpîlot}
Pakêtên bi ku ji bo birêvebirina fîzîkî mihendisiya bikaranîn
bêmirov.

Nimûne: meshdeck, skyjack

\ Subsubsection {blackarch-kedxwariyê}
Pakêtên ku digire û însiyatîfa serketinên di bernameyên din an xizmetên.

Nimûne: Armitage, metasploit, zarp

\ Subsubsection {blackarch-iface}
Pakêtên ku biçêrin alavên biyometrîk tilîka.

Nimûne: dns-map, p0f, httprint

\ Subsubsection {blackarch-firmware}
Pakêtên ji kuştiyan li firmware biçêrin

Nimûne: None yet, sîxor asap.

\ Subsubsection {blackarch-tipa}
Pakêtên ku bi bikaranîn û ji bo dîtina welat li ser dîskên fizîkî yan jî bîra bicîbûyî de.

Nimûne: aesfix, nfex, karkuki

\ Subsubsection {blackarch-fuzzer}
Pakêtên ku bikaranîna prensîba testkirina Fuzz, ango
"Kevir" qurs random li mijarê ji bo dîtina ka çi diqewime.

Nimûne: MSF, mdk3, wfuzz

\ Subsubsection {blackarch-hardware}
Pakêtên ku biçêrin an bi rê ve tiştekî ji bo ku ez bi
hardware fîzîkî.

Nimûne: arduino, smali

\ Subsubsection {blackarch-honeypot}
Pakêtên ku wekî "honeypots" tevbigerin, ango bernameyên ku ji bo xuya
be xizmetên nazik bikaranîn ku bala hackers nav xefik.

Nimûne: topan, bluepot, wifi-honey

\ Subsubsection {blackarch-keylogger}
Pakêtên ku record û nebihûrin tikandî li ser sîstema din.

Nimûne: None yet, sîxor asap.

\ Subsubsection {blackarch-Pêşeka}
Pakêtên ku weke cûre yên nivîsbariyê mailên an count
detection Pêşeka.

Nimûne: malwaredetect, peepdf, yara

\ Subsubsection {blackarch-misc}
Pakêtên ku bi taybetî jî di nav ti categories ne bi kêr in.

Nimûne: oh-min-zsh-git, winexe, stompy

\ Subsubsection {blackarch-mobile}
Pakêtên ku destwerdanê di platformên mobile.

Nimûne: android-SDK-platform-Amûrên, android-udev-qaîdeyên

\ Subsubsection {blackarch-tora}
Pakêta ku hilperike tora IP.

Nimûne: Anything hema

\ Subsubsection {blackarch-NFC}
Pakêtên ku bi kar tînin NFC (bi beratên nêzîk-zeviyê).

Nimûne: nfcutils

\ Subsubsection {blackarch-packer}
Pakêtên ku li ser an Packers invlove kar.

\ Textit {Packers bernameyên ku embed avabûnên di nava din xebitandinê in.}

Nimûne: packerid

\ Subsubsection {blackarch-proxy}
Pakêtên ku wek qasid tevdigere, ango beralîkirin trafîkê
bi saya node din li ser înternetê.

Nimûne: burpsuite, ratproxy, sslnuke

\ Subsubsection {blackarch-rayven}
Pakêtên ku çalak digere dimije nazik di
bejî. More komeke sîwana bo pakêtên similar.

Nimûne: canri, dnsrecon, rûpoşa

\ Subsubsection {blackarch-çêv}
Ev komeke sîwana bo tu Decompiler e,
disassembler an tu bernameya similar.

Nimûne: capstone, radare2, zerowine

\ Subsubsection {blackarch-scanner}
Pakêtên ku bê dahûrandin sîstemên helbijartî ji bo kuştiyan.

Examples: scanssh, tiger, zmap

\subsubsection{blackarch-sniffer}
Packages that involve analyzing network traffic.

Examples: hexinject, pytactle, xspy

\subsubsection{blackarch-social}
Packages that primarily attack social networking sites.

Examples: jigsaw, websploit

\subsubsection{blackarch-spoof}
Packages that attempt to spoof the attacker such, in that
the attacker doesn't show up as an attacker to the victim.

Examples: arpoison, lans, netcommander

\subsubsection{blackarch-threat-model}
Packages that would be used for reporting/recording the
threat model outlined in a particular scenario.

Examples: magictree

\subsubsection{blackarch-tunnel}
Packages that are used to tunnel network traffic on a given
network.

Examples: ctunnel, iodine, ptunnel

\subsubsection{blackarch-unpacker}
Packages that are used to extract pre-packed malware from an
executable.

Examples: js-beautify

\subsubsection{blackarch-voip}
Packages that operate on voip programs and protocols.

Examples: iaxflood, rtp-flood, teardown

\subsubsection{blackarch-webapp}
Packages that operate on internet-facing applications.

Examples: metoscan, whatweb, zaproxy

\subsubsection{blackarch-windows}
This group is for any native Windows package that runs via wine.

Examples: 3proxy-win32, pwdump, winexe

\subsubsection{blackarch-wireless}
Packages that operates on wireless networks on any level.

Examples: airpwn, mdk3, wiffy

\section{Repository structure}
You can find the main BlackArch git repo here:
\href{https://github.com/BlackArch/blackarch}{https://github.com/BlackArch/blackarch}.
There are also several secondary repos here:
\href{https://github.com/BlackArch}{https://github.com/BlackArch}.

Within the main git repo, there are three important directories:

\begin{itemize}
\item docs - Documentation.
\item packages - PKGBUILD files.
\item scripts - Useful little scripts.
\end{itemize}

\subsection{Scripts}
Here is a reference for scripts in the \verb|scripts/| directory:

\begin{itemize}
\item baaur - Soon, this will upload packages to the AUR.
\item babuild - Build a package.
\item bachroot - Manage a chroot for testing.
\item baclean - Clean old .pkg.tar.xz files from the package repo.
\item baconflict - Soon this will replace \verb|scripts/conflicts|.
\item bad-files - Find bad files in built packages.
\item balock - Obtain or release the package repo lock.
\item banotify - Notify IRC about package pushes.
\item barelease - Release packages to the package repo.
\item baright - Print the BlackArch copyright info.
\item basign - Sign packages.
\item basign-key - Sign a key.
\item blackman - This behaves sort of like pacman but builds from git (not to be
    confused with nrz's Blackman).
\item check-groups - Check groups.
\item checkpkgs - Check packages for errors.
\item conflicts - Check for file conflicts.
\item dbmod - Modify a package database.
\item depth-list - Create a list sorted by dependency depth.
\item deptree - Create a dependency tree, listing only blackarch-provided packages.
\item get-blackarch-deps - Get a list of blackarch dependencies for a package.
\item get-official - Get official packages for release.
\item list-loose-packages - List packages that are not in groups and are not
    dependencies for other packages.
\item list-needed - List missing dependencies.
\item list-removed - List packages that are in the package repo but not in git.
\item list-tools - List tools.
\item outdated - Look for packages that are out-dated in the package repo
    relative to the git repo.
\item pkgmod - Modify a build package.
\item pkgrel - Increment pkgrel in a package.
\item prep - Clean up a PKGBUILD file's style and find errors.
\item sitesync - Sync between a local copy of the package repo and a remote copy.
\item size-hunt - Hunt for large packages.
\item source-backup - Backup package source files.
\end{itemize}

\section{Contributing to repository}
This section shows you how to contribute to the BlackArch Linux project. We
accept pull requests of all sizes, from tiny typo fixes to new packages.\\For
help, suggestions, or questions feel free to contact us.
\\\\
Everyone is welcome to contribute. All contributions are appreciated.

\subsection{Required tutorials}
Please read the following tutorials before contributing:
\begin{itemize}
\item
\href{https://wiki.archlinux.org/index.php/Arch\_Packaging\_Standards)}{Arch
Packaging Standards}
\item \href{https://wiki.archlinux.org/index.php/Creating\_Packages}{Creating
Packages}
\item \href{https://wiki.archlinux.org/index.php/PKGBUILD}{PKGBUILD}
\item \href{https://wiki.archlinux.org/index.php/Makepkg}{Makepkg}
\end{itemize}

\ Bêgor {Steps ji bo beşdariya}
Ji bo submit guhertinên ku we ji bo vê projeyê BlackArchLinux, li pey van
gavên:
\ Destpê {xaka}
\ Babete Fork ensîklopediya ji
\ Url {https://github.com/BlackArch/blackarch}
\ Babete Hack pelên pêwîst, (raxîne. PKGBUILD, files .patch, û hwd.).
\ Babete Commit Guherandinên xwe.
\ Babete Push Guherandinên xwe.
\ Babete ji me bipirse merge li Guherandinên te, û germahıya bi daxwaza vekêşana.
\ Dawiya {xaka}

\ Bêgor {mînak}
Mînaka jêrîn nîşan dide radest a pakêta nû ji bo BlackArch
rêvename. Em \ href bi kar {https://wiki.archlinux.org/index.php/yaourt} {yaourt}
(Tu baş bi kar pacaur) xeberdanê a file PKGBUILD pre-heyî ji bo
\ Textbf {nfsshell} ji \ href {https://aur.archlinux.org/} {Aur} û eyar ew
li gor hewcedariyên me.

\ Subsubsection {Fetch PKGBUILD}
Pędivî bi \ textit {PKGBUILD} pelê bi kar yaourt an pacaur:
\ Destpê {lstlisting}
  user @ blackarchlinux $ yaourt -G nfsshell
  ==> çavkaniyên Download nfsshell
  x LICENSE
  x PKGBUILD
  x gcc.patch
  user @ blackarchlinux $ cd nfsshell /
\ Dawiya {lstlisting}

\ Subsubsection {paqij bikin PKGBUILD}
Paqij bikin û bi \ textit {PKGBUILD} file û rizgarkirina hin dem:
\ Destpê {lstlisting}
  user @ blackarchlinux nfsshell $ ./blarckarch/scripts/prep PKGBUILD
  paqijkirina 'PKGBUILD' ...
  berfireh hilpekînên ...
  raneke modeline vim ...
  raneke Rayi id ...
  raneke, tenya û Pêşdebir comments ...
  ditepisînin xetên zêde vala ...
  raneke '|| vegerr'...
  raneke sereke line vala ...
  raneke $ pkgname ...
  şop û dirêj whitespace ...
\ Dawiya {lstlisting}

\ Subsubsection {Adjust PKGBUILD}
Eyar \ textit li {PKGBUILD} file:
\ Destpê {lstlisting}
  user @ blackarchlinux nfsshell $ vi PKGBUILD
\ Dawiya {lstlisting}

\ Subsubsection {Build pakêta}
Avakirina vê pakêtê de:
\ Destpê {lstlisting} user @ blackarchlinux nfsshell $ makepkg -sf
==> Making pakêta: nfsshell 19980519-1 (Mon Dec 2 17:23:51 CET 2013)
==> girêdanên Kontrol runtime ...
==> girêdanên Kontrol buildtime ...
==> Retrieving çavkaniyên ...
-> Daxistina nfsshell.tar.gz ...
% Total% pêşwazî% Xferd Average Speed ​​Time Time Time
CurrentDload Upload Total Spent Çep Speed100 29213 100 29213 0
0 48150 0 -: -: -: -: -: -: -: -: - 48206
-> gcc.patch hatin
-> LICENSE hatin
...
<Gelek pêvajoya build û encam berhevkar here>
...
==> Leaving jîngehê fakeroot.
==> çêkirina Dawî: nfsshell 19980519-1 (Mon Dec 2 17:23:53 CET 2013)
\ Dawiya {lstlisting}

\ Subsubsection {Install û test pakêta}
Install û ceribandina vê pakêtê de:
\ Destpê {lstlisting}
  user @ blackarchlinux nfsshell $ pacman -U nfsshell-19980519-1-x86_64.pkg.tar.xz
  user @ blackarchlinux nfsshell $ nfsshell # biceribîne
\ Dawiya {lstlisting}

\ Subsubsection {lê zêde bike, commit û pakêta push}
Lê zêde bike, commit û konaxa package
\ Destpê {lstlisting} user @ blackarchlinux nfsshell $ cd / blackarchlinux / pakêtên
user @ blackarchlinux ~ / blackarchlinux / pakêtên $ mv ~ / nfsshell.
user @ blackarchlinux ~ / blackarchlinux / pakêtên $ git commit nfsshell ere && push git
\ Dawiya {lstlisting}

\ Subsubsection {Create a bi daxwaza vekêşana}
Create daxwaza vekêşana li ser \ href {https://github.com/} {github.com}
\ Destpê {lstlisting}
  firefox https://github.com/<contributor>/blackarchlinux
\ Dawiya {lstlisting}

\ Subsubsection {zêdekirin a ji dûr ve ji bo diherikî}
A tiştê smart to do, eger hûn dixebitin diherikî û li ser duryanekê ye ji bo vekêşana fork xwe bi xwe û lê zêde bike ji repo ba sereke wek dûr
\ Destpê {lstlisting}
  user @ blackarchlinux ~ / blackarchlinux $ git -yê dûr
  eslê <url of fork xwe> (xeberdanê)
  eslê <url of fork xwe> (push)
  user @ blackarchlinux ~ / blackarchlinux $ git ji dûr ve lê zêde bike diherikî https://github.com/blackarch/blackarch
  user @ blackarchlinux ~ / blackarchlinux $ git -yê dûr
  eslê <url of fork xwe> (xeberdanê)
  eslê <url of fork xwe> (push)
  https://github.com/blackarch/blackarch upstream (xeberdanê)
  https://github.com/blackarch/blackarch upstream (push)
\ Dawiya {lstlisting}

Wekî standard, git, divê rasterast ji eslê xwe inkar bike, lê belê bila config git te ye
kardike avakirin. Ev wê ne pirseke heta tu tewanê maf wekî
hûn ne wê bikaribin push herikî bê wan.

Eger hûn xwedî şiyana ku qetilirin, hûn serkeftina zêdetir bikaranîna hene
\ Textit {git@github.com: blackarch / blackarch.git} lê belê ev ji bo we.

\ Bêgor {Requests}
\ Destpê {xaka}
\ Babete Ma \ textbf {Miqate yê} an jî \ textbf {Contributor} comments to zêde ne
\ Textit {PKGBUILD} files. Lê zêde bike maintainer û tenya navên bi
Nivîs- beşa rêber BlackArch.
\ Babete ji bo xatirê yên hevgirtî, ji kerema xwe ve li pey style giştî ya din
\ Textit {PKGBUILD} files di repo û bikaranîna indentation du-space.
\ Dawiya {xaka}

\ Bêgor {tips General}
\ Href {http://wiki.archlinux.org/index.php/Namcap} {namcap} dikarin pakêtên ku ji bo kontrol
de çewtî.

% ------------------%
% Chapter 4%
% ------------------%

\ {Guide Tools} beş
Ji nêzda...

\ Beşa {Coming Soon}
Ji nêzda...

%%% PAŞKOYA %%%
\revîyê kor
\ Di nav {hevpêçekî}

\ Dawiya {belge}

%%% EOF %%%
