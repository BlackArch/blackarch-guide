%%%%%%%%%%%%%%%%%%%%%%%%%%%%%%%%%%%%%%%%%%%%%%%%%%%%%%%%%%%%%%%%%%%%%%%%%%%%%%%%
%                                                                              %
% BlackArch Linux Handbuch                                                     %
%                                                                              %
%%%%%%%%%%%%%%%%%%%%%%%%%%%%%%%%%%%%%%%%%%%%%%%%%%%%%%%%%%%%%%%%%%%%%%%%%%%%%%%%

\documentclass[a4paper, oneside, 11pt]{book}

%%% INCLUDES %%%
\renewcommand{\familydefault}{\sfdefault}

\usepackage{array}
\usepackage{color}
\usepackage{enumerate}
\usepackage{fancyhdr}
\usepackage{fancyvrb}
\usepackage{geometry}
\usepackage{graphicx}
\usepackage{hyperref}
\usepackage{ifpdf}
\usepackage{listings}
\usepackage{pstricks}
\usepackage{supertabular}
\usepackage{tocloft}
\usepackage[utf8]{inputenc}
\usepackage[T1]{fontenc}
\usepackage{fullpage}
\usepackage[ngerman]{babel}
\usepackage{graphicx}
\clubpenalty10000
\widowpenalty10000
\displaywidowpenalty=10000
\usepackage{ltablex}

\keepXColumns

%%% LAYOUT %%%
\setlength{\parindent}{0em}
\setlength{\parskip}{1.5ex plus0.5ex minus0.5ex}
\geometry{left=2.5cm, textwidth=16cm, top=3cm, textheight=25cm, bottom=3cm}
\widowpenalty=2000
\clubpenalty=1000
\frenchspacing
\sloppy
\pagecolor[HTML]{FFFFFF}
\color[HTML]{333333}
\setcounter{tocdepth}{10}
\setcounter{secnumdepth}{10}

\hypersetup{
  pdftitle={BlackArch Linux, Das BlackArch Linux Handbuch},
  pdfsubject={BlackArch Linux, Das BlackArch Linux Handbuch},
  pdfauthor={BlackArch Linux, BlackArch Linux},
  pdfkeywords={BlackArch Linux, Penetration Testing, Security, Hacking, Linux},
  pdfcenterwindow=true,
  colorlinks=true,
  breaklinks=true,
  linkcolor=red,
  menucolor=red,
  urlcolor=red
}

% syntax highlighting
% all options should be set here document wide
\lstset{
backgroundcolor=\color[HTML]{EEEEEE},
frame=single,
basicstyle=\footnotesize\ttfamily,
float,
deletekeywords={return},
otherkeywords={mkdir, curl, sudo, sha1sum, grep, cut, sort, wget, makepkg,
pacman, blackman},
keywordstyle=\color{orange},
commentstyle=\color{blue},
stringstyle=\color{red},
language=bash,
showspaces=false,
showtabs=false,
tabsize=2,
texcl=true
breaklines=true
}

%%% HEADER / FOOTER %%%
\setlength{\headheight}{33pt}
\setlength{\headsep}{33pt}
\lhead{{\includegraphics[width=1cm,height=1cm]{images/logo.png}}}
\rhead{Das BlackArch Linux Handbuch}

%%% CUSTOM MACROS %%%
% for html links
\ifpdf\else
\def\href#1#2{\htmladdnormallink{#2}{#1}}
\fi

%------------------%
%  TITLE PAGE      %
%------------------%
\begin{document}
\pagestyle{empty}
\begin{center}
\begin{figure}[htbp]
\centering
\vspace{0.5cm}
\includegraphics[width=8cm]{images/logo.png}
\label{fig:logo}
\end{figure}
\vspace{0.5cm}
\Huge{\textbf{The BlackArch Linux Guide}}\\
\vspace{1cm}
\Large{\color{red}https://www.blackarch.org/}\\
\vspace{0.5cm}
\end{center}
\newpage
\tableofcontents
\newpage
\pagestyle{fancy}

%------------------%
%  Chapter 1       %
%------------------%

\chapter{Einführung}

\section{Übersicht}
Das BlackArch Linux Handbuch ist in verschiedene Teile aufgeteilt:
\begin{itemize}
\item Einführung - Gibt einen breiten Überblick, eine Einführung, und weitere hilfreiche Projektinformationen
\item Nutzerhandbuch - Alles was ein typischer Nutzer wissen muss um BlackArch zu benutzen
\item Entwicklerhandbuch - Wie kann man zu BlackArch beitragen und entwickeln
\item Tool Guide - Tiefgehende Details zu Tools und Beispiele zur Benutzung (WIP)
\end{itemize}

\section{Was ist BlackArch Linux?}
BlackArch ist eine vollwertige Linux Distribution für Penetration Tester und Security Researcher.
Es basiert auf \href{https://www.archlinux.org/}{ArchLinux} und Nutzer können BlackArch Komponenten einzeln oder in Gruppen installieren.

Das Toolset wird mittel eines inoffiziellem Benutzer Repositories verteilt, so dass man BlackArch auf einer existierenden Arch Linux Installation installieren kann. Pakete können individuell oder über Kategorien installiert werden.
\href{https://wiki.archlinux.org/index.php/Unofficial\_User\_Repositories}
{Inoffizielles Nutzerrepository}

Das konstant wachsende Repository beinhaltet aktuell über \href{https://www.blackarch.org/tools.html}{1300} tools.
Alle tools werden intensiv getestet bevor sie zur Codebasis hinzugefügt werden, um die Qualität des Repositories zu gewährleisten.
% should quickly describe the testing methods/code review procedures etc

\section{Geschichte von BlackArch Linux}
Coming soon...

\section{Unterstützte Plattformen}
Coming soon...

\section{Mitmachen}
Man kann über folgende Wege mit dem BlackArch Team in Kontakt treten:

Website: \url{https://www.blackarch.org/}

Mail: \href{mailto:team@blackarch.org}{team@blackarch.org}

IRC: \url{irc://irc.freenode.net/blackarch}

Twitter: \url{https://twitter.com/blackarchlinux}

Github: \url{https://github.com/Blackarch/}

%------------------%
%  Kapitel 2       %
%------------------%


\chapter{Benutzerhandbuch}

\section{Installation}
Der folgende Abschnitt zeigt, wie man das BlackArch Repository einrichtet und Pakete installiert. BlackArch unterstützt sowohl die Installation von Binärpaketen als auch die Installation über selbstkompilierten Quellcode. 

BlackArch ist kompatibel mit regulären Arch installationen. Es verhält sich wie ein inoffizelles Nutzerreporisotry.
Wenn stattdessen ein ISO benötigt wird, siehe den Abschnitt
\href{https://www.blackarch.org/downloads.html#iso}{Live ISO}.

\subsection{Installation basierend auf einer vorhandenen ArchLinux Installation}
Führe \href{https://blackarch.org/strap.sh}{strap.sh} als root aus und folge den Anweisungen. 

Hier ein Beispiel.
\begin{lstlisting}
   curl -O https://blackarch.org/strap.sh
   sha1sum strap.sh # should match: 86eb4efb68918dbfdd1e22862a48fda20a8145ff
   sudo ./strap.sh
\end{lstlisting}

Jetzt lade eine frische Kopie der Master Paket Liste und synchronisiere die Pakete:
\begin{lstlisting}
  sudo pacman -Syyu
\end{lstlisting}


\subsection{Paketinstallattion}
Jetzt können Tools aus dem BlackArch Repository installiert werden.
\begin{enumerate}
\item Um alle verfügbaren Tools aufzulisten:
\begin{lstlisting}
  pacman -Sgg | grep blackarch | cut -d' ' -f2 | sort -u
\end{lstlisting}

\item Um alle Tools zu installieren:
\begin{lstlisting}
  pacman -S blackarch
\end{lstlisting}

\item Um eine Toolkategorie zu installieren:
\begin{lstlisting}
  pacman -S blackarch-<category>
\end{lstlisting}

\item Um die BlackArch Kategorien zu sehen:
\begin{lstlisting}
  pacman -Sg | grep blackarch
\end{lstlisting}

\end{enumerate}

\subsection{Paketinstallation auf Quellcodebasis}
Alternativ können BlackArch-Pakete auch aus Quellcode gebaut werden. Die PKGBUILDS können auf \href{https://github.com/BlackArch/blackarch/tree/master/packages}{github} gefunden werden. Um das gesamte Repository zu bauen, kann das \href{https://github.com/BlackArch/blackman}{Blackman} tool genutzt werden.
\begin{itemize}
\item Als erste muss Blackman installiert werden. Wenn das BlackArch Reposity auf ihrer Maschine eingerichtet ist, kann Blackman installiert werden:
\begin{lstlisting}
  pacman -S blackman
\end{lstlisting}

\item Blackman kann von Quellcode gebaut und installiert werden: 
\begin{lstlisting}
  mkdir blackman
  cd blackman
  wget https://raw2.github.com/BlackArch/blackarch/master/packages/blackman/PKGBUILD
 # Sicherstellen, dass die PKGBUILD nicht bösartig verändert worden sind.
  makepkg -s
\end{lstlisting}

\item Blackman kann auch aus dem AUR installiert werden:
\begin{lstlisting}
  <Verwendeter AUR Helfer> -S blackman
\end{lstlisting}

\end{itemize}

\subsection{Grundlegende Verwendung von Blackman} Blackman ist sehr einfach zu nutzen, auch wenn sich die flags von dem unterscheiden, was man typischerweise von pacman erwarten würde. 
Die Grundlegende Benutzung wird im folgenden gezeigt.
\begin{itemize}
\item Herunterladen, kompilieren and installieren von Paketen:
\begin{lstlisting}
  sudo blackman -i package
\end{lstlisting}

\item Herunterladen, kompilieren und installieren einer ganzen Kategorie:
\begin{lstlisting}
  sudo blackman -g group
\end{lstlisting}

\item Herunterladen, kompilieren und installieren aller BlackArch Tools:
\begin{lstlisting}
  sudo blackman -a
\end{lstlisting}

\item Auflistung aller BlackArch Kategorien:
\begin{lstlisting}
  blackman -l
\end{lstlisting}

\item Auflistung der Tools einer Kategorie:
\begin{lstlisting}
  blackman -p category
\end{lstlisting}

\end{itemize}

\subsection{Installing from live-, netinstall- ISO or ArchLinux}
BlackArch Linux kann von unseren live- oder netinstall-ISOs intalliert werden. \\Siehe
\url{https://www.blackarch.org/download.html#iso}. Die folgenden Schritte sind nötig wenn die ISO gebootet ist.
\begin{itemize}
\item Installieren des blackarch-installer Pakets:
\begin{lstlisting}
  sudo pacman -S blackarch-installer
\end{lstlisting}

\item Run
\begin{lstlisting}
  sudo blackarch-install
\end{lstlisting}

\end{itemize}

%------------------%
%  Chapter 3       %
%------------------%

\chapter{Entwicklerhandbuch}

\section{Das Arch Build System und Repositories}

PKGBUILD Dateien sind Build Skripte. Jedes beschreibt makepkg(1) wie ein Paket gebaut wird. PKGBUILD Dateien werden in Bash geschrieben.

Für weitere Informationen, lese (oder überfliege) folgende Seiten:
\begin{itemize}
\item \href{https://wiki.archlinux.org/index.php/Creating_Packages}{Arch Wiki: Erzeuge Packages}
\item \href{https://wiki.archlinux.org/index.php/Makepkg}{Arch Wiki: makepkg}
\item \href{https://wiki.archlinux.org/index.php/PKGBUILD}{Arch Wiki: PKGBUILD}
\item \href{https://wiki.archlinux.org/index.php/Arch_Packaging_Standards}{Arch Wiki: Arch Packetierungs Standards}
\end{itemize}

\section{Blackarch PKGBUILD Standards}
Der Einfachkeit halber sind unsere PKGBUILDs dem des AUR sehr ähnlich, die kleinen Unterschiede werden im weiteren Text beschrieben. 
Jedes Paket muss mindestens zu blackarch gehören, es wird aber auch viele beziehungen über mehrere pakete die zu mehreren Gruppen gehören geben.

\subsection{Gruppen}
Um es Nutzern zu ermöglichen eine ganze Reihe von Paketen schnell und einfach zu installieren,
wurden Pakete in Gruppen eingeteilt.
Gruppen ermöglichen es den benutzern mit einem einfachen "pacman -S <group name>" eine Menge von Paketen zu bekommen.

\subsubsection{blackarch}
Die blackarch gruppe ist die basis-Gruppe zu der alle Pakete gehören müssen. 
Das ermöglicht es den Nutzern einfach alle Pakete zu installieren.

Was sollte hier drin sein: Alles.

\subsubsection{blackarch-anti-forensic}
Pakete die dazu benutzt werden, forensische Aktivitäten zu umgehen. Das beinhaltet Verschlüsselung, Steganographie 
und alles was es ermöglicht Datei/Ordner Attribute zu manipulieren.
Das alles beinhaltet Tools die allgemein veränderungen an einem System durchführen mit dem Zweck,
Information zu verstecken. 

Beispiele: luks, TrueCrypt, Timestomp, dd, ropeadope, secure-delete

\subsubsection{blackarch-automation}
Pakete zur tool oder workflow Automatisierung.

Beispiele: blueranger, tiger, wiffy

\subsubsection{blackarch-backdoor}
Pakete zur Ausnutzung oder Öffnen von backdoors auf bereits verwundbaren
Systemen.

Beispiele: backdoor-factory, rrs, weevely

\subsubsection{blackarch-binary}
Pakete die auf irgendwelchen Binärdateien arbeiten.

Beispiele: binwally, packerid

\subsubsection{blackarch-bluetooth}
Pakete die alles exploiten was mit dem Bluetooth Standard (802.15.1) zu tun hat.

Beispiele: ubertooth, tbear, redfang

\subsubsection{blackarch-code-audit}
Pakete die bestehenden Code analysieren um Sicherheitslücken zu finden.

Beispiele: flawfinder, pscan

\subsubsection{blackarch-cracker}
Pakete die zum cracken von kryptographischen Funktionen, zum Beispiel Hashes.

Beispiele: hashcat, john, crunch

\subsubsection{blackarch-crypto}
Pakete die mit kryptographie arbeiten, mit der Ausnahme vom cracken.

Beispiele: ciphertest, xortool, sbd

\subsubsection{blackarch-database}
Pakete die Datenbank-Exploits auf jedem Level betreffen.

Beispiele: metacoretex, blindsql

\subsubsection{blackarch-debugger}
Pakete die es dem Nutzer erlauben in Echtzeit zu sehen, was ein bestimmtes Programm tut.

Beispiele: radare2, shellnoob

\subsubsection{blackarch-decompiler}
Pakete die versuchen kompilierte Programm in Quellcode zu konvertieren.

Beispiele: flasm, jd-gui

\subsubsection{blackarch-defensive}
Pakete die Versuchen den Nutzer vor Malware und Attacken anderer Nutzer zu schützen.

Beispiele: arpon, chkrootkit, sniffjoke

\subsubsection{blackarch-disassembler}
Ähnlich zu blackarch-decompiler; Hier gibt es vermutlich einige Programme die
in beide Kategorien fallen, mit dem Unterschied das diese Pakete Assembler ausgeben
statt den puren Quellcode.

Beispiele: inguma, radare2

\subsubsection{blackarch-dos}
Pakete die DoS (Denial of Service) Angriffe nutzen.

Beispiele: 42zip, nkiller2

\subsubsection{blackarch-drone}
Pakete die zur Verwaltung von echten Drohnen verwendet werden.

Beispiele: meshdeck, skyjack

\subsubsection{blackarch-exploitation}
Pakete die exploits anderer Programme oder Dienste nutzen.

Beispiele: armitage, metasploit, zarp

\subsubsection{blackarch-fingerprint}
Pakete die Fingerabdrücke biometrischer Systeme exploiten.

Beispiele: dns-map, p0f, httprint

\subsubsection{blackarch-firmware}
Pakete die Schwachstellen in Firmware ausnutzen.

Beispiele: Noch keine, asap hinzufügen.

\subsubsection{blackarch-forensic}
Pakete die benutzt werden um Daten auf physischen Festplatten oder Speicher zu finden.

Beispiele: aesfix, nfex, wyd

\subsubsection{blackarch-fuzzer}
Pakete die die Fuzzy Testprinzipien nutzen, zum Beispiel
zufälligen Input "reinzuwerfen" und zu sehen was passiert.

Beispiele: msf, mdk3, wfuzz

\subsubsection{blackarch-hardware}
Pakete die alles verwalten oder ausnutzen was mit
physischer Hardware zu tun hat.

Beispiele: arduino, smali

\subsubsection{blackarch-honeypot}
Pakete die als "honeypots" fungieren. Zum Beispiel Programme
die sich als verwundbare Dienste ausgeben und Hacker in eine Falle locken
sollen.

Beispiele: artillery, bluepot, wifi-honey

\subsubsection{blackarch-keylogger}
Pakete die Tastendrücke auf anderen Systemen aufnehmen und speichern.

Beispiele: None yet, amend asap.

\subsubsection{blackarch-malware}
Pakete die zu Malware zählen oder Malware erkennung.

Beispiele: malwaredetect, peepdf, yara

\subsubsection{blackarch-misc}
Pakete die nicht unbedingt in eine spezielle Kategorie passen.

Beispiele: oh-my-zsh-git, winexe, stompy

\subsubsection{blackarch-mobile}
Pakete die Mobile Plattformen manipulieren.

Beispiele: android-sdk-platform-tools, android-udev-rules

\subsubsection{blackarch-networking}
Pakete die IP Netzerke betreffen.

Beispiele: TODO

\subsubsection{blackarch-nfc}
Pakete die NFC (near-field communication) nutzen.

Beispiele: nfcutils

\subsubsection{blackarch-packer}
Pakete die Packer bedienen oder beinhalten.

\textit{Packer sind Programme die malware in anderen Executables einbetten. }

Beispiele: packerid

\subsubsection{blackarch-proxy}
Pakete die als Proxy fungieren, also zum Beispiel Netzwerkverkehr durch einen
anderen Knoten im Internet umleiten.

Beispiele: burpsuite, ratproxy, sslnuke

\subsubsection{blackarch-recon}
Pakete die aktiv verwundbare exploits suchen.
Eine Obergruppe für ähnliche Pakete.

Beispiele: canri, dnsrecon, netmask

\subsubsection{blackarch-reversing}
Ûbergruppe für jegliche decompiler, 
disassembler oder ähnliche Programme.

Beispiele: capstone, radare2, zerowine

\subsubsection{blackarch-scanner}
Pakete die ausgewählte Systeme auf Schwachstellen scannen.

Beispiele: scanssh, tiger, zmap

\subsubsection{blackarch-sniffer}
Pakete die mit dem analysieren von Netzwerkverkehr
zu tun haben.

Beispiele: hexinject, pytactle, xspy

\subsubsection{blackarch-social}
Pakete die hauptsächlich soziale netzwerke angreifen.

Beispiele: jigsaw, websploit

\subsubsection{blackarch-spoof}
Pakete die versuchen den Angreifer zu spoofen, 
sodass der Angreifer nicht als Angreifer fure das Opfer zu 
erkennen ist.

Beispiele: arpoison, lans, netcommander

\subsubsection{blackarch-threat-model}
Pakete die zum Reporten/Aufnehmen des Threat-Models
in einem speziellen Szenario benutzt werden.

Beispiele: magictree

\subsubsection{blackarch-tunnel}
Pakete die dazu genutzt werden, Netzwerkverkehr zu einem
gegebenen Netzwerk zu tunneln.

Beispiele: ctunnel, iodine, ptunnel

\subsubsection{blackarch-unpacker}
Pakete die dazu genutzt werden, vorgepackten Schadcode von 
einer executable auszupacken.

Beispiele: js-beautify

\subsubsection{blackarch-voip}
Pakete die auf VOIP Programmen und Protokollen arbeiten.

Beispiele: iaxflood, rtp-flood, teardown

\subsubsection{blackarch-webapp}
Pakete die auf internet-zugewandten Anwendungen arbeiten.

Beispiele: metoscan, whatweb, zaproxy

\subsubsection{blackarch-windows}
Diese Gruppe ist für native Windows Pakete die unter wine laufen.

Beispiele: 3proxy-win32, pwdump, winexe

\subsubsection{blackarch-wireless}
Pakete die auf drahtlosen Netzwerken arbeiten.

Beispiele: airpwn, mdk3, wiffy

\section{Repository Struktur}
Das primäre git repo für BlackArch befindet sich hier:
\href{https://github.com/BlackArch/blackarch}{https://github.com/BlackArch/blackarch}.
Es gibt ausserdem verschiedene Sekundäre Repositories hier:
\href{https://github.com/BlackArch}{https://github.com/BlackArch}.

Innerhalb des Hauptrepos gibt es drei wichtige Verzeichnisse:

\begin{itemize}
\item docs - Dokumentation.
\item packages - PKGBUILD Dateien.
\item scripts - Nützliche kleine Skripte.
\end{itemize}

\subsection{Scripts}
Hier eine Referenz für Skripte im \verb|scripts/| Verzeichnis:

\begin{itemize}
\item baaur - Coming soon: Wird Pakete in das AUR hochladen.
\item babuild - Baut ein Paket.
\item bachroot - Managen eines chroot zum testen.
\item baclean - Räumt alte .pkg.tar.xz Dateien aus dem Paket Repository.
\item baconflict - Wird bald \verb|scripts/conflicts| ersetzen.
\item bad-files - Findet schlechte Dateien in gebauten Paketen.
\item balock - Anlegen oder lösen des Repository locks.
\item banotify - IRC benachrichtigen über Paket pushes.
\item barelease - Veröffentlicht Pakete in das Repository.
\item baright - Gibt die BlackArch Copyright Informationen aus.
\item basign - Signiert Packete.
\item basign-key - Signiert einen Schlüssel.
\item blackman - Verhält sich ähnlich wie pacman, baut aber aus git. 
	(Nicht zu verwechseln mit nrz's Blackman)
\item check-groups - Überprüft groups.
\item checkpkgs - Überprüft Pakete auf Fehler.
\item conflicts - Sucht nach Dateikonflikten.
\item dbmod - Modifiziert eine Paketdatenbank.
\item depth-list - Erzeugt eine Liste sortiert nach Abhängigkeitspfad.
\item deptree - Erzeugt einen Abhängigkeitsbaum, der nur blackarch Pakete enthält.
\item get-blackarch-deps - Liefert eine List von blackarch Abhängigkeiten für ein Paket.
\item get-official - Liefert offizielle Pakete zum Release.
\item list-loose-packages - Listet Pakete die weder in Gruppen noch Abhängigkeiten anderer Pakete sind.
\item list-needed - Liste fehlender Abhängigkeiten.
\item list-removed - Liste von Pakete die im Paketrepository sind aber nicht im git.
\item list-tools - Liste der Tools.
\item outdated - Sucht nach veralteten Paketen im Repository im Vergleich zum git Repository.
\item pkgmod - Modifiziert ein Buildpaket.
\item pkgrel - Zählt die pkgrel in einem Paket hoch.
\item prep - Aufräumen des PKGBUILD Datei-Styles und Fehlersuche.
\item sitesync - Synchronisiert zwischein einer lokalen Kopie des Paketrepositories und der Remote.
\item size-hunt - Sucht nach grossen Paketen.
\item source-backup - Backup von package source Dateien.
\end{itemize}

\section{Beitragen zum BlackArch Repository}
Dieser Abschnitt zeigt, wie Beiträge im BlackArch Linux Projekt gemacht werden.
wir akzeptieren Pull Requests jeglicher Grösse, von kleinen Tippfehler-Korrekturen 
bis zu neuen Paketen. \\Für Hilfe, Vorschlage oder Fragen Kontaktiere uns. 
\\\\
Jeder ist willkommen. Alle Beiträge werden geschätzt.

\subsection{Benötigte Tutorials}
Bitte lies folgende Tutorials bevor du mitmachst:
\begin{itemize}
\item
\href{https://wiki.archlinux.org/index.php/Arch\_Packaging\_Standards)}{Arch
Packaging Standards}
\item \href{https://wiki.archlinux.org/index.php/Creating\_Packages}{Paketerzeugung}
\item \href{https://wiki.archlinux.org/index.php/PKGBUILD}{PKGBUILD}
\item \href{https://wiki.archlinux.org/index.php/Makepkg}{Makepkg}
\end{itemize}

\subsection{Schritte zum Mitmachen}
Um Änderungen zum BlackArchLinux Projekt zu submitten, folge diesen Schritten:
steps:
\begin{enumerate}
\item Fork das Repository von
\url{https://github.com/BlackArch/blackarch}
\item Hacke die benötigten Dateien (z.B. PKGBUILD, .patch files, usw).
\item Committe deine Änderungen.
\item Pushe deine Äderungen.
\item Bitte uns darum deine changes zu mergen, am liebsten durch einen Pull Request.
\end{enumerate}

\subsection{Beispiel}
Das folgende Beispiel zeigt, wie ein neues Paket zum BlackArch Projekt
submitted wird.
Wir benutzen \href{https://wiki.archlinux.org/index.php/yaourt}{yaourt}
(pacaur kann auch benutzt werden) um eine bereits existierende PKGBUILD Datei für
\textbf{nfsshell} aus dem \href{https://aur.archlinux.org/}{AUR} herunter zu laden und nach unseren
Bedürfnissen anzupassen.

\subsubsection{Fetch PKGBUILD}
Die \textit{PKGBUILD} Datei mit yaourt oder pacaur holen:
\begin{lstlisting}
  user@blackarchlinux $ yaourt -G nfsshell
  ==> Download nfsshell sources
  x LICENSE
  x PKGBUILD
  x gcc.patch
  user@blackarchlinux $ cd nfsshell/
\end{lstlisting}

\subsubsection{Aufräumen der PKGBUILD}
Aufräumen der \textit{PKGBUILD} Datei und ein bisschen Zeit sparen:
\begin{lstlisting}
  user@blackarchlinux nfsshell $ ./blackarch/scripts/prep PKGBUILD
  cleaning 'PKGBUILD'...
  expanding tabs...
  removing vim modeline...
  removing id comment...
  removing contributor and maintainer comments...
  squeezing extra blank lines...
  removing '|| return'...
  removing leading blank line...
  removing $pkgname...
  removing trailing whitespace...
\end{lstlisting}

\subsubsection{PKGBUILD anpassen}
Anpassen der \textit{PKGBUILD} Datei:
\begin{lstlisting}
  user@blackarchlinux nfsshell $ vi PKGBUILD
\end{lstlisting}

\subsubsection{Das Paket bauen}
Bau das Paket:
\begin{lstlisting}user@blackarchlinux nfsshell $ makepkg -sf
==> Making package: nfsshell 19980519-1 (Mon Dec  2 17:23:51 CET 2013)
==> Checking runtime dependencies...
==> Checking buildtime dependencies...
==> Retrieving sources...
-> Downloading nfsshell.tar.gz...
% Total    % Received % Xferd  Average Speed   Time    Time     Time
CurrentDload  Upload   Total   Spent    Left  Speed100 29213  100 29213    0
0  48150      0 --:--:-- --:--:-- --:--:-- 48206
-> Found gcc.patch
-> Found LICENSE
...
<lots of build process and compiler output here>
...
==> Leaving fakeroot environment.
==> Finished making: nfsshell 19980519-1 (Mon Dec  2 17:23:53 CET 2013)
\end{lstlisting}

\subsubsection{Installieren und testen des Pakets}
Installiere und teste das Paket:
\begin{lstlisting}
  user@blackarchlinux nfsshell $ pacman -U nfsshell-19980519-1-x86_64.pkg.tar.xz
  user@blackarchlinux nfsshell $ nfsshell # test it
\end{lstlisting}

\subsubsection{Adde, commite and pushe das Paket}
Füge das Paket hinzu, mach den Commit und Pushe.
\begin{lstlisting}user@blackarchlinux nfsshell $ cd /blackarchlinux/packages
user@blackarchlinux ~/blackarchlinux/packages $ mv ~/nfsshell .
user@blackarchlinux ~/blackarchlinux/packages $ git commit -am nfsshell && git push
\end{lstlisting}

\subsubsection{Erzeuge einen Pull Request}
Erzeuge einen Pull Request auf \href{https://github.com/}{github.com}
\begin{lstlisting}
  firefox https://github.com/<contributor>/blackarchlinux
\end{lstlisting}

\subsubsection{Füge eine upstream remote hinzu.}
Es ist eine gute Idee wenn man upstream auf einem Fork arbeitet, den eigenen Fork zu pullen und das Haupt-BlackArch repository 
als eine Remote hinzuzufügen.
\begin{lstlisting}
  user@blackarchlinux ~/blackarchlinux $ git remote -v
  origin <the url of your fork> (fetch)
  origin <the url of your fork> (push)
  user@blackarchlinux ~/blackarchlinux $ git remote add upstream https://github.com/blackarch/blackarch
  user@blackarchlinux ~/blackarchlinux $ git remote -v
  origin <the url of your fork> (fetch)
  origin <the url of your fork> (push)
  upstream https://github.com/blackarch/blackarch (fetch)
  upstream https://github.com/blackarch/blackarch (push)
\end{lstlisting}

Standardmäßig sollte git direkt auf origin pushen, aber stelle sicher das deine git konfiguration
richtig konfiguriert ist. Das sollte kein Problem sein, solange du commit rechte hast, da du ohne diese
nicht upstream pushen kannst.

Wenn du nicht committen kannst, könntest du mehr Erfolg mit
\textit{git@github.com:blackarch/blackarch.git} haben.

\subsection{Requests}
\begin{enumerate}
\item Füge keine \textbf{Maintainer} oder \textbf{Contributor} Kommentare zu
\textit{PKGBUILD} Dateien hinzu. Füge maintainer und contributor Namen zu der
AUTHORS sektion im BlackArch guide hinzu.
\item Der Konsistenz willen, bitte folge dem generellen Stil anderer
\textit{PKGBUILD} Dateien im repo und nutze doppel-space Einrückungen.
\end{enumerate}

\subsection{Generelle tips}
\href{http://wiki.archlinux.org/index.php/Namcap}{namcap} kann Pakete auf Fehler überprüfen.

%------------------%
%  Kapitel 4       %
%------------------%

\chapter{Tool Handbuch}
Coming soon...

\section{Coming Soon}
Coming soon...

%%% APPENDIX %%%
\appendix
%%%%%%%%%%%%%%%%%%%%%%%%%%%%%%%%%%%%%%%%%%%%%%%%%%%%%%%%%%%%%%%%%%%%%%%%%%%%%%%%
%                                                                              %
% BlackArch Linux plantilla de ap\'endice                                      %
%                                                                              %
%%%%%%%%%%%%%%%%%%%%%%%%%%%%%%%%%%%%%%%%%%%%%%%%%%%%%%%%%%%%%%%%%%%%%%%%%%%%%%%%

\appendix

\chapter{Appendix}

\section{FAQ's}

\begin{tabularx}{\linewidth}{lX}
		\textbf{Q:} & Ist BlackArch Linux die richtige Pentesting-Distribution für mich? \vspace{0.2cm} \\
		\textbf{A:} & BlackArch ist eine Linux-Pentesting-Distribution, die auf ArchLinux basiert. Wenn du mit ArchLinux oder Linux im Allgemeinen nicht vertraut bist, empfehlen wir dir dringend, BlackArch wegen der Lernkurve für neue Benutzer zu nicht zu verwenden. \vspace{0.7cm} \\
		\textbf{Q:} & Wo starte ich mit BlackArch? \vspace{0.2cm} \\
		\textbf{A:} & Du musst zuerst eine ISO auf der Seite \href{https://www.blackarch.org/downloads.html}{downloads} herunterladen und installieren, indem du den Anweisungen des Installationsskripts folgst. Ein Tutorial, das den Prozess Schritt für Schritt zeigt, findest du hier \href{https://www.blackarch.org/blackarch-install.html}{Blackarch-Installation}. Wenn du irgendwelche Probleme hast und Hilfe benötigst, kannst du auf unserem \href{irc://irc.freenode.net/blackarch}{FREENODE IRC channel} (\#blackarch) vorbeischauen.
		\vspace{0.5cm} \\
		\textbf{Q:} & Ist BlackArch aktuell? \vspace{0.2cm} \\
		\textbf{A:} & BlackArch wird ständig aktualisiert und bietet die neuesten Pakete auf Github. Wir veröffentlichen vierteljährlich eine neue ISO. Diese neuen Images enthalten Pakete, die auf dem neuesten Stand sind, und beinhalten in der Regel Bugfixes. Wenn du ein veraltetes Paket findest und es auf den neuesten Stand bringen möchtest, melde es bitte als Problem in unserem \href{https://github.com/BlackArch/blackarch}{Github-Repository}.
		\vspace{0.5cm} \\
		\textbf{Q:} & Wie kann ich das neueste verfügbare Update abrufen/installieren? \vspace{0.2cm} \\
		\textbf{A:} & Indem du einfach den Befehl \texttt{pacman -Syyu --needed --overwrite '*' blackarch} ausführst. Dieser Befehl benötigt root Rechte. \href{https://github.com/BlackArch/blackarch}{Github}. \vspace{0.5cm} \\
		\textbf{Q:} & Warum erhalte ich eine ungültige Schlüsselbundsignatur? \vspace{0.2cm} \\
		\textbf{A:} &  Das kann aus einer Vielzahl an Gründen passieren. Unten findest du einige Lösungsvorschläge.
		\begin{itemize}
			\item Du hast keine Internetverbindung (wie du dir vorstellen kannst ein seltenes und einfach zu lösendes Problem).
			\item Du könntest ein DNS Problem haben, welcher \url{pgp.mit.edu} nicht korrekt auflösen kann. Bitte überprüfe deine DNS Einstellungen.
			\item Möglicherweise hast du ein Netzwerkproblem, welches sich von den oben genannten unterscheidet. Das ist für uns schwer zu lösen, weil es eine Vielzahl an Dingen sein kann. Zum Beispiel: DNS-Caching.
			\item Du könntest ein Zeit/Uhr Problem haben.
			\item Es könnte daran liegen, dass etwas die Kommunikation zum \url{mit.edu} Server blockiert, bspw. die Firewall.
			\item Wenn du mit einem VPN verbunden bist, versuche es kurzzeitig auszuschalten und \texttt{strap.sh} erneut auszuführen.
			\item \url{pgp.mit.edu} könnte aus irgendwelchen Gründen nicht erreichbar sein (ja, das kann passieren) . Siehe dann Möglichkeiten 2 und 3 weiter unten für mehr Informationen.
		\end{itemize}
		Nachdem alle oben genannten Probleme ausgeschlossen wurden, versuche die unten genannten Möglichkeiten:
\end{tabularx}
\textbf{Erste Möglichkeit:}
\begin{lstlisting}
	# rm -rf /etc/pacman.d/gnupg
	# pacman-key --populate
	# pacman-key --update.
\end{lstlisting}
\vspace{0.5cm}

\textbf{Zweite Möglichkeit:}\\
Du könntest die IP Adresse des \url{pgp,mit.edu} Servers mit dem folgenden Befehl verwenden:
\begin{lstlisting}
	# curl -O https://blackarch.org/strap.sh
	# chmod +x strap.sh
	# sha1sum strap.sh
	# sed -i "s|pgp.mit.edu|18.9.60.141|g" strap.sh
	# sh strap.sh
\end{lstlisting}
Denk nur daran, dass die oben genannte IP Adresse die aktuelle IP Adresse ist und sich zu jeder Zeit ändern kann. Überprüfe das bevor du die Befehle ausführst. Außerdem muss sha1sum mit dem Hash in den \href{https://blackarch.org/downloads.html}{Downloads} übereinstimmen.
\vspace{0.5cm}

\textbf{Dritte Möglichkeit:}\\
Wenn die zweite Möglichkeit das Problem nicht gelöst hat, lade dir eine neue \texttt{strap.sh} Datei herunter und versuche dann folgende Befehle auszuführen:
\begin{lstlisting}
	# curl -O https://blackarch.org/strap.sh
	# chmod +x strap.sh
	# sha1sum strap.sh
	# sed -i "s|pgp.mit.edu|hkp://pool.sks-keyservers.net|g" strap.sh
	# sh strap.sh
\end{lstlisting}
Auch hier muss die ha1sum mit dem Hash auf der \href{https://blackarch.org/downloads.html}{Downloads} Seite übereinstimmen.

Es ist sehr wichtig die obigen Vorschläge zu befolgen UND die Archlinux Wiki Seiten zu lesen, um eine möglichst gute Hilfe zu bekommen. Wenn du trotzdem noch Probleme hast, besuche uns doch hier \href{irc://irc.freenode.net/blackarch}{\#blackarch (Freenode)}.

\begin{tabularx}{\linewidth}{lX}
	\textbf{Q:} & Wo kann ich Hilfe bei meinem Problem finden? \vspace{0.2cm} \\
	\textbf{A:} & Abhängig von dem Problem kannst du ein Issue auf Github öffnen:
	\begin{itemize}
		\item \href{https://github.com/BlackArch/blackarch-site/issues}{BlackArch Website Repository}: Probleme mit unserer Website. Beispielsweise, wenn ein Link nicht mehr funktioniert oder ein Bild nicht lädt.
		\item \href{https://github.com/BlackArch/blackarch/issues}{BlackArch Repository}: Probleme mit unseren Paketen. Beispielsweise, wenn das Paket schon lange nicht mehr aktualisiert wurde oder wenn es nicht funktioniert.
		\item \href{https://github.com/BlackArch/blackarch-installer}{BlackArch Installer Repository}: Probleme mit unserem Installer. Beispielsweise, wenn die Installation nicht funktioniert oder du nach einer erfolgreichen Installation nicht booten kannst.
	\end{itemize}
Du kannst auch in unseren \href{https://github.com/BlackArch}{anderen Repositories} schauen.
Wenn du dennoch keine Lösung findest, schaue auf unserem \href{irc://irc.freenode.net/blackarch}{IRC Kanal}  vorbei und frage um Rat. Aber habe im Hinterkopf, dass BlackArch Nutzer auf dem ganzen Planet verteilt sind (also sich in unterschiedlichen Zeitzonen befinden), deshalb sei geduldig nachdem du eine Frage gestellt hast.
	\vspace{0.7cm} \\
	\textbf{Q:} & Ich würde gerne helfen, wie kann ich das tun? \vspace{0.2cm} \\
	\textbf{A:} & BlackArch ist ein riesiges Projekt, wir fügen täglich neue Programme und Features hinzu. Wenn du uns mit was auch immer helfen willst, besuche doch unseren \href{irc://irc.freenode.net/blackarch}{IRC Kanal}. Denke nur daran geduldig zu sein und auf eine Antwort zu waren, wir sind in verschiedenen Zeitzonen.
	\vspace{0.7cm} \\
\end{tabularx}

\section{AUTOREN}
\textbf{Die folgenden Personen haben direkt zu BlackArch beigetragen:}
\begin{itemize}
\item Tyler Bennnett (tylerb@trix2voip.com)
\item fnord0 (fnord0@riseup.net)
\item nrz (nrz@nullsecurity.net)
\item Ellis Kenyo (elken.tdos@gmail.com)
\item CaledoniaProject (the.warl0ck.1989@gmail.com)
\item sudokode (sudokode@gmail.com)
\item Valentin Churavy (v.churavy@gmail.com)
\item Boy Sandy Gladies Arriezona (reno.esper@gmail.com)
\item Mathias Nyman
\item Johannes Löthberg (demizide@gmail.com)
\item Thiago da Silva Teixeira (teixeira.zeus@gmail.com)
\item Levon Kayan (noptrix@blackarch.org)
\item Pedro Freitas (psf@blackarch.org)
\item Pierre B.(pi3rrot@blackarch.org)
\item Hannes Eichblatt (heichblatt@blackarch.org)
\item Stefan Venz (ikstream@blackarch.org)
\item Sepehrdad Sh (sepehrdad@blackarch.org)
\item Eduard Toloza (edu4rdshl@blackarch.org)
\item German Vechtomov (mrsmith0x00@blackarch.org)
\item Richard Baumann (ohaz@blackarch.org)
\item Leon L. (tazmain@blackarch.org)
\item OSO SPEED (oso@blackarch.org)
\item Jeremy Marlow (i3\_arch@blackarch.org)
\item Alexandre Zanni (noraj@blackarch.org)
\item Harry P. (luserx0@blackarch.org)
\item Semtex (s7x@blackarch.org)
\item Filipe Lains (ffy00@blackarch.org)
\item Erik (deep\_m4gic@blackarch.org)
\item Alexander Rothenberg (eiswiesel@blackarch.org)
\item Dimitri Mader (anyon3@blackarch.org)
\end{itemize}

\textbf{Die Hall of Fame von ehemaligen Entwicklern:}
\begin{itemize}
	\item Evan Teitelman (teitelmanevan@gmail.com)
	\item Javier (nrz@nullsecurity.net)
	\item Ellis Kenyo (elken.tdos@gmail.com)
	\item Louis Dion-Marcil (louis.dionmarcil@gmail.com)
	\item Halit Alptekin (me@halitalptekin.com)
	\item Ano Nymous (sudokode@gmail.com)
	\item Guy Marquez (guy.marquez@yandex.com)
	\item Felipe Montes (felipe@smartdefence.cl)
	\item Abelardo Ricart (aricart@gmail.com)
\end{itemize}

\textbf{Die folgenden Personen haben direkt an ArchPwn mitgearbeitet und sind BlackArch beigetreten:}
\begin{itemize}
\item Francesco Piccinno (stack.box@gmail.com)
\item jensp (jens@jenux.homelinux.org)
\item Valentin Churavy (v.churavy@gmail.com)
\end{itemize}

\textbf{Wir haben den Build-Code von den folgenden Personen übernommen:}
\begin{itemize}
\item  3ED (krzysztof1987@gmail.com)
\item  AUR Perl (aurperl@juster.info)
\item  Aaron Griffin (aaron@archlinux.org)
\item  Abakus (java5@arcor.de)
\item  Adam Wolk (netprobe@gmail.com)
\item  Aleix Pol (aleixpol@kde.org)
\item  Aleshus (aleshusi@gmail.com)
\item  Alessandro Pazzaglia (jackdroido@gmail.com)
\item  Alessandro Sagratini (ale\_sagra@hotmail.com)
\item  Alex Cartwright (alexc223@googlemail.com)
\item  Alexander De Sousa (archaur.xandy21@spamgourmet.com)
\item  Alexander Rødseth (rodseth@gmail.com)
\item  Allan McRae (allan@archlinux.org)
\item  AmaN (gabroo.punjab.da@gmail.com)
\item  Andre Klitzing (aklitzing@online.de)
\item  Andrea Scarpino (andrea@archlinux.org)
\item  Andreas Schönfelder (passtschu@freenet.de)
\item  Andrej Gelenberg (andrej.gelenberg@udo.edu)
\item  Angel Velasquez (angvp@archlinux.org)
\item  Antoine Lubineau (antoine@lubignon.info)
\item  Anton Bazhenov (anton.bazhenov@gmail.com)
\item  Arkham (arkham@archlinux.us)
\item  Arthur Danskin (arthurdanskin@gmail.com)
\item  Balda (balda@balda.ch)
\item  Balló György (ballogyor+arch@gmail.com)
\item  Bartek Piotrowski (barthalion@gmail.com)
\item  Bartosz Feński (fenio@debian.org)
\item  Bartłomiej Piotrowski (nospam@bpiotrowski.pl)
\item  Bogdan Szczurek (thebodzio@gmail.com)
\item  Brad Fanella (bradfanella@archlinux.us)
\item  Brian Bidulock (bidulock@openss7.org)
\item  C Anthony Risinger (anthony@xtfx.me)
\item  CRT (crt.011@gmail.com)
\item  Can Celasun (dcelasun@gmail.com)
\item  Chaniyth (chaniyth@yahoo.com)
\item  Chris Brannon (cmbrannon79@gmail.com)
\item  Chris Giles (Chris.G.27@gmail.com) \& daschu117
\item  Christoph Siegenthaler (csi@gmx.ch)
\item  Christoph Zeiler (archNOSPAM@moonblade.org)
\item  Clément DEMOULINS (clement@archivel.fr)
\item  Corrado Primier (bardo@aur.archlinux.org)
\item  Daenyth (Daenyth+Arch@gmail.com)
\item  Dale Blount (dale@archlinux.org)
\item  Damir Perisa (damir.perisa@bluewin.ch)
\item  Dan Fuhry (dan@fuhry.us)
\item  Dan Serban (dserban01@yahoo.com)
\item  Daniel A. Campoverde Carrión
\item  Daniel Golle
\item  Daniel Griffiths (ghost1227@archlinux.us)
\item  Daniel J Griffiths (ghost1227@archlinux.us)
\item  Daniel Micay (danielmicay@gmail.com)
\item  Dave Reisner (dreisner@archlinux.org)
\item  Dawid Wrobel (cromo@klej.net)
\item  Devaev Maxim (mdevaev@gmail.com)
\item  Devin Cofer (ranguvar@archlinux.us)
\item  DigitalPathogen (aur@InfoSecResearchLabs.co.uk)
\item  DigitalPathogen (aur@digitalpathogen.co.uk)
\item  Dmitry A. Ilyashevich (dmitry.ilyashevich@gmail.com)
\item  Dominik Heidler (dheidler@gmail.com)
\item  DrZaius (lou@fakeoutdoorsman.com)
\item  Ebubekir KARUL (ebubekirkarul@yandex.com)
\item  Eduard "bekks" Warkentin (eduard.warkentin@gmail.com)
\item  Elmo Todurov (todurov@gmail.com)
\item  Emmanuel Gil Peyrot (linkmauve@linkmauve.fr)
\item  Eric Belanger (eric@archlinux.org)
\item  Ermak (ermak@email.it)
\item  Evangelos Foutras (evangelos@foutrelis.com)
\item  Fabian Melters (melters@gmail.com)
\item  Fabiano Furtado (fusca14@gmail.com)
\item  Federico Quagliata (ntp@quaqo.org)
\item  Firmicus (francois.archlinux@org)
\item  Florian Pritz (bluewind@jabber.ccc.de)
\item  Florian Pritz (flo@xinu.at)
\item  Francesco Piccinno (stack.box@gmail.com)
\item  François Charette (francois@archlinux.org)
\item  Gaetan Bisson (bisson@archlinux.org)
\item  Geoffroy Carrier (geoffroy.carrier@koon.fr)
\item  Georg Grabler (STiAT)
\item  George Hilliard (gh403@msstate.edu)
\item  Gerardo Exequiel Pozzi (vmlinuz386@yahoo.com.ar)
\item  Gilles CHAUVIN (gcnweb@gmail.com)
\item  Giovanni Scafora (giovanni@archlinux.org)
\item  Gordin (9ordin@gmail.com)
\item  Guillaume ALAUX (guillaume@archlinux.org)
\item  Guillermo Ramos (0xwille@gmail.com)
\item  Gustavo Alvarez (sl1pkn07@gmail.com)
\item  Hugo Doria (hugo@archlinux.org)
\item  Hyacinthe Cartiaux (hyacinthe.cartiaux@free.fr)
\item  James Fryman (jfryman@gmail.com)
\item  Jan "heftig" Steffens (jan.steffens@gmail.com)
\item  Jan de Groot (jgc@archlinux.org)
\item  Jaroslav Lichtblau (dragonlord@aur.archlinux.org)
\item  Jaroslaw Swierczynski (swiergot@aur.archlinux.org)
\item  Jason Chu (jason@archlinux.org)
\item  Jason R Begley (jayray@digitalgoat.com)
\item  Jason Rodriguez
\item  Jason St. John (jstjohn@purdue.edu)
\item  Jawmare (victor2008@gmail.com)
\item  Jeff Mickey (jeff@archlinux.org)
\item  Jens Pranaitis (jens@chaox.net)
\item  Jens Pranaitis (jens@jenux.homelinux.org)
\item  Jinx (jinxware@gmail.com)
\item  John D Jones III (jnbek1972@gmail.com)
\item  John Proctor (jproctor@prium.net)
\item  Jon Bergli Heier (snakebite@jvnv.net)
\item  Jonas Heinrich
\item  Jonathan Steel (jsteel@aur.archlinux.org)
\item  Joris Steyn (jorissteyn@gmail.com)
\item  Josh VanderLinden (arch@cloudlery.com)
\item  Jozef Riha (jose1711@gmail.com)
\item  Judd Vinet (jvinet@zeroflux.org)
\item  Juergen Hoetzel (jason@archlinux.org)
\item  Juergen Hoetzel (juergen@archlinux.org)
\item  Justin Davis (jrcd83@gmail.com)
\item  Kaiting Chen (kaitocracy@gmail.com)
\item  Kaos
\item  Kevin Piche (kevin@archlinux.org)
\item  Kory Woods (kory@virlo.net)
\item  Kyle Keen (keenerd@gmail.com)
\item  Larry Hajali (larryhaja@gmail.com)
\item  LeCrayonVert
\item  Le\_suisse (lesuisse.dev+aur@gmail.com)
\item  Lekensteyn (lekensteyn@gmail.com)
\item  Limao Luo (luolimao+AUR@gmail.com)
\item  Lucien Immink
\item  Lukas Fleischer (archlinux@cryptocrack.de)
\item  Manolis Tzanidakis
\item  Marcin "avalan" Falkiewicz (avalatron@gmail.com)
\item  Mariano Verdu (verdumariano@gmail.com)
\item  Marti Raudsepp (marti@juffo.org)
\item  MatToufoutu (mattoufootu@gmail.com)
\item  Matthew Sharpe (matt.sharpe@gmail.com)
\item  Mauro Andreolini (mauro.andreolini@unimore.it)
\item  Max Pray a.k.a. Synthead (synthead@gmail.com)
\item  Max Roder (maxroder@web.de)
\item  Maxwell Pray a.k.a. Synthead (synthead@gmail.com)
\item  Maxwell Pray a.k.a. Synthead (synthead1@gmail.com)
\item  Mech (tiago.bmp@gmail.com)
\item  Michael Düll (mail@akurei.me)
\item  Michael P (ptchinster@archlinux.us)
\item  Michal Krenek (mikos@sg1.cz)
\item  Michal Zalewski (lcamtuf@coredump.cx)
\item  Miguel Paolino (mpaolino@gmail.com)
\item  Miguel Revilla (yo@miguelrevilla.com)
\item  Mike Roberts (noodlesgc@gmail.com)
\item  Mike Sampson (mike@sambodata.com)
\item  Nassim Kacha (nassim.kacha@gmail.com)
\item  Nicolas Pouillard (nicolas.pouillard@gmail.com)
\item  Nicolas Pouillard https://nicolaspouillard.fr
\item  Niklas Schmuecker
\item  Oleander Reis (oleander@oleander.cc)
\item  Olivier Le Moal (mail@olivierlemoal.fr)
\item  Olivier Médoc "oliv" (o\_medoc@yahoo.fr)
\item  Pascal E. (archlinux@hardfalcon.net)
\item  Patrick Leslie Polzer (leslie.polzer@gmx.net)
\item  Paul Mattal (paul@archlinux.org)
\item  Paul Mattal (pjmattal@elys.com)
\item  Pengyu CHEN (cpy.prefers.you@gmail.com)
\item  Peter Wu (lekensteyn@gmail.com)
\item  Philipp 'TamCore' B. (philipp@tamcore.eu)
\item  Pierre Schmitz (pierre@archlinux.de)
\item  Pranay Kanwar (pranay.kanwar@gmail.com)
\item  Pranay Kanwar (warl0ck@metaeye.org)
\item  PyroPeter (googlemail@com.abi1789)
\item  PyroPeter (googlemail.com@abi1789)
\item  Ray Rashif (schiv@archlinux.org)
\item  Remi Gacogne
\item  Renan Fernandes (renan@kauamanga.com)
\item  Richard Murri (admin@richardmurri.com)
\item  Roberto Alsina (ralsina@kde.org)
\item  Robson Peixoto (robsonpeixoto@gmail.com)
\item  Roel Blaauwgeers (roel@ttys0.nl)
\item  Rorschach (r0rschach@lavabit.com)
\item  Ruben Schuller (shiml@orgizm.net)
\item  Rudy Matela (rudy@matela.com)
\item  Ryon Sherman (ryon.sherman@gmail.com)
\item  Sabart Otto \item  Seberm (seberm@gmail.com)
\item  SakalisC (chrissakalis@gmail.com)
\item  Sam Stuewe (halosghost@archlinux.info)
\item  SanskritFritz (SanskritFritz@gmail.com)
\item  Sarah Hay (sarahhay@mb.sympatico)
\item  Sebastian Benvenuti (sebastianbenvenuti@gmail.com)
\item  Sebastian Nowicki (sebnow@gmail.com)
\item  Sebastien Duquette (ekse.0x@gmail.com)
\item  Sebastien LEDUC (sebastien@sleduc.fr)
\item  Sebastien Leduc (sebastien@sleduc.fr)
\item  Sergej Pupykin (pupykin.s+arch@gmail.com)
\item  Sergio Rubio (rubiojr@biondofu.net)
\item  Sheng Yu (magicfish1990@gmail.com)
\item  Simon Busch (morphis@gravedo.de)
\item  Simon Legner (Simon.Legner@gmail.com)
\item  Sirat18 (aur@sirat18.de)
\item  SpepS (dreamspepser@yahoo.it)
\item  Spider.007 (archPackage@spider007.net)
\item  Stefan Seering
\item  Stephane Travostino (stephane.travostino@gmail.com)
\item  Stéphane Gaudreault (stephane@archlinux.org)
\item  Sven Kauber (celeon@gmail.com)
\item  Sven Schulz (omee@archlinux.de)
\item  Sébastien Duquette (ekse.0x@gmail.com)
\item  Sébastien Luttringer (seblu@archlinux.org)
\item  TDY (tdy@gmx.com)
\item  Teemu Rytilahti (tpr@iki.fi)
\item  Testuser\_01
\item  Thanx (thanxm@gmail.com)
\item  Thayer Williams (thayer@archlinux.org)
\item  Thomas S Hatch (thatch45@gmail.com)
\item  Thorsten Töpper
\item  Tilmann Becker (tilmann.becker@freenet.de)
\item  Timothy Redaelli (timothy.redaelli@gmail.com)
\item  Timothée Ravier (tim@siosm.fr)
\item  Tino Reichardt
\item  Tobias Kieslich (tobias@justdreams.de)
\item  Tobias Powalowski (tpowa@archlinux.org)
\item  Tom K (tomk@runbox.com)
\item  Tom Newsom (Jeepster@gmx.co.uk)
\item  Tomas Lindquist Olsen (tomas@famolsen.dk)
\item  Travis Willard (travisw@wmpub.ca)
\item  Valentin Churavy (v.churavy@gmail.com)
\item  ViNS (gladiator@fastwebnet.it)
\item  Vlatko Kosturjak (kost@linux.hr)
\item  Wes Brown (wesbrown18@gmail.com)
\item  William Rea (sillywilly@gmail.com)
\item  Xavier Devlamynck (magicrhesus@ouranos.be)
\item  Xiao\item Long Chen (chenxiaolong@cxl.epac.to)
\item  aeolist (aeolist@hotmail.com)
\item  ality@pbrane.org
\item  astaroth (astaroth\_@web.de)
\item  bender02@archlinux.us
\item  billycongo (billycongo@gmail.com)
\item  bslackr (brendan@vastactive.com)
\item  cbreaker (cbreaker@tlen.pl)
\item  chimeracoder (dev@chimeracoder.net)
\item  damir (damir@archlinux.org)
\item  danitool
\item  darkapex (me@jailuthra.in)
\item  daronin
\item  dkaylor (dpkaylor@gmail.com)
\item  dobo (dobo90\_at\_gmail@com)
\item  dorphell (dorphell@archlinux.org)
\item  evr (evanroman@at.gmail)
\item  fnord0 (fnord0@riseup.net)
\item  fxbru (frxbru@gmail)
\item  hcar
\item  icarus (icarus.roaming@gmail.com)
\item  iceman (icemanf@gmail.com)
\item  kastor (kastor@fobos.org)
\item  kfgz (kfgz@interia.pl)
\item  linuxSEAT (linuxSEAT@gmail.com)
\item  m4xm4n (max@maxfierke.com)
\item  mar77i (mysatyre@gmail.com)
\item  marc0s (marc0s@fsfe.org)
\item  mickael9 (mickael9@gmail.com)
\item  nblock (nblock@archlinux.us)
\item  nofxx (x@nofxx.com)
\item  onny (onny@project
\item  pootzko (pootzko@gmail.com)
\item  revel (revel@muub.net)
\item  rich\_o (rich\_o@lavabit.com)
\item  s1gma (s1gma@mindslicer.com)
\item  sandman (r.coded@gmail.com)
\item  sebikul (sebikul@gmail.com)
\item  sh0 (mee@sh0.org)
\item  shild (sxp@bk.ru)
\item  simo (simo@archlinux.org)
\item  snuo
\item  sudokode (sudokode@gmail.com)
\item  tobias (tobias@archlinux.org)
\item  trashstar (trash@ps3zone.org)
\item  unexist (unexist@subforge.org)
\item  untitled (rnd0x00@gmail.com)
\item  virtuemood (virtue@archlinux.us)
\item  wido (widomaker2k7@gmail.com)
\item  wodim (neikokz@gmail.com)
\item  yannsen (ynnsen@gmail.com)
\end{itemize}


\end{document}

%%% EOF %%%
