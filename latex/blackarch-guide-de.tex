%%%%%%%%%%%%%%%%%%%%%%%%%%%%%%%%%%%%%%%%%%%%%%%%%%%%%%%%%%%%%%%%%%%%%%%%%%%%%%%%
%                                                                              %
% BlackArch Linux Handbuch                                                     %
%                                                                              %
%%%%%%%%%%%%%%%%%%%%%%%%%%%%%%%%%%%%%%%%%%%%%%%%%%%%%%%%%%%%%%%%%%%%%%%%%%%%%%%%

\documentclass[a4paper, oneside, 11pt]{book}

%%% INCLUDES %%%
\renewcommand{\familydefault}{\sfdefault}

\usepackage{array}
\usepackage{color}
\usepackage{enumerate}
\usepackage{fancyhdr}
\usepackage{fancyvrb}
\usepackage{geometry}
\usepackage{graphicx}
\usepackage{html}
\usepackage{hyperref}
\usepackage{ifpdf}
\usepackage{listings}
\usepackage{pstricks}
\usepackage{supertabular}
\usepackage{tocloft}
\usepackage[utf8]{inputenc}

%%% LAYOUT %%%
\setlength{\parindent}{0em}
\setlength{\parskip}{1.5ex plus0.5ex minus0.5ex}
\geometry{left=2.5cm, textwidth=16cm, top=3cm, textheight=25cm, bottom=3cm}
\widowpenalty=2000
\clubpenalty=1000
\frenchspacing
\sloppy
\pagecolor[HTML]{FFFFFF}
\color[HTML]{333333}
\setcounter{tocdepth}{10}
\setcounter{secnumdepth}{10}

\hypersetup{
  pdftitle={BlackArch Linux, Das BlackArch Linux Handbuch},
  pdfsubject={BlackArch Linux, Der BlackArch Linux Handbuch},
  pdfauthor={BlackArch Linux, BlackArch Linux},
  pdfkeywords={BlackArch Linux, Penetration Testing, Security, Hacking, Linux},
  pdfcenterwindow=true,
  colorlinks=true,
  breaklinks=true,
  linkcolor=red,
  menucolor=red,
  urlcolor=red
}

% syntax highlighting
% all options should be set here document wide
\lstset{
backgroundcolor=\color[HTML]{EEEEEE},
frame=single,
basicstyle=\footnotesize\ttfamily,
float,
deletekeywords={return},
otherkeywords={mkdir, curl, sudo, sha1sum, grep, cut, sort, wget, makepkg,
pacman, blackman},
keywordstyle=\color{orange},
commentstyle=\color{blue},
stringstyle=\color{red},
language=bash,
showspaces=false,
showtabs=false,
tabsize=2
}

%%% HEADER / FOOTER %%%
\setlength{\headheight}{33pt}
\setlength{\headsep}{33pt}
\lhead{{\includegraphics[width=1cm,height=1cm]{images/logo.png}}}
\rhead{Das BlackArch Linux Handbuch}

%%% CUSTOM MACROS %%%
% for html links
\ifpdf\else
\def\href#1#2{\htmladdnormallink{#2}{#1}}
\fi

%------------------%
%  TITLE PAGE      %
%------------------%
\begin{document}
\pagestyle{empty}
\begin{center}
\begin{figure}[htbp]
\centering
\vspace{0.5cm}
\includegraphics[width=8cm]{images/logo.png}
\label{fig:logo}
\end{figure}
\vspace{0.5cm}
\Huge{\textbf{The BlackArch Linux Guide}}\\
\vspace{1cm}
\Large{\color{red}https://www.blackarch.org/}\\
\vspace{0.5cm}
\end{center}
\newpage
\tableofcontents
\newpage
\pagestyle{fancy}

%------------------%
%  Kapitel 1       %
%------------------%

\chapter{Einführung}

\section{Übersicht}
Der BlackArch Linux guide ist in verschiedene Teile aufgeteilt:
\begin{itemize}
\item Einführung - Gibt einen breiten Überblick, eine Einführung, und weitere Hilfreiche Projektinformationen
\item Nutzerhandbuch - Alles was ein typischer Nutzer wissen muss um BlackArch zu benutzen
\item Entwicklerhandbuch - Wie kann man zu BlackArch beitragen und entwickeln
\item Tool Guide - Tiefgehende Details zu Tools und Beispiele zur Benutzung (WIP)
\end{itemize}

\section{Was ist BlackArch Linux?}
BlackArch ist eine vollwertige Linux Distribution für Penetration Tester und Security Researcher.
Es basiert auf \href{https://www.archlinux.org/}{ArchLinux} und Nutzer können BlackArch Komponenten einzeln oder in Gruppen installieren.

Das Toolset wird mittel eines inoffiziellem Benutzer Repositories verteilt, so dass man BlackArch auf einer existierenden Arch Linux Installation installieren kann. Pakete können individuell oder mittels einer Kategorie installiert werden.
\href{https://wiki.archlinux.org/index.php/Unofficial\_User\_Repositories}
{unofficial user repository}

Das konstant wachsende Repository beinhaltet aktuell über \href{https://www.blackarch.org/tools.html}{1300} tools.
Alle tools werden intensiv getestet bevor sie zur Codebasis hinzugefügt werden, um die Qualität des Repositories zu gewährleisten.
% should quickly describe the testing methods/code review procedures etc

\section{Geschichte von BlackArch Linux}
Coming soon...

\section{Supported platforms}
Coming soon...

\section{Get involved}
Man kann über folgende Wege mit dem BlackArch Team in Kontakt treten:
Website: \url{https://www.blackarch.org/}

Mail: \href{mailto:team@blackarch.org}{team@blackarch.org}

IRC: \url{irc://irc.freenode.net/blackarch}

Twitter: \url{https://twitter.com/blackarchlinux}

Github: \url{https://github.com/Blackarch/}

%------------------%
%  Kapitel 2       %
%------------------%


\chapter{Benutzerhandbuch}

\section{Installation}
Der folgende Abschnitt zeigt, wie man das BlackArch Repository einrichtet und Pakete installiert. BlackArch unterstützt sowohl die Installation von Binärpaketen als auch die Installation über selbstkompilierten Quellcode. 

BlackArch ist kompatibel mit regulären Arch installationen. Es verhält sich wie ein inoffizelles Nutzerreporisotrz.
Wenn stattdessen ein ISO benötigt wird, siehe den Abschnitt
\href{https://www.blackarch.org/downloads.html#iso}{Live ISO}.

\subsection{Installation basierend auf einer vorhandenen ArchLinux Installation}
Führe \href{https://blackarch.org/strap.sh}{strap.sh} als root aus und folge den Anweisungen. 

Hier ein Beispiel.
\begin{lstlisting}
   curl -O https://blackarch.org/strap.sh
   sha1sum strap.sh # should match: 86eb4efb68918dbfdd1e22862a48fda20a8145ff
   sudo ./strap.sh
\end{lstlisting}

Jetzt lade eine frische Kopie der Master Paket Liste und synchronisiere die Pakete:
\begin{lstlisting}
  sudo pacman -Syyu
\end{lstlisting}


\subsection{Paketinstallattion}
Jetzt können Tools aus dem BlackArch Repository installiert werden.
\begin{enumerate}
\item Um alle verfügbaren Tools aufzulisten:
\begin{lstlisting}
  pacman -Sgg | grep blackarch | cut -d' ' -f2 | sort -u
\end{lstlisting}

\item To install all of the tools, run
\begin{lstlisting}
  pacman -S blackarch
\end{lstlisting}

\item Um eine Toolkategorie zu installieren:
\begin{lstlisting}
  pacman -S blackarch-<category>
\end{lstlisting}

\item Um die BlackArch Kategorien zu sehen:
\begin{lstlisting}
  pacman -Sg | grep blackarch
\end{lstlisting}

\end{enumerate}

\subsection{Paketinstallation auf Quellcodebasis}
Alternativ können BlackArch-Pakete auch aus Quellcode gebaut werden. Die PKGBUILDS können auf \href{https://github.com/BlackArch/blackarch/tree/master/packages}{github} gefunden werden. Um das gesamte Repository zu bauene, kann das \href{https://github.com/BlackArch/blackman}{Blackman} tool genutzt werden.
\begin{itemize}
\item Als erste muss Blackman installiert werden. Wenn das BlackArch Reposity auf ihrer Maschine eingerichtet ist, kann Blackman installiert werden:
\begin{lstlisting}
  pacman -S blackman
\end{lstlisting}

\item Blackman kann von Quellcode gebaut und installiert werden: 
\begin{lstlisting}
  mkdir blackman
  cd blackman
  wget https://raw2.github.com/BlackArch/blackarch/master/packages/blackman/PKGBUILD
  # Sicherstellen dass die PKGBUILD nicht bösartig verändert worden sind.
  makepkg -s
\end{lstlisting}

\item Blackman kann auch aus dem AUR installiert werden:
\begin{lstlisting}
  <Verwendeter AUR Helfer> -S blackman
\end{lstlisting}

\end{itemize}

\subsection{Grundlegende Verwendung von Blackman} Blackman ist sehr einfach zu nutzen, auch wenn sich die flags von dem unterscheiden, was man typischerweise von pacman erwarten wuerde. 
Die Grundlegende Benutzung wird im folgenden gezeigt.
\begin{itemize}
\item Herunterladen, kompilieren and installieren von Paketen:
\begin{lstlisting}
  sudo blackman -i package
\end{lstlisting}

\item Herunterladen, kompilieren und installieren einer ganzen Kategorie:
\begin{lstlisting}
  sudo blackman -g group
\end{lstlisting}

\item Herunterladen, kompilieren und installieren aller BlackArch Tools:
\begin{lstlisting}
  sudo blackman -a
\end{lstlisting}

\item Auflistung aller BlackArch Kategorien:
\begin{lstlisting}
  blackman -l
\end{lstlisting}

\item Auflistung der Tools einer Kategorie:
\begin{lstlisting}
  blackman -p category
\end{lstlisting}

\end{itemize}

\subsection{Installing from live-, netinstall- ISO or ArchLinux}
BlackArch Linux kann von unseren live- oder netinstall-ISOs intalliert werden. \\Siehe
\url{https://www.blackarch.org/download.html#iso}. Die folgenden Schritte sind noetig wenn die ISO gebootet ist.
\begin{itemize}
\item Installieren des blackarch-installer Pakets:
\begin{lstlisting}
  sudo pacman -S blackarch-installer
\end{lstlisting}

\item Run
\begin{lstlisting}
  sudo blackarch-install
\end{lstlisting}

\end{itemize}

%------------------%
%  Kapitel 3       %
%------------------%

\chapter{Entwicklerhandbuch}

\section{Arch's Build System und Repositories}

PKGBUILD Dateien sind build skripte. Jedes beschreibt makepkg(1) wie ein Paket gebaut wird. PKGBUILD Dateien werden in Bash geschrieben.

Fuer weitere Informationen, lese (oder ueberfliege) folgende Seiten:
\begin{itemize}
\item \href{https://wiki.archlinux.org/index.php/Creating_Packages}{Arch Wiki: Erzeuge Packages}
\item \href{https://wiki.archlinux.org/index.php/Makepkg}{Arch Wiki: makepkg}
\item \href{https://wiki.archlinux.org/index.php/PKGBUILD}{Arch Wiki: PKGBUILD}
\item \href{https://wiki.archlinux.org/index.php/Arch_Packaging_Standards}{Arch Wiki: Arch Packaging Standards}
\end{itemize}

\section{Blackarch PKGBUILD Standards}
Der Einfachkeit halber sind unsere PKGBUILDs dem des AUR sehr aehnlich, die kleinen Unterschiede werden im weiteren Text beschrieben. 
Jedes Paket muss mindestens zu blackarch gehoeren, es wird aber auch viele beziehungen ueber mehrere pakete die zu mehreren Gruppen gehoeren geben.

\subsection{Gruppen}
Um es Nutzern zu ermoeglichen eine ganze Reihe von Paketen schnell und einfach zu installieren,
wurden Pakete in Gruppen aufgeteilt.
Gruppen ermoeglichen es den benutzern mit einem einfachen "pacman -S <group name>" eine Menge von Paketen zu bekommen.

\subsubsection{blackarch}
Die blackarch gruppe ist die basis-Gruppe zu der alle Pakete gehoeren muessen. 
Das ermoeglicht es den Nutzern einfach alle Pakete zu installieren.

Was sollte hier drin sein: Alles.

\subsubsection{blackarch-anti-forensic}
Pakete die dazu benutzt werden, forensische Aktivitaeten zu umgehen. Das beinhaltet Verschluesselung, Steganographie 
und alles was es ermoeglicht Datei/Ordner Attribute zu manipulieren.
Das alles beinhaltet tools die allgemein veraenderungen an einem System durchfuehren mit dem Zweck,
Information zu verstecken. 

Beispiele: luks, TrueCrypt, Timestomp, dd, ropeadope, secure-delete

\subsubsection{blackarch-automation}
Pakete zur tool oder workflow Automatisierung.

Beispiele: blueranger, tiger, wiffy

\subsubsection{blackarch-backdoor}
Pakete zur Ausnutzung oder Oeffnung von backdoors auf bereits verwundbaren
Systemen.

Beispiele: backdoor-factory, rrs, weevely

\subsubsection{blackarch-binary}
Pakete die auf irgendwelchen Binaerdateien arbeiten.

Beispiele: binwally, packerid

\subsubsection{blackarch-bluetooth}
Pakete die alles exploiten was mit dem Bluetooth Standard(802.15.1) zu tun hat.

Beispiele: ubertooth, tbear, redfang

\subsubsection{blackarch-code-audit}
Pakete die bestehenden Code analysieren um Sicherheitsluecken zu finden.

Beispiele: flawfinder, pscan

\subsubsection{blackarch-cracker}
Pakete die zum cracken von kryptographischen Funktionen, zum Beispiel Hashes.

Beispiele: hashcat, john, crunch

\subsubsection{blackarch-crypto}
Pakete die mit kryptographie arbeiten, mit der Ausnahme vom cracken.

Beispiele: ciphertest, xortool, sbd

\subsubsection{blackarch-database}
Pakete die Datenbank-Exploits auf jedem Level betreffen.

Beispiele: metacoretex, blindsql

\subsubsection{blackarch-debugger}
Pakete die es dem Nutzer erlauben in Echtzeit zu sehen, was ein bestimmtes Programm tut.

Beispiele: radare2, shellnoob

\subsubsection{blackarch-decompiler}
Pakete die versuchen kompilierte Programm in Quellcode zu konvertieren.

Beispiele: flasm, jd-gui

\subsubsection{blackarch-defensive}
Pakete die Versuchen den Nutzer vor Malware und Attacken anderer Nutzer zu schuetzen.

Beispiele: arpon, chkrootkit, sniffjoke

\subsubsection{blackarch-disassembler}
Aehnlich zu blackarch-decompiler> Hier gibt es vermutlich einige Programme die
in beide Kategorien fallen, mit dem Unterschied das diese Pakete Assembler ausgeben
statt den puren Quellcode.

Beispiele: inguma, radare2

\subsubsection{blackarch-dos}
Pakete die DoS (Denial of Service) Angriffe nutzen.

Beispiele: 42zip, nkiller2

\subsubsection{blackarch-drone}
Pakete die zur Verwaltung von echten Drohnen verwendet werden.

Beispiele: meshdeck, skyjack

\subsubsection{blackarch-exploitation}
Pakete die exploits anderer Programme oder Dienste nutzen.

Beispiele: armitage, metasploit, zarp

\subsubsection{blackarch-fingerprint}
Pakete die Fingerabdruecke biometrischer Systeme exploiten.

Beispiele: dns-map, p0f, httprint

\subsubsection{blackarch-firmware}
Pakete die Schwachstellen in Firmware ausnutzen.

Beispiele: Noch keine, asap hinzufuegen.

\subsubsection{blackarch-forensic}
Pakete die benutzt werden um Daten auf physischen Festplatten oder Speicher zu finden.

Beispiele: aesfix, nfex, wyd

\subsubsection{blackarch-fuzzer}
Pakete die die Fuzzy Testprinzipien nutzen, zum Beispiel
zufaelligen Input "reinzuwerfen" und zu sehen was passiert.

Beispiele: msf, mdk3, wfuzz

\subsubsection{blackarch-hardware}
Pakete die alles verwalten oder ausnutzen was mit
physischer Hardware zu tun hat.

Beispiele: arduino, smali

\subsubsection{blackarch-honeypot}
Pakete die als "honeypots" fungieren, zum Beispiel Programme
die sich als verwundbare Dienste ausgeben und hacker in eine Falle locken
sollen.

Beispiele: artillery, bluepot, wifi-honey

\subsubsection{blackarch-keylogger}
Pakete die Tastendruecke auf anderen Systemen aufnehmen und speichern.

Beispiele: None yet, amend asap.

\subsubsection{blackarch-malware}
Pakete die zu Malware zaehlen oder Malware erkennung.

Beispiele: malwaredetect, peepdf, yara

\subsubsection{blackarch-misc}
Pakete die nicht unbedingt in eine spezielle Kategorie passen.

Beispiele: oh-my-zsh-git, winexe, stompy

\subsubsection{blackarch-mobile}
Pakete die mobile Plattformen manipulieren.

Beispiele: android-sdk-platform-tools, android-udev-rules

\subsubsection{blackarch-networking}
Pakete die IP Netzerke betreffen.

Beispiele: TODO

\subsubsection{blackarch-nfc}
Pakete die NFC (near-field communication) nutzen.

Beispiele: nfcutils

\subsubsection{blackarch-packer}
Pakete die Packer bedienen oder beinhalten.

\textit{Packer sind Programme die malware in anderen Executables einbetten. }

Beispiele: packerid

\subsubsection{blackarch-proxy}
Pakete die als Proxy fungieren, also zum Beispiel Netzwerkverkehr durch einen
anderen Knoten im Internet umleiten.

Beispiele: burpsuite, ratproxy, sslnuke

\subsubsection{blackarch-recon}
Pakete die aktiv verwundbare exploits suchen.
Eine Obergruppe fuer aehnliche Pakete.

Beispiele: canri, dnsrecon, netmask

\subsubsection{blackarch-reversing}
Uebergruppe fuer jegliche decompiler, 
disassembler oder aehnliche Programme.

Beispiele: capstone, radare2, zerowine

\subsubsection{blackarch-scanner}
Pakete die ausgewaehlte Systeme auf Schwachstellen scannen.

Beispiele: scanssh, tiger, zmap

\subsubsection{blackarch-sniffer}
Pakete die mit dem analysieren von Netzwerkverkehr
zu tun haben.

Beispiele: hexinject, pytactle, xspy

\subsubsection{blackarch-social}
Pakete die hauptsaechlich soziale netzwerke angreifen.

Beispiele: jigsaw, websploit

\subsubsection{blackarch-spoof}
Pakete die versuchen den angreifer zu spoofen, 
sodass der Angreifer nicht als Angreifer fure das Opfer zu 
erkennen ist.

Beispiele: arpoison, lans, netcommander

\subsubsection{blackarch-threat-model}
Pakete die zum Reporten/Aufnehmen des Threat-Models
in einem speziellen Szenario benutzt werden.

Beispiele: magictree

\subsubsection{blackarch-tunnel}
Pakete die dazu genutzt werden, Netzwerkverkehr zu einem
gegebenen Netzwerk zu tunneln.

Beispiele: ctunnel, iodine, ptunnel

\subsubsection{blackarch-unpacker}
Pakete die dazu genutzt werden, vorgepackten Schadcode von 
einer executable auszupacken.

Beispiele: js-beautify

\subsubsection{blackarch-voip}
Pakete die auf VOIP Programmen und Protokollen arbeiten.

Beispiele: iaxflood, rtp-flood, teardown

\subsubsection{blackarch-webapp}
Pakete die auf internet-zugewandten Anwendungen arbeiten.

Beispiele: metoscan, whatweb, zaproxy

\subsubsection{blackarch-windows}
Diese Gruppe ist fuer native Windows Pakete die via wine laufen.

Beispiele: 3proxy-win32, pwdump, winexe

\subsubsection{blackarch-wireless}
Pakete die auf drahtlosen Netzwerken arbeiten.

Beispiele: airpwn, mdk3, wiffy

\section{Repository structure}
Das primaere git repo fuer BlackArch befindet sich hier:
\href{https://github.com/BlackArch/blackarch}{https://github.com/BlackArch/blackarch}.
Es gibt ausserdem verschiedene Sekundare Repositories hier:
\href{https://github.com/BlackArch}{https://github.com/BlackArch}.

Innerhalb des Hauptrepos gibt es drei wichtige Verzeichnisse:

\begin{itemize}
\item docs - Dokumentation.
\item packages - PKGBUILD Dateien.
\item scripts - Nuetzliche kleine Skripte.
\end{itemize}

\subsection{Scripts}
Hier eine Referenz fuer Skripte im \verb|scripts/| Verzeichnis:

\begin{itemize}
\item baaur - Coming soon: Wird Pakete in das AUR hochladen.
\item babuild - Baut ein Paket.
\item bachroot - Managen eines chroot zum testen.
\item baclean - Raeumt alte .pkg.tar.xz Dateien aus dem Paket Repository.
\item baconflict - Wird bald \verb|scripts/conflicts| ersetzen.
\item bad-files - Findet schlechte Dateien in gebauten Paketen.
\item balock - Anlegen oder loesen des Repository locks.
\item banotify - IRC benachrichtigen ueber Paket pushes.
\item barelease - Veroeffentlicht Pakete in das Repository.
\item baright - Gibt die BlackArch Copyright Informationen aus.
\item basign - Signiert Packete.
\item basign-key - Signiert einen Schluessel.
\item blackman - Verhaelt sich aehnlich wie pacman, baut aber aus git. 
	(Nicht zu verwechseln mit nrz's Blackman)
\item check-groups - Ueberprueft groups.
\item checkpkgs - Ueberprueft Pakete auf Fehler.
\item conflicts - Sucht nach Dateikonflikten.
\item dbmod - Modifiziert eine Paketdatenbank.
\item depth-list - Erzeugt eine Liste sortiert nach Abhaengigkeitspfad.
\item deptree - Erzeugt einen Abhaengigkeitsbaum, der nur blackarch Pakete enthaelt.
\item get-blackarch-deps - Liefert eine List von blackarch Abhaengigkeiten fuer ein Paket.
\item get-official - Liefert offizielle Pakete zum Release.
\item list-loose-packages - Listet Pakete die weder in Gruppen noch Abhaengigkeiten anderer Pakete sind.
\item list-needed - Liste fehlender Abhaengigkeiten.
\item list-removed - Liste von Pakete die im Paketrepository sind aber nicht im git.
\item list-tools - Liste der Tools.
\item outdated - Sucht nach veralteten Paketen im Repository im Vergleich zum git Repository.
\item pkgmod - Modifiziert ein Buildpaket.
\item pkgrel - Zaehlt die pkgrel in einem Paket hoch.
\item prep - Aufraeumen des PKGBUILD Datei-Styles und Fehlersuche.
\item sitesync - Synchronisiert zwischein einer lokalen Kopie des Paketrepositories und der Remote.
\item size-hunt - Sucht nach grossen Paketen.
\item source-backup - Backup von package source Dateien.
\end{itemize}

\section{Beitragen zum BlackArch Repository}
Dieser Abschnitt zeigt, wie Beitraege im BlackArch Linux Projekt gemacht werden.
wir akzeptieren Pull Requests jeglicher Groesse, von kleinen Tippfehler-Korrekturen 
bis zu neuen Paketen. \\Fuer Hilfe, Vorschlage oder Fragen Kontaktiere uns. 
\\\\
Jeder ist willkommen. Alle Beitraege werden geschaetzt.

\subsection{Benoetigte Tutorials}
Bitte lies folgende Tutorials bevor du mitmachst:
\begin{itemize}
\item
\href{https://wiki.archlinux.org/index.php/Arch\_Packaging\_Standards)}{Arch
Packaging Standards}
\item \href{https://wiki.archlinux.org/index.php/Creating\_Packages}{Paketerzeugung}
\item \href{https://wiki.archlinux.org/index.php/PKGBUILD}{PKGBUILD}
\item \href{https://wiki.archlinux.org/index.php/Makepkg}{Makepkg}
\end{itemize}

\subsection{Schritte zum Mitmachen}
Um Aenderungen zum BlackArchLinux Projekt zu submitten, folge diesen Schritten:
steps:
\begin{enumerate}
\item Fork das Repository von
\url{https://github.com/BlackArch/blackarch}
\item Hacke die benoetigten Dateien (z.B. PKGBUILD, .patch files, usw).
\item Committe deine Aenderungen.
\item Pushe deine Aenderungen.
\item Bitte uns darum deine changes zu mergen, am liebsten durch einen Pull Request.
\end{enumerate}

\subsection{Beispiel}
The following example demonstrates submitting a new package to the BlackArch
project. We use \href{https://wiki.archlinux.org/index.php/yaourt}{yaourt}
(you can use pacaur as well) to fetch a pre-existing PKGBUILD file for
\textbf{nfsshell} from the \href{https://aur.archlinux.org/}{AUR} and adjust it
according to our needs.

\subsubsection{Fetch PKGBUILD}
Fetch the \textit{PKGBUILD} file using yaourt or pacaur:
\begin{lstlisting}
  user@blackarchlinux $ yaourt -G nfsshell
  ==> Download nfsshell sources
  x LICENSE
  x PKGBUILD
  x gcc.patch
  user@blackarchlinux $ cd nfsshell/
\end{lstlisting}

\subsubsection{Clean up PKGBUILD}
Clean up the \textit{PKGBUILD} file and save some time:
\begin{lstlisting}
  user@blackarchlinux nfsshell $ ./blarckarch/scripts/prep PKGBUILD
  cleaning 'PKGBUILD'...
  expanding tabs...
  removing vim modeline...
  removing id comment...
  removing contributor and maintainer comments...
  squeezing extra blank lines...
  removing '|| return'...
  removing leading blank line...
  removing $pkgname...
  removing trailing whitespace...
\end{lstlisting}

\subsubsection{Adjust PKGBUILD}
Adjust the \textit{PKGBUILD} file:
\begin{lstlisting}
  user@blackarchlinux nfsshell $ vi PKGBUILD
\end{lstlisting}

\subsubsection{Build the package}
Build the package:
\begin{lstlisting}user@blackarchlinux nfsshell $ makepkg -sf
==> Making package: nfsshell 19980519-1 (Mon Dec  2 17:23:51 CET 2013)
==> Checking runtime dependencies...
==> Checking buildtime dependencies...
==> Retrieving sources...
-> Downloading nfsshell.tar.gz...
% Total    % Received % Xferd  Average Speed   Time    Time     Time
CurrentDload  Upload   Total   Spent    Left  Speed100 29213  100 29213    0
0  48150      0 --:--:-- --:--:-- --:--:-- 48206
-> Found gcc.patch
-> Found LICENSE
...
<lots of build process and compiler output here>
...
==> Leaving fakeroot environment.
==> Finished making: nfsshell 19980519-1 (Mon Dec  2 17:23:53 CET 2013)
\end{lstlisting}

\subsubsection{Install and test the package}
Install and test the package:
\begin{lstlisting}
  user@blackarchlinux nfsshell $ pacman -U nfsshell-19980519-1-x86_64.pkg.tar.xz
  user@blackarchlinux nfsshell $ nfsshell # test it
\end{lstlisting}

\subsubsection{Add, commit and push package}
Add, commit and push the package
\begin{lstlisting}user@blackarchlinux nfsshell $ cd /blackarchlinux/packages
user@blackarchlinux ~/blackarchlinux/packages $ mv ~/nfsshell .
user@blackarchlinux ~/blackarchlinux/packages $ git commit -am nfsshell && git push
\end{lstlisting}

\subsubsection{Erzeugt a pull request}
Erzeugt a pull request on \href{https://github.com/}{github.com}
\begin{lstlisting}
  firefox https://github.com/<contributor>/blackarchlinux
\end{lstlisting}

\subsubsection{Adding a remote for upstream}
A smart thing to do if you're working upstream and on a fork is to pull your own fork and add the main ba repo as a remote
\begin{lstlisting}
  user@blackarchlinux ~/blackarchlinux $ git remote -v
  origin <the url of your fork> (fetch)
  origin <the url of your fork> (push)
  user@blackarchlinux ~/blackarchlinux $ git remote add upstream https://github.com/blackarch/blackarch
  user@blackarchlinux ~/blackarchlinux $ git remote -v
  origin <the url of your fork> (fetch)
  origin <the url of your fork> (push)
  upstream https://github.com/blackarch/blackarch (fetch)
  upstream https://github.com/blackarch/blackarch (push)
\end{lstlisting}

By default, git should push straight to origin, but make sure your git config is
configured correctly. This won't be an issue unless you have commit rights as
you won't be able to push upstream without them.

If you do have the ability to commit, you might have more success using
\textit{git@github.com:blackarch/blackarch.git} but it's up to you.

\subsection{Requests}
\begin{enumerate}
\item Don't add \textbf{Maintainer} or \textbf{Contributor} comments to
\textit{PKGBUILD} files. Add maintainer and contributor names to the
AUTHORS section of BlackArch guide.
\item For the sake of consistency, please follow the general style of the other
\textit{PKGBUILD} files in the repo and use two-space indentation.
\end{enumerate}

\subsection{General tips}
\href{http://wiki.archlinux.org/index.php/Namcap}{namcap} can check packages for
errors.

%------------------%
%  Chapter 4       %
%------------------%

\chapter{Tools Guide}
Coming soon...

\section{Coming Soon}
Coming soon...

%%% APPENDIX %%%
\appendix
\include{appendix}

\end{document}

%%% EOF %%%
