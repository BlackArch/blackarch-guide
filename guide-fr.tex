%%%%%%%%%%%%%%%%%%%%%%%%%%%%%%%%%%%%%%%%%%%%%%%%%%%%%%%%%%%%%%%%%%%%%%%%%%%%%%%%
%                                                                              %
% BlackArch Linux Guide                                                        %
%                                                                              %
%%%%%%%%%%%%%%%%%%%%%%%%%%%%%%%%%%%%%%%%%%%%%%%%%%%%%%%%%%%%%%%%%%%%%%%%%%%%%%%%


%%% BEGIN %%%
\documentclass[a4paper, oneside, 11pt]{book}


%%% INCLUDES %%%
\renewcommand{\familydefault}{\sfdefault}
\usepackage{verbatim}
\usepackage[utf8]{inputenc}
\usepackage{geometry}
\usepackage{array}
\usepackage{graphicx}
\usepackage{supertabular}
\usepackage{verbatim}
\usepackage{pstricks}
\usepackage{fancyhdr}
\usepackage{fancyvrb}
\usepackage{tocloft}
\usepackage{color}
\usepackage{enumerate}
\usepackage[french]{babel}
\usepackage[
pdftitle={BlackArch Linux, The BlackArch Linux Guide},
pdfsubject={BlackArch Linux, The BlackArch Linux Guide},
pdfauthor={BlackArch Linux, BlackArch Linux},
pdfkeywords={BlackArch Linux, Penetration Testing, Security, Hacking, Linux},
pdfcenterwindow=true
]{hyperref}


%%% LAYOUT %%%
\setlength{\parindent}{0em}
\setlength{\parskip}{1.5ex plus0.5ex minus0.5ex}
\geometry{left=2.5cm, textwidth=16cm, top=3cm, textheight=25cm, bottom=3cm}
\widowpenalty=2000
\clubpenalty=1000
\frenchspacing
\sloppy
\pagecolor{black}
\color{white}
\hypersetup{pdftex=true, colorlinks=true, breaklinks=true, linkcolor=red,
menucolor=red, pagecolor=red, urlcolor=red}
\setcounter{tocdepth}{10}
\setcounter{secnumdepth}{10}


%%% HEADER / FOOTER %%%
\setlength{\headheight}{1.0cm}
\setlength{\headsep}{1.0cm}
\lhead{{\includegraphics[width=1cm,height=1cm]{logo.png}}}
\rhead{The BlackArch Linux Guide}
%\lfoot{\small{foo}}
%\rfoot{\small{bar}}


%%% OWN MACROS %%%
% put own macros here, like renewcommand newcommand etc.


%%% DOCUMENT %%%
\begin{document}
\pagestyle{empty}
\begin{center}
\begin{figure}[htbp]
\centering
\vspace{1cm}
\Huge{\textbf{BlackArch \color{red}Linux}}\\
\vspace{2cm}
\includegraphics[width=8cm]{logo.png}
\label{fig:logo}
\end{figure}
\vspace{1cm}
\Huge{\textbf{Guide de BlackArch}}\\
\vspace{1cm}
\Large{\color{red}http://www.blackarch.org/}\\
\vspace{0.5cm}
\Large{\today}
\end{center}
\newpage
\tableofcontents
\newpage
\pagestyle{fancy}

%%%%%%%%%%%%%%%%%%%%%%%%%%%%%%%%%%%%%%%%%%%%%%%%%%%%%%%%%%%%%%%%%%%%%%%%%%%%%%%%

\chapter{Introduction}

\section{Qu'est-ce que BlackArch Linux?}
\href{http://www.blackarch.org}{BlackArch Linux} est une légère expansion d'Arch 
Linux destinée aux professionnels de la sécurité informatique.
\\\\
L'ensemble d'outils est distribué tel un  
\href{https://wiki.archlinux.org/index.php/Unofficial\_User\_Repositories}
{dépôt non-officiel} d'Arch Linux, vous permettant d'installer les composantes de
BlackArch sur une installation d'Arch Linux existante. Les paquets peuvent y être
installés individuellement ou par catégorie.
\\\\
Le dépot logiciel de Black Arch contient plus de 650 outils et ce nombre augmente
sans cesse. Tous ces outils sont minutieusement testés avant d'être ajouté aux
dépots, afin de garantir leurs bon fonctionnement.

%\section{The story of BlackArch Linux}
%foo bar

%\section{Supported platforms}
%foo bar

\section{S'impliquer}
Il est possible d'entrer en contact avec l'équipe Black Arch à l'aide de ces
informations:
\\\\
Web: \url{http://www.blackarch.org/}
\\\\
Courriel: \href{mailto:blackarchlinux@gmail.com}{blackarchlinux@gmail.com}
\\\\
IRC: \url{irc://irc.freenode.net/blackarch}

%%%%%%%%%%%%%%%%%%%%%%%%%%%%%%%%%%%%%%%%%%%%%%%%%%%%%%%%%%%%%%%%%%%%%%%%%%%%%%%%

\chapter{Guide d'utilisation}

\section{Installation}
La section suivante démontre comment configurer le dépôt de paquets de Black 
Arch ainsi que la procédure à suivre afin d'installer des logiciels à partir de 
celle-ci. BlackArch supporte l'installation de fichiers binaires précompilés 
ainsi que l'installation à l'aide du code source original des paquets.
\\\\
BlackArch est compatible avec une installation de base d'Arch Linux. Il s'agira 
d'un dépôt non-officiel de paquets. Si vous désirez une image ISO à la place,
visitez la section \href{http://www.blackarch.org/download.html#iso}{Live ISO}.
\\\\

\subsection{Installer le dépôt}
Il suffit de 6 étapes afin d'installer le dépot de BlackArch sur une installation
existante de Arch Linux. Vous devez suivre la recette dans l'ordre.
You must follow the instuctions in order. N'ajoutez pas \textbf{blackarch} 
à votre fichier \textit{pacman.conf} sans avoir complété les étapes 0 à 2.
\begin{enumerate} \itemsep4pt
\item Si vous avez déjà installé BlackArch et qu'il y existe déjà une entrée
\textbf{[blackarch]} dans \textit{/etc/pacman.conf}, vous devrez la supprimer ou
la mettre en commentaire, et ensuite exécuter la commande \textit{pacman -Syy}.
\item Exécutez les commandes suivantes en tant que root. Ces commande
téléchargeront et installeront les clés nécessaire pour garantir l'authenticité
des paquets via une signature digitale.
{\small
\color{gray}
\begin{verbatim}
wget -q http://blackarch.org/keyring/blackarch-keyring.pkg.tar.xz{,.sig}
gpg --keyserver hkp://pgp.mit.edu --recv 4345771566D76038C7FEB43863EC0ADBEA87E4E3
gpg --keyserver-o no-auto-key-retrieve --with-f blackarch-keyring.pkg.tar.xz.sig
pacman-key --init
rm blackarch-keyring.pkg.tar.xz.sig
pacman --noc -U blackarch-keyring.pkg.tar.xz
\end{verbatim}
}
\item Si désiré, vérifiez l'authenticité des clés autant de sources que vous 
désirez afin de vous convaincre de leurs validités.
\item Ajoutez les lignes suivantes à votre fichier \textit{/etc/pacman.conf}:
{\small
\color{gray}
\begin{verbatim}
[blackarch]
Server = <site_miroir>/$repo/os/$arch
\end{verbatim}
}
Replacez \textit{\textless site\_mirroir\textgreater} par le miroir officiel
de votre choix. La liste des miroirs est disponible sur
\href{http://www.blackarch.org/}{notre site web}.

\item Exécutez:
{\small
\color{gray}
\begin{verbatim}
# pacman -Syyu
\end{verbatim}
}
\end{enumerate}

\subsection{Installation de paquets}
Vous pouvez désormais installer des outils directement du dépôt de BlackArch
\begin{enumerate}
\item Afin d'obtenir la liste de tous les outils disponibles exécutez
{\small
\color{gray}
\begin{verbatim}
# pacman -Sgg | grep blackarch | cut -d' ' -f2 | sort -u
\end{verbatim}
}
\item Afin d'installer tous les outils, exécutez
{\small
\color{gray}
\begin{verbatim}
# pacman -S blackarch
\end{verbatim}
}
\item Afin d'installer une catégorie complète d'outils, exécutez
{\small
\color{gray}
\begin{verbatim}
# pacman -S blackarch-<nom de la catégorie>
\end{verbatim}
}
\item Afin d'obtenir la liste des catégories de paquets disponibles, exécutez
{\small
\color{gray}
\begin{verbatim}
# pacman -Sg | grep blackarch
\end{verbatim}
}
\end{enumerate}

\subsection{Installation des paquets depuis le code source}
Il existe une méthode alternative à l'installation de paquets sous forme binaire.
BlackArch vous offre la possibilité d'obtenir le code source directement du 
dépôt officiel du paquet désiré, et BlackArch le compilera pour vous. Vous 
trouverez les fichiers PKGBUILD sur notre 
\href{https://github.com/BlackArch/blackarch/tree/master/packages}{github}. Afin
d'obtenir le dépôt en entier, vous pourrez utiliser l'outil 
\href{https://github.com/BlackArch/blackman}{blackman}.
\begin{itemize} \itemsep10pt
\item Vous devez premièrement installer blackman. Si vous avez déjà configuré
le dépôt de paquets de BlackArch sur votre système, tel qu'indiqué au début de 
ce document, vous vous pouvez installer blackman à l'aide de pacman à l'aide de
cette commande:
{\small
\color{gray}
\begin{verbatim}
pacman -S blackman
\end{verbatim}
}
\item Vous pouvez aussi installer blackman directement du code source:
{\small
\color{gray}
\begin{verbatim}
mkdir blackman
cd blackman
wget https://raw2.github.com/BlackArch/blackarch/master/packages/blackman/PKGBUILD
# Make sure the PKGBUILD has not been maliciously tampered with.
makepkg -s
\end{verbatim}
}
\item Afin de télécharger et compiler un paquet:
{\small
\color{gray}
\begin{verbatim}
$ sudo blackman -i package
\end{verbatim}
}
\item Afin de télécharger, compiler, et installer une catégorie entière:
{\small
\color{gray}
\begin{verbatim}
$ sudo blackman -g group
\end{verbatim}
}
\item Afin de télécharger, compiler, et installer tous les outils présents
dans le dépôt de paquets de BlackArch:
{\small
\color{gray}
\begin{verbatim}
$ sudo blackman -a
\end{verbatim}
}
\item Afin d'obtenir la liste de toutes les catégories de paquets de BlackArch:
{\small
\color{gray}
\begin{verbatim}
$ blackman -l
\end{verbatim}
}
\item Si vous voulez ensuite obtenir la liste de tous les paquets compris dans
une catégorie particulière, vous pouvez obtenir la liste de ceux-ci ainsi qu'une
courte description pour chaque paquets à l'aide de la commande suivante:
{\small
\color{gray}
\begin{verbatim}
$ blackman -p <nom de la catégorie>
\end{verbatim}
}
\end{itemize}

\subsection{Installation d'un LiveCD, netinstall, ou d'Arch Linux}
Vous pouvez installer Black Arch d'un de nos images ISO, que ce soit un LiveCD
ou un netcd. Vous les trouverez sur 
\href{http://www.blackarch.org/download.html#iso}{notre site}. Les étapes 
suivantes sont nécéssaire afin de créer une image bootable:
\begin{itemize}
\item Installez le paquet blackarch-install-scripts:
{\small
\color{gray}
\begin{verbatim}
$ sudo pacman -S blackarch-install-scripts
\end{verbatim}
}
\item Ensuite exécutez:
{\small
\color{gray}
\begin{verbatim}
$ sudo blackarch-install
\end{verbatim}
}
\end{itemize}

%%%%%%%%%%%%%%%%%%%%%%%%%%%%%%%%%%%%%%%%%%%%%%%%%%%%%%%%%%%%%%%%%%%%%%%%%%%%%%%%


\chapter{Guide de développement}

\section{Le Arch Build System et les dépôts de paquets}

Les fichiers PKGBUILD sont de simples scripts permettant d'installer et de 
compiler et installer un programme sur plusieurs environnements différents. 
Chaque PKGBUILD indique à la commande makepkg(1) comment créer un paquet valide
qui pourra ensuite être installé. Les fichiers PKGBUILD sont écrits en bash.

Pour plus d'information, il est suggéré de lire ces pages:
\begin{itemize}
\item \href{http://wiki.archlinux.fr/Standard_paquetage}{Arch Wiki FR:
Standard paquetage}
\item \href{http://wiki.archlinux.fr/Makepkg}{Arch Wiki FR: Makepkg}
\item \href{http://wiki.archlinux.fr/PKGBUILD}{Arch Wiki FR: PKGBUILD}
\end{itemize}

\section{Le standard de paquetage de BlackArch}

Afin de garde les choses simples, nos PKGBUILDs sont similaires avec ceux du
Arch User Repository, à quelques différences près. Tous les paquets doivent
être membre du groupe \textit{blackarch}. Il est possible qu'un paquet soit
membre de plus d'un groupe.

\subsection{Groupes}

Afin de permettre aux utilisateurs d'installer certaines catégories de paquets
exclusivement, ceux-ci ont étés répertoriés en plusieurs groupes. Ces groupes
permettent aux utilisateurs d'exécuter la commande 
\textit{pacman -S \textless nom\_du\_groupe \textgreater } afin d'installer tous
 les paquets compris dans ce groupe.

\subsubsection{blackarch}

En théorie, le groupe \textit{blackarch} doit contenir tous les paquets qu'offre
BlackArch. La majorité des paquets devraient être membre de ce groupe.

\subsubsection{blackarch-anti-forensic}

Groupe contenant les paquets utilisés pour contrer les tentatives 
d'investigation informatique de systèmes. Vous y trouverez des paquets offrants
des solutions de cryptage de données, stéganographie, modification d'attributs 
de fichiers, et tout autre type de logiciels permettant la modification d'un
système de fichiers dans le but d'y masquer de l'information.

Exemples: luks, TrueCrypt, Timestomp, dd, ropeadope, secure-delete

\subsubsection{blackarch-automation}

Groupe contenant les paquets utilisés pour l'automation de plusieurs tâches.
Puisque les paquets présents dans ce groupe varient énormément en 
fonctionnalités, il vous est suggéré d'y consulter directement les paquets
afin de vous informer sur leurs capacités.

Exemples: blueranger, tiger, wiffy

\subsubsection{blackarch-backdoor}


Groupe contenant les paquets reliés à l'exploitation ou à l'ouverture de 
backdoors sur de systèmes vulnérables.

Exemples: backdoor-factory, rrs, weevely

\subsubsection{blackarch-binary}

Groupe contenant les paquets reliés aux fichiers binaires. Vous y trouverez des
paquets permettant l'extraction d'archives d'un firmware, des logiciels de
rétro-ingénierie, logiciels aidant à l'exploitation de fichiers binaires, et
autres.

Exemples: binwally, packerid, hex2bin, binwalk

\subsubsection{blackarch-bluetooth}

Groupe contenant les paquets aidant à l'exploitation du standard Bluetooth
(802.15.1).

Exemples: ubertooth, tbear, redfang

\subsubsection{blackarch-code-audit}

Groupe contenant les paquets à fin d'audit de code source. Ces paquets analysent
statiquement le code source de votre projet afin d'y déceler des vulnérabilités.

Exemples: flawfinder, pscan

\subsubsection{blackarch-cracker}

Groupe contenant les paquets utilisés afin de tester la robustesses de multiples
fonctions cryptographiques.

Exemples: hashcat, john, crunch

\subsubsection{blackarch-crypto}

Groupe contenant tous types de paquets ayant rapport de proche ou de loin à la
cryptographie, que ce soit de l'analyse de ou au craquage de hache.

Exemples: ciphertest, xortool, sbd

\subsubsection{blackarch-database}

Groupe contenant les paquets reliés à l'exploitation de bases de données.

Exemples: metacoretex, blindsql

\subsubsection{blackarch-debugger}

Groupe contenant les paquets permettant à l'usager d’investiguer le
fonctionnement interne d'un programme lors de son utilisation.

Exemples: radare2, shellnoob

\subsubsection{blackarch-decompiler}

Groupe contenant les paquets qui tentent d'effectuer de la rétro-ingénierie sur
un programme compilé.

Exemples: flasm, jd-gui

\subsubsection{blackarch-defensive}

Groupe contenant les paquets utilisés afin de protéger un système des logiciels
et usagers malveillants.

Exemples: arpon, chkrootkit, sniffjoke

\subsubsection{blackarch-disassembler}

Ce groupe est fortement similaire à \textit{blackarch-decompiler} dans le sens
que ces deux groupes tentent d'effectuer une rétro-ingénierie générique de
fichiers exécutables, mais les paquets compris dans
\textit{blackarch-disassembler} en extrairont le code assembleur à la place d'un
pseudocode.

Exemples: inguma, radare2

\subsubsection{blackarch-dos}

Groupe contenant les paquets aidant aux attaques de types \textit{déni de service}.

Exemples: 42zip, nkiller2

\subsubsection{blackarch-drone}

Groupe contenant les paquets aidant à l'entretien et l'utilisation de drones.

Exemples: meshdeck, skyjack

\subsubsection{blackarch-exploitation}

Groupe contenant les paquets utilisés lors de l'exploitation de multiples 
programmes et services.

Exemples: armitage, metasploit, zarp

\subsubsection{blackarch-fingerprint}

Groupe contenant les paquets utilisés à fin de reconnaissance réseautique.

Exemples: dns-map, p0f, httprint

\subsubsection{blackarch-firmware}

Groupe contenant les paquets utilisés lors de l'exploitation de firmware.

Exemples: None yet, amend asap.

\subsubsection{blackarch-forensic}

Groupe contenant les paquets utilisés lors d'investigation informatique portant
sur le recouvrement de données potentiellement effacés.

Exemples: aesfix, nfex, wyd

\subsubsection{blackarch-fuzzer}

Groupe contenant les paquets aidant au fuzzing. 

Exemples: msf, mdk3, wfuzz

\subsubsection{blackarch-hardware}

Groupe contenant les paquets aidant à l'exploitation de matériel physique.

Exemples: arduino, smali

\subsubsection{blackarch-honeypot}

Groupe contenant les paquets agissant comme des \textit{honeypots}, c'est-à-dire
des services sécuritaire, mais qui laissent croire à un usager malveillant
autrement afin d'observer son comportement.

Exemples: artillery, bluepot, wifi-honey

\subsubsection{blackarch-keylogger}

Groupe contenant les paquets utilisés lors d'installation de \textit{keylogging}
, soit l'enregistrement de touches clavier sur un système qui pourrait être
utilisé par un usager tiers afin d'en soutirer des informations confidentielles.

Exemples: klogger, logkeys, xspy

\subsubsection{blackarch-malware}

Groupe contenant tous types de paquets ayant un lien au \textit{malware}, que ce
soit à l'utilisation de ceux-ci ou à leur détection.

Exemples: malwaredetect, peepdf, yara

\subsubsection{blackarch-misc}

Groupe contenant les paquets ne semblant pas prendre part dans aucunes autres
catégories.

Exemples: oh-my-zsh-git, winexe, stompy

\subsubsection{blackarch-mobile}

Groupe contenant les paquets aidant aux activités d'entretien et d'audits 
d'applications, de développement d'application, et de matériel mobile..

Exemples: android-sdk-platform-tools, android-udev-rules

\subsubsection{blackarch-networking}

Groupe contenant les paquets aidant à multiples tâches reliés à la réseautique.

Exemples: Anything pretty much

\subsubsection{blackarch-nfc}

Groupe contenant les paquets utilisant la technologie \textit{NFC} (near-field
communications).

Exemples: nfcutils

\subsubsection{blackarch-packer}

Groupe contenant les paquets utilisant de loin ou de près les \textit{packers},
soit des logiciels qui créent des applications contenant un programme 
malveillant caché à l'intérieur.

Exemples: packerid

\subsubsection{blackarch-proxy}

Groupe contenant les paquets agissant tel un proxy, soit un logiciel redirigeant
le trafic internet afin de l'analyser ou de le modifier.

Exemples: burpsuite, ratproxy, sslnuke

\subsubsection{blackarch-recon}

Groupe contenant les paquets servant à la recherche de services exploitable dans
le monde réel.

Exemples: canri, dnsrecon, netmask

\subsubsection{blackarch-reversing}

Groupe contenant les paquets de décompilation, de désassemblage, et tout autres
types de programmes ayant à voir avec la rétro-ingénierie. 

Exemples: capstone, radare2, zerowine

\subsubsection{blackarch-scanner}

Groupe contenant les paquets utilisés afin de scanner multiples systèmes afin
d'y déceler des vulnérabilités actives.

Exemples: scanssh, tiger, zmap

\subsubsection{blackarch-sniffer}

Groupes contenant les paquets aidant dans l'analyse de trafic réseau.

Exemples: hexinject, pytactle, xspy

\subsubsection{blackarch-social}

Groupe contenant les paquets attaquant principalement les sites de réseaux
sociaux.

Exemples: jigsaw, websploit

\subsubsection{blackarch-spoof}

Groupe contenant les paquets qui tentent de dissimuler un attaquant, de tel que
la victime ne puisse le percevoir comme un acteur malicieux.

Exemples: arpoison, lans, netcommander

\subsubsection{blackarch-threat-model}

Groupe contenant les paquets utilisé pour enregistré et créer des rapports sur
plusieurs modèles de menaces pouvant se produire lors d'un scénario spécifique.

Exemples: magictree

\subsubsection{blackarch-tunnel}

Groupe contenant les paquets utilisés afin d'encapsuler le contenu réseau via un
tiers parti, afin de dissimuler la provenance initiale de données. 

Exemples: ctunnel, iodine, ptunnel

\subsubsection{blackarch-unpacker}

Groupe contenant les paquets utilisés lors de l'extraction de logiciel
malveillants de possibles logiciels \textit{packés} ou obfuscé.

Exemples: js-beautify

\subsubsection{blackarch-voip}

Groupe contenant les paquets utilisés afin d'effectuer des actions ou l'analyse
de la téléphonie VoIP.

Exemples: iaxflood, rtp-flood, teardown

\subsubsection{blackarch-webapp}

Groupe contenant les paquets effectuant des opérations de validation de sécurité
sur des applications accessible via l'internet, afin d'en garantir leurs 
robustesse. 

Exemples: metoscan, whatweb, zaproxy

\subsubsection{blackarch-windows}

Groupe contenant les paquets Windows qui s'exécuteront via Wine.

Exemples: 3proxy-win32, pwdump, winexe

\subsubsection{blackarch-wireless}

Groupe contenant les paquets utilisés lors d'audits de sécurité sur une 
infrastructure sans-fil.

Exemples: airpwn, mdk3, wiffy

\section{Structure du dépôt}

Vous pouvez trouver le dépot principal de BlackArch à 
\href{https://github.com/BlackArch/blackarch}
{https://github.com/BlackArch/blackarch}. Il existe aussi plusieurs dépots de
code à \href{https://github.com/BlackArch}{https://github.com/BlackArch}.

À l'intérieur du dépôt principal, vous trouverez les trois dossiers principaux,
soit :

\begin{itemize}
\item docs - Contient la documentation du projet
\item packages - Contient les fichiers PKGBUILD
\item scripts - Contient multiples scripts d'automation
\end{itemize}

\subsection{Scripts}

Voici une référence des scripts que vous trouverez dans le dossier
\verb|scripts/| :

\begin{itemize}
\item baaur - Éventuellement, ce script enverra nos paquets à la banque de donné
du AUR.
\item babuild - Construit un paquet.
\item bachroot - Prépare un chroot afin d'effectuer des tests.
\item baclean - Néttoie les enciens fichiers .pkg.tar.xz du dépôt.
\item baconflict - Éventuellement, ce script remplacera \verb|scripts/conflicts|.
\item bad-files - Trouve des fichiers erronés dans les paquets construits.
\item balock - Crée ou termine le lock sur le dépôt de paquets.
\item banotify - Indique IRC des push de paquets.
\item barelease - Envoie les paquets au dépôt.
\item baright - Affiche les informations de copyright de BlackArch.
\item basign - Signe les paquets.
\item basign-key - Signe une clé.
\item blackman - Opère sensiblement comme pacman mais construits les paquets à
partir de git. (à ne pas confondre avec le programme Blackman, écrit par nzr)
\item check-groups - Valide les groupes.
\item checkpkgs - Valide les paquets.
\item conflicts - Valide les paquets pour des conflits.
\item dbmod - Modifie la base de donnée d'un paquet.
\item depth-list - Crée une liste de dépendences ordonnée par profondeur.
\item deptree - Crée une arbre de dépendance, ne listant que les paquets obtenu
via BlackArch.
\item get-blackarch-deps - Affiche une liste des dépendances de paquets obtenu 
via BlackArch pour un paquet donné.
\item get-official - Obtien les paquets officiels.
\item list-loose-packages - Liste les paquets qui ne se trouvent dans aucun
groupes et ne sont pas des dépendances pour d'autres paquets.
\item list-needed - Liste les dépendances manquantes.
\item list-removed - Liste les paquets qui sont dans le dépôt mais pas dans le
répertoire git.
\item list-tools - Liste les outils.
\item outdated - Cherche les paquets qui ne sont plus à jour lorsque comparés 
avec le répertoire git.
\item pkgmod - Modifie un paquets de construction.
\item pkgrel - Incrémente le pkgrel d'un paquet.
\item prep - Nettoie un fichier PKGBUILD de ses erreurs de style et autre.
\item sitesync - Synchronise entre une copie locale et sa version distante.
\item size-hunt - Cherche de gros paquets.
\item source-backup - Sauvegarde les fichiers sources d'un paquets.
\end{itemize}

\section{Contribuer au dépôt}
Cette section indique comment contribuer au projet BlackArch Linux. Nous
acceptons les pull requests de toutes tailles, que ce soit de petites
modifications ou un ajout de paquet.\\Pour recevoir de l'aide, n'hésitez pas à
nous contacter.
\\\\
Tout le monde est bienvenue à contrivuer. Toute contribution sera appréciée.

\subsection{Tutoriels requis.}
Veuillez lire les documents suivants avant de contribuer:
\begin{itemize}
\item \href{http://wiki.archlinux.fr/Standard_paquetage}{Arch Wiki FR:
Standard paquetage}
\item \href{http://wiki.archlinux.fr/Makepkg}{Arch Wiki FR: Makepkg}
\item \href{http://wiki.archlinux.fr/PKGBUILD}{Arch Wiki FR: PKGBUILD}
\end{itemize}

\subsection{Étapes à suivre afin de contribuer}
Afin de contribuer au projet BlackArch Linux, suivez ces étapes:
\begin{enumerate}
\item Forkez le dépôt sur github à partir de
\url{https://github.com/BlackArchLinux/blackarchlinux}.
\item Effectuez vos modifications sur les fichiers nécéssaires (e.g. PKGBUILD,
.patch files, etc).
\item Commitez vos changements.
\item Poussez vos changements.
\item Créez un pull request vers le dépôt.
\end{enumerate}

\subsection{Exemple}
L'exemple suivant démontre la soumissions d'un nouveau paquets au projet
BlackArch. Nous utilisons \href{https://wiki.archlinux.org/index.php/yaourt}{yaourt}
(vous pouvez aussi utiliser pacaur) afin d'obtenir le PKGBUILD existant de
\textbf{nfsshell}, et nous pouvons ensuite l'éditer, au besoin.

\subsubsection{Obtenir le PKGBUILD}
En utilisant yaourt ou pacaur, obtenir \textit{PKGBUILD}:
{\small
\color{gray}
\begin{verbatim}
user@blackarchlinux $ yaourt -G nfsshell
==> Download nfsshell sources
x LICENSE
x PKGBUILD
x gcc.patch
user@blackarchlinux $ cd nfsshell/
\end{verbatim}
}

\subsubsection{Nettoiement du PKGBUILD}
Nettoyer le fichier \textit{PKGBUILD} à l'aide de \textit{prep}, afin de sauver
du temps:
{\small
\color{gray}
\begin{verbatim}
user@blackarchlinux nfsshell $ ./blarckarch/scripts/prep PKGBUILD
cleaning 'PKGBUILD'...
expanding tabs...
removing vim modeline...
removing id comment...
removing contributor and maintainer comments...
squeezing extra blank lines...
removing '|| return'...
removing leading blank line...
removing $pkgname...
removing trailing whitespace...
\end{verbatim}
}

\subsubsection{Ajustement du PKGBUILD}
Ajustez le fichier \textit{PKGBUILD}:
{\small
\color{gray}
\begin{verbatim}
user@blackarchlinux nfsshell $ vi PKGBUILD
\end{verbatim}
}

\subsubsection{Construire le paquet}
Construire le paquet:
{\small
\color{gray}
\begin{verbatim}
user@blackarchlinux nfsshell $ makepkg -sf
==> Making package: nfsshell 19980519-1 (Mon Dec 2 17:23:51 CET 2013)
==> Checking runtime dependencies...
==> Checking buildtime dependencies...
==> Retrieving sources...
-> Downloading nfsshell.tar.gz...
% Total % Received % Xferd Average Speed Time Time Time
CurrentDload Upload Total Spent Left Speed100 29213 100 29213 0
0 48150 0 --:--:-- --:--:-- --:--:-- 48206
-> Found gcc.patch
-> Found LICENSE
...
<beaucoup d'output du processus de build et compilation ici>
...
==> Leaving fakeroot environment.
==> Finished making: nfsshell 19980519-1 (Mon Dec 2 17:23:53 CET 2013)
\end{verbatim}
}

\subsubsection{Installation et test du paquet}
Installation et test du paquet:
{\small
\color{gray}
\begin{verbatim}
user@blackarchlinux nfsshell $ pacman -U nfsshell-19980519-1-x86_64.pkg.tar.xz
user@blackarchlinux nfsshell $ nfsshell # test it
\end{verbatim}
}

\subsubsection{Ajout, commit et push du paquet}
Ajoute, commit et push du paquet:
{\small
\color{gray}
\begin{verbatim}
user@blackarchlinux nfsshell $ cd /blackarchlinux/packages
user@blackarchlinux ~/blackarchlinux/packages $ mv ~/nfsshell .
user@blackarchlinux ~/blackarchlinux/packages $ git commit -am nfsshell && git push
\end{verbatim}
}

\subsubsection{Création d'un pull request}
Créez un pull request de votre fork vers
 \href{https://github.com/BlackArchLinux/blackarchlinux}{le projet officiel}:
{\small
\color{gray}
\begin{verbatim}
firefox https://github.com/<contributor>/blackarchlinux
\end{verbatim}
}

\subsubsection{Ajout d'un remote pour l'upstream}
Une fois votre modification faite, il serait une bonne idée d'ajouter le projet
officiel comme remote, afin d'ajouter les modifications au projet à votre fork.
Cela vous permettra de synchroniser votre fork avec BlackArch.
{\small
\color{gray}
\begin{verbatim}
user@blackarchlinux ~/blackarchlinux $ git remote -v
origin <the url of your fork> (fetch)
origin <the url of your fork> (push)
user@blackarchlinux ~/blackarchlinux $ git remote add upstream https://github.com/blackarch/blackarch
user@blackarchlinux ~/blackarchlinux $ git remote -v
origin <the url of your fork> (fetch)
origin <the url of your fork> (push)
upstream https://github.com/blackarch/blackarch (fetch)
upstream https://github.com/blackarch/blackarch (push)
\end{verbatim}
}
Par défaut, git devrait push directement à l'origine, mais assurez-vous que
votre configuration git est correcte. Cela ne sera pas un problème à moins
d'avoir les droits de commit puisque vou sne pourrez pas push upstream sans
cela.

Si vous avez les droits de commit, vous aurez possiblement plus de succès en
utilisant \textit{git@github.com:blackarch/blackarch.git}. Le choix est votre.

\subsection{Requêtes}
\begin{enumerate}
\item N'ajoutez pas de commentaire \textbf{Maintainer} ou \textbf{Contributor}
au fichiers \textit{PKGBUILD}. Ajoutez ces commentaires à la section
\textbf{AUTHORS} du guide BlackArch.
\item Afin de conserver la consistance du projet, veuillez suivre les règles
générales de style des autres fichiers \textit{PKGBUILD} en utilisant un
indentation de deux espaces.
\end{enumerate}

\subsection{Conseils généraux}
\href{http://wiki.archlinux.org/index.php/Namcap}{namcap} peut trouver des
erreurs dans vos paquets. Il est recommendé de l'utiliser avant de soumettre vos
modifications.

%%% APPENDIX %%%
\appendix
\include{appendix}

\end{document}

%%% EOF %%%